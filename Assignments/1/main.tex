\documentclass[letterpaper,11pt,twoside]{article}
\usepackage{graphicx} % Required for inserting images
\usepackage[table,xcdraw,dvipsnames]{xcolor}
\usepackage{amsmath,amsfonts,amssymb,amsthm}
\usepackage{listings}
\usepackage{lipsum}
\usepackage{hyperref}
\usepackage{enumitem}

\usepackage{tikz}
\usepackage[siunitx, RPvoltages]{circuitikz}
\usetikzlibrary{3d}
\usepackage{comment}
\usepackage{caption,subcaption}
\usepackage{pgfplots}
\pgfplotsset{compat=newest} % or a newer version if available
\usepgfplotslibrary{groupplots}
\usetikzlibrary{pgfplots.groupplots}
\usetikzlibrary{shapes.geometric, arrows}
\tikzstyle{arrow} = [->,>=stealth,shorten >=2pt]
\tikzstyle{darrow}=[<->,shorten >=2pt,shorten <=2pt,>=stealth]
\usepackage{cancel}
\usepackage{bm}
\usepackage{fancyhdr}
\usepackage[utf8x]{inputenc}
\usepackage[T1]{fontenc}
\usepackage[margin=0.8in,top=1in,bottom=1in]{geometry}
%%%%%
\begin{filecontents*}{refs.bib}
@book{bornwolf,
  author    = {Born, M. and Wolf, E.},
  title     = {Principles of Optics},
  publisher = {Pergamon Press},
  edition   = {7},
  year      = {1999}
}
@book{hecht,
  author    = {Hecht, E.},
  title     = {Optics},
  publisher = {Addison-Wesley},
  edition   = {5},
  year      = {2016}
}
\end{filecontents*}
%
\newcommand{\institution}{University of Arizona}
\newcommand{\autor}{Nicolás Hernández Alegría}
\newcommand{\course}{OPTI 502 Optical Design and Instrumentation I}
\newcommand{\assignment}{Assignment 1}
%
\title{\textbf{\assignment}\\\course\\{\Large\institution}}
\author{\autor}
\date{\today}
%
\renewcommand{\sectionmark}[1]{\markright{#1}}
\fancypagestyle{mainstyle}{
    \fancyhf{} % Clear all header and footer fields
    \fancyfoot[C]{\thepage}
    \fancyhead[LE,RO]{\course} % Section name on odd pages
    \fancyhead[LO,RE]{\assignment}
    % Optional: Thin rules
    \renewcommand{\headrulewidth}{0pt} % Header rule
    \renewcommand{\footrulewidth}{0pt} % No footer rule
}
%
\begin{document}

\pagestyle{mainstyle}
\maketitle
%%
\section{Exercise 1}
The sign conventions studied in the class was the following:
\begin{enumerate}
  \item Right-Above directions are positive, in the $zy$ plane.
  \item Counter clock-wise angles are positive.
  \item Radius of curvature to its radius, in this direction, is positive.
\end{enumerate}
The scheme given is reproduced here for its analysis.
\begin{figure}[htbp]
  \centering
  \begin{circuitikz}
    \draw[arrow](0,0)--(0,4)node[midway,right]{$f$};
    \draw[arrow](0,4)--(2,4)node[midway,above]{$g$};
    \draw[arrow](2,4)--(2,3)node[midway,right]{$h$};
    \draw[arrow](0,0)--(1.5,0)node[midway,above]{$e$};
    \draw[arrow](1.5,0)--(1.5,1)node[midway,right]{$d$};
    \draw[arrow](3,1)--(1.5,1)node[midway,below]{$c$};
    \draw[arrow](3,1)--(3,.5)node[midway,right]{$b$};
    \draw[arrow](4.5,.5)--(3,.5)node[midway,below]{$a$};
    \draw[very thick](4.5,.5)--(2,3);
    \draw[arrow](2,4.5)--(4.5,4.5)node[midway,below]{$\bm{z}$};
    \draw[arrow](5,.5)--(5,3)node[midway,right]{$\bm{y}$};
    \draw[arrow,thick](4,.5)arc(180:125:.5)node[midway,left]{$\bm{u}$};
    \draw[dashed](2,4.5)--(2,4)(4.5,4.5)--(4.5,.5)(5,.5)--(4.5,.5)(5,3)--(2,3);
  \end{circuitikz}
  \caption{Original scheme.}
\end{figure}

\begin{enumerate}[itemsep=0pt,topsep=0pt,label=\alph*)]
  \item The tangent of the angle $\bm{u}$ is computed analyzing the directions
  of arrows $\bm{z}$ and $\bm{y}$. Both are positively defined, but the angle direction is clockwise, so that 
  $u<0$. Consequently, we plug a minus sign into the fraction $y/z$.
  \begin{align}
    \tan(\bm{-u})=-\tan\bm{u}=-\frac{\bm{y}}{\bm{z}}.
  \end{align}
  If we wish to develop a little the expression we can use the others rays to reexpress $\bm{z}$ and $\bm{y}$.
  On the one hand, we have
  \begin{align}
    \bm{z}+g=e-c-a\longrightarrow \bm{z}=e-c-a-g.
    \label{eq:zexercise1}
  \end{align}
  Then we do the same for $\bm{y}$:
  \begin{align*}
    f=(d+b)+\bm{y}-h\longrightarrow\bm{y}=f+h-b-d.
  \end{align*}
  Therefore,
  \begin{align}
    \tan\bm{u}=-\frac{\bm{y}}{\bm{z}}=-\frac{f+h-b-d}{e-c-a-g}.
  \end{align}
  \item We have already expressed the directed distance $z$ in equation \eqref{eq:zexercise1}, but we are going to 
  explain the derivation. To obtain $\bm{z}$, we calculate the total length of the diagram as $g+\bm{z}$. This result
  is equated to a positive oriented length that is obtained by summing the rays at the bottom: $e-c-a$.
  Once the equation is constructed, we can solve for $\bm{z}$ obtaining in that way the equation \eqref{eq:zexercise1}. 
\end{enumerate}

%%
\section{Exercise 2}
For the derivation of the law of reflection, we are going to use the following scheme illustrated in figure \ref{fig:exercise2}.
An incident ray with path length $L_1$ hits a plane mirror with an angle $\theta_1$. A reflection is obtained with an angle $\theta_2$
which propagates with optical length $L_2$.
\begin{figure}[htbp]
  \centering
  \begin{circuitikz}
    \fill[gray!20,draw=black](0,-2)rectangle(.25,2);
    \draw[arrow](-4,0)--(3,0)node[right]{$z$};
    \draw[arrow](0,-4)--(0,4)node[right]{$y$};
    \draw[arrow,very thick](235:3)--(0:0)node[midway,right]{$L_1$}--(135:4)node[midway,right]{$L_2$};
    \draw[arrow,thick](175:1)arc(175:135:1)node[midway,left]{$\theta_2$};
    \draw[arrow,thick](185:1)arc(185:235:1)node[midway,left]{$\theta_1$};
    \draw[dashed](135:4)--(.5,{4*sin(135)})node[right]{$\Delta_y-y$}(235:3)--(.5,{3*sin(235)})node[right]{$y$}
    (235:3)--({3*cos(235)},.1)node[above]{$z_1$}(135:4)--({4*cos(135)},-.1)node[below]{$z_2$};
    \draw(-4,1)node[]{$n$};
    \draw[darrow,thick](.5,{3*sin(235)})--(.5,{4*sin(135)})node[midway,right]{$\Delta_y$};
  \end{circuitikz}
  \caption{Scheme used for the derivation of the law of reflection.}
  \label{fig:exercise2}
\end{figure}

This figure has the right-hand convention so that the quantities listed have the following properties:
\begin{align*}
  \theta_1>0,\quad\theta_2<0,\quad z_1>0,\quad z_2<0,\quad y>0,\quad (\Delta_y-y)>0.
\end{align*}
The position coordinates were evaluated in terms of theirs direction. As both $z_1$ and $y_1$ intend to go to the positive 
quadrant, they are considered positive. The same idea was applied to the $L_2$ ray.

To begin with, the optical path length is the sum of the terms $nL_1$ and $nL_2$:
\begin{align*}
  \text{OPL}(y)=nL_1+nL_2=n\left[\sqrt{z_1^2+y^2}+\sqrt{(-z_2)^2+(\Delta_y-y)^2}\right]=n\left[\sqrt{z_1^2+y^2}+\sqrt{z_2^2+(\Delta_y-y)^2}\right].
\end{align*}
where $n$ is factored out as the ray remains in the same medium, and the squares correspond to the geometrical distance using
pytagoras theorem.

To apply the fermat principle, we set $d\text{OPL}/dy=0$:
\begin{align*}
  \frac{d\text{OPL}}{dy}=n\frac{d}{dy}\left[\sqrt{z_1^2+y^2}+\sqrt{z_2^2+(\Delta_y-y)^2}\right]=n\left[\frac{y}{\sqrt{z_1^2+y^2}}+\frac{\Delta_y-y}{\sqrt{z_2^2+(\Delta y-y)^2}}\right]=0
\end{align*}
The last result can be reduced by substituting back the definition of $L_1$ and $L_2$:
\begin{align*}
  \frac{d\text{OPL}}{dy}=\frac{y}{L_1}+\frac{\Delta_y-y}{L_2}=\sin\theta_1+\sin\theta_2=0
\end{align*}
Finally, using the odd property $f(x)=-f(x)$ of the sine function and the last result, we obtain the law of refraction:
\begin{align*}
  \sin\theta_1&=-\sin\theta_2\\
  \sin\theta_1&=\sin(-\theta_2)\bigr/\sin^{-1}(\cdot)\\
  \theta_1&=-\theta_2.
\end{align*}




%\nocite{*}
%\bibliographystyle{plain}   % or unsrt, alpha, apalike, etc.
%\bibliography{refs}

\end{document}
