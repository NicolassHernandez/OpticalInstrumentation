\documentclass[letterpaper,11pt,twoside]{article}
\usepackage{graphicx} % Required for inserting images
\usepackage[table,xcdraw,dvipsnames]{xcolor}
\usepackage{amsmath,amsfonts,amssymb,amsthm}
\usepackage{listings}
\usepackage{lipsum}
\usepackage{hyperref}
\usepackage{enumitem}
\usepackage{pdflscape}
\usepackage{rotating}
\usepackage{tikz}
\usepackage[siunitx, RPvoltages]{circuitikz}
\usetikzlibrary{3d}
\usepackage{comment}
\usepackage{caption,subcaption}
\usepackage{pgfplots}
\pgfplotsset{compat=newest} % or a newer version if available
\usepgfplotslibrary{groupplots}
\usetikzlibrary{pgfplots.groupplots}
\usetikzlibrary{shapes.geometric, arrows}
\tikzstyle{arrow} = [->,>=stealth,shorten >=2pt]
\tikzstyle{darrow}=[<->,shorten >=2pt,shorten <=2pt,>=stealth]
\usepackage{cancel}
\usepackage{bm}
\usepackage{fancyhdr}
\usepackage[utf8x]{inputenc}
\usepackage[T1]{fontenc}
\usepackage[margin=0.8in,top=1in,bottom=1in]{geometry}
%%%%%
\begin{filecontents*}{refs.bib}
@book{bornwolf,
  author    = {Born, M. and Wolf, E.},
  title     = {Principles of Optics},
  publisher = {Pergamon Press},
  edition   = {7},
  year      = {1999}
}
@book{hecht,
  author    = {Hecht, E.},
  title     = {Optics},
  publisher = {Addison-Wesley},
  edition   = {5},
  year      = {2016}
}
\end{filecontents*}
%
\newcommand{\institution}{University of Arizona}
\newcommand{\autor}{Nicolás Hernández Alegría}
\newcommand{\course}{OPTI 502 Optical Design and Instrumentation I}
\newcommand{\assignment}{Assignment 10}
%
\title{\textbf{\assignment}\\\course\\{\Large\institution}}
\author{\autor}
\date{\today}
%
\renewcommand{\sectionmark}[1]{\markright{#1}}
\fancypagestyle{mainstyle}{
    \fancyhf{} % Clear all header and footer fields
    \fancyfoot[C]{\thepage}
    \fancyhead[LE,RO]{\course} % Section name on odd pages
    \fancyhead[LO,RE]{\assignment}
    % Optional: Thin rules
    \renewcommand{\headrulewidth}{0pt} % Header rule
    \renewcommand{\footrulewidth}{0pt} % No footer rule
}
%
\begin{document}

\pagestyle{mainstyle}
\maketitle
%%
\section*{Exercise 1}
\begin{enumerate}[itemsep=0pt,topsep=0pt,label=\alph*)]
  \item The focal length of each element can be determined using the following formulas:
  \begin{align*}
    \phi_1=\frac{1}{f_1}=\frac{\nu_1}{\nu_1-\nu_2}\phi,\quad\text{and}\quad\phi_2=\frac{1}{f_2}=\frac{\nu_2}{\nu_2-\nu_1}\phi,
  \end{align*}
  where $\phi$ is the total power of the system. By evaluating each achromat, we find:
  \begin{align*}
    \text{Achromat 1}:\quad f_1=\frac{56.0-33.8}{56.0}100=39.643\;mm,\quad f_2=\frac{33.8}{33.8-56.0}100=-65.680\;mm.\\
    \text{Achromat 2}:\quad f_1=\frac{60.3-47.5}{60.3}100=21.227\;mm,\quad f_2=\frac{47.5}{47.5-60.3}100=-26.948\;mm.
  \end{align*}
  \item The elements have different power so that when combined the overall is the power desired. The elements also have equal and opposite longitudinal chromatic aberration and they are cancelled.
  \item The excess power is $\Delta\nu$ between the Abbe number of each element, where the larger $\Delta\nu$ the less excess power.
  For each achromat we have:
  \begin{align*}
    \text{Achromat 1}:\quad\Delta\nu_1=56.0-33.8=22.2\\
    \text{Achromat 2}:\quad\Delta\nu_2=60.3-47.5=12.8
  \end{align*} 
  Because $\Delta\nu_1<\Delta\nu_2$, the achromat 1 has less excess power.
  \item The ahcromat is composed of two thin lenses which are used to force the same focus for F and C light, but however, d light is focues at a different location. This effect is called secondary chromatic aberration. To correct it, we need a third lens.
  \item The secondary chromatic aberration is computing using 
  \begin{align*}
    \delta f_{Cd}=\frac{\Delta P}{\Delta\nu}f.
  \end{align*}
  Evaluating for each achromat, yields:
  \begin{align*}
    \text{Achromat 1}:\quad\delta f_{Cd}=\frac{0.303-0.292}{22.2}100=0.0495.\\
    \text{Achromat 2}:\quad\delta f_{Cd}=\frac{0.305-0.301}{12.8}100=0.0312.
  \end{align*}
  The achromat 2 has less secondary chromatic aberration.
\end{enumerate}

%%
\section*{Exercise 2}
This exercise can be performed purely by raytrace. We used an excel template for the raytrace where the elements were placed; the table is shown below.
\begin{figure}[h!]
  \centering
  \includegraphics[width=\columnwidth]{raytrace.png}
\end{figure}

\begin{enumerate}[itemsep=0pt,topsep=0pt,label=\alph*)]
  \item The EP and XP locations correspond to:
  \begin{align*}
    z_{EP}=-66.667\;mm,\quad\text{and}\quad z_{EP}=-150\;mm.
  \end{align*}
  Meaning that the entrance pupil is to the right of the first lens, while the exit pupil is to the left of the third lens.
  \item The focal length of the system and the back focal distance are:
  \begin{align*}
    f=\frac{1}{\phi}=131.579\;mm,\quad\text{and}\quad\text{BFD}=47.368\;mm.
  \end{align*}
  \item If the system stop is $20\;mm$, we need to scale the potential marginal ray from an infinity object by multiplying it by the following factor:
  \begin{align*}
    m_{\text{MR}}=\frac{10\;mm}{y_{\text{stop}}}=\frac{10\;mm}{0.6\;mm}=16.667\;mm.
  \end{align*} 
  By looking at the table, we have that 
  \begin{align*}
    D_{\text{EP}}=33.333\;mm,\quad\text{and}\quad D_{\text{XP}}=50\;mm.
  \end{align*}
  \item The unvignetted FOV given allow us to scale the potential chief ray by:
  \begin{align*}
    m_{CR}=\frac{\tan(12^\circ)}{\bar{y}_{\text{EP}}}=\frac{0.2126}{0.06}=3.543.
  \end{align*}
  From the table, we have that 
  \begin{align*}
    h=27.9711\;mm.
  \end{align*}
  \item The required lenses diameters to support this FOV are given by looking at the height of the chief and marginal ray at each lenses. Therefore, we have:
  \begin{align*}
    D_{L_1}\geq|y_{L_1}|+|\bar{y}_{L_1}|=61.678\;mm\\
    D_{L_2}\geq|y_{L_2}|+|\bar{y}_{L_2}|=20.000\;mm\\
    D_{L_3}\geq|y_{L_3}|+|\bar{y}_{L_3}|=54.516\;mm.
  \end{align*}
\end{enumerate}





%\nocite{*}
%\bibliographystyle{plain}   % or unsrt, alpha, apalike, etc.
%\bibliography{refs}

\end{document}