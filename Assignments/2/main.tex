\documentclass[letterpaper,11pt,twoside]{article}
\usepackage{graphicx} % Required for inserting images
\usepackage[table,xcdraw,dvipsnames]{xcolor}
\usepackage{amsmath,amsfonts,amssymb,amsthm}
\usepackage{listings}
\usepackage{lipsum}
\usepackage{hyperref}
\usepackage{enumitem}

\usepackage{tikz}
\usepackage[siunitx, RPvoltages]{circuitikz}
\usetikzlibrary{3d}
\usepackage{comment}
\usepackage{caption,subcaption}
\usepackage{pgfplots}
\pgfplotsset{compat=newest} % or a newer version if available
\usepgfplotslibrary{groupplots}
\usetikzlibrary{pgfplots.groupplots}
\usetikzlibrary{shapes.geometric, arrows}
\tikzstyle{arrow} = [->,>=stealth,shorten >=2pt]
\tikzstyle{darrow}=[<->,shorten >=2pt,shorten <=2pt,>=stealth]
\usepackage{cancel}
\usepackage{bm}
\usepackage{fancyhdr}
\usepackage[utf8x]{inputenc}
\usepackage[T1]{fontenc}
\usepackage[margin=0.8in,top=1in,bottom=1in]{geometry}
%%%%%
\begin{filecontents*}{refs.bib}
@book{bornwolf,
  author    = {Born, M. and Wolf, E.},
  title     = {Principles of Optics},
  publisher = {Pergamon Press},
  edition   = {7},
  year      = {1999}
}
@book{hecht,
  author    = {Hecht, E.},
  title     = {Optics},
  publisher = {Addison-Wesley},
  edition   = {5},
  year      = {2016}
}
\end{filecontents*}
%
\newcommand{\institution}{University of Arizona}
\newcommand{\autor}{Nicolás Hernández Alegría}
\newcommand{\course}{OPTI 502 Optical Design and Instrumentation I}
\newcommand{\assignment}{Assignment 2}
%
\title{\textbf{\assignment}\\\course\\{\Large\institution}}
\author{\autor}
\date{\today}
%
\renewcommand{\sectionmark}[1]{\markright{#1}}
\fancypagestyle{mainstyle}{
    \fancyhf{} % Clear all header and footer fields
    \fancyfoot[C]{\thepage}
    \fancyhead[LE,RO]{\course} % Section name on odd pages
    \fancyhead[LO,RE]{\assignment}
    % Optional: Thin rules
    \renewcommand{\headrulewidth}{0pt} % Header rule
    \renewcommand{\footrulewidth}{0pt} % No footer rule
}
%
\begin{document}

\pagestyle{mainstyle}
\maketitle
%%
\section{Exercise 1}
\begin{enumerate}[itemsep=0pt,topsep=0pt,label=\alph*)]
  \item \emph{Tunnel diagrams} are schemes that unfolds the optical path at each reflection, so that the total 
  propagation of the ray remains straight. For the prism below, at each reflecting surface it must be flipped to 
  achieve the above goal. The diagram then consists of a geometric figure composed of multiples rotation of the original prism.
  
  On the other hand, \emph{parity change} is the change in orientation of the object when looking backward through the $z$ optical
  axis to the object. Therefore, the parity change will be due to the $y$ axis (reversion) and $x$ axis (inversion). However,
  no parity change is due rotations with respect to the $z$ axis. In addition, only an odd number of reflection will change the parity.

  In the following three prisms, we will assume the ray propagates with total internal reflection or the surface is coated so that only
  reflection is studied. The last surface is not coated and the ray exists perpendicular to its normal.
  \begin{enumerate}[label=\roman*.]
    \item \textbf{Right-angle prism}
    The ray is reflected twice (figure \ref{fig:1a}), therefore there is no parity change.  
    \item \textbf{Dove prism}
    The ray is reflected three times (figure \ref{fig:1b}), so there will be a parity change. 
    \item \textbf{Pentaprism}
      The ray is reflected twice (figure \ref{fig:1c}) so that there will be no parity change.
  \end{enumerate}
    \begin{figure}[htbp]
      \centering
      \begin{subfigure}{.25\columnwidth}
        \centering
        \includegraphics[width=.9\columnwidth]{RightAnglePrism.png}
        \caption{Right-Angle prism.}
        \label{fig:1a}
      \end{subfigure}
      \hfill
      \begin{subfigure}{.3\columnwidth}
        \centering
        \includegraphics[width=\columnwidth]{DovePrism.png}
        \caption{Dove prism}
        \label{fig:1b}
      \end{subfigure}
      \hfill
      \begin{subfigure}{.35\columnwidth}
        \centering
        \includegraphics[width=\columnwidth]{PentaPrism.png}
        \caption{Penta prism}
        \label{fig:1c}
      \end{subfigure}
      \caption{Tunnel diagram for the given prisms.}
    \end{figure}
\item To obtain an equivalent effect with the Dove prism, we must see the propagation inside the geometry. In figure \ref{fig:1b} we have shown how can be disposed to achieve this, with dashed lines 
reinforcing this position as it can be seen as similar to a truncated right-angle prism.
\begin{figure}[htbp]
  \centering
  \includegraphics[width=.5\columnwidth]{1b.png}
  \caption{Equivalent orientation of the Dove prism to achieve an equivalent propagation as the Right-Angle prism.}
  \label{fig:1b}
\end{figure}
\end{enumerate}

%%
\section{Exercise 2}
The configuration is a lens of focal legnth $f=100\;mm$ with a pentaprism of $n=1.5$ just behind with some sides of length $a$.
The image plane is required to be at $f$ and outside the prism. If we denote $d_{in}$ as the distance traveled by the ray inside the 
prism and $d_{out}$ as the distance when the ray outs the prism, then the tocal length must be 
\begin{align*}
  d_{in}+d_{out}=f=100\;mm.
\end{align*}
\begin{enumerate}[itemsep=0pt,topsep=0pt,label=\alph*)]
  \item In this case, the maximum value of $a$ will imply that there is no space left for the air propagation and therefore $d_{out}=0$.
  If we create the tunnel diagram of the pentaprism (figure \ref{fig:2a}), we will see that the distance traveled by the ray is twice the length $a$ plus $\sqrt{2}a$ due to 
  the diagonal propagation along the geometry. Therefore, the reduced thickness is
  \begin{align*}
    d_{in}=\frac{(2+\sqrt{2})a}{n}.
  \end{align*}
  Thus, replacing $n=1/5$ and solving for $a$ yields
  \begin{align*}
    \frac{(2+\sqrt{2})a_{max}}{1.5}+0&=100\\
    a_{max}&=\frac{100(1.5)}{2+\sqrt{2}}=43.934\;mm.
  \end{align*}

  \item If the length is given to $a=40\;mm$ and the refractive index changed to $n=1.66$, then the image will be located $d_{out}$ away from the exit surface 
  of the pentaprism, that is,
  \begin{align*}
    \frac{(2+\sqrt{2})(40)}{1.66}+d_{out}&=100\\
    d_{out}&=100-\frac{(2+\sqrt{2})(40)}{1.66}=17.729\;mm.
  \end{align*}
  Figure \ref{fig:2b} illustrates this case.
\end{enumerate}
  \begin{figure}[h!]
    \centering
    \begin{subfigure}{.45\columnwidth}
      \centering
      \includegraphics[width=\columnwidth]{2a.png}
      \caption{Distance only atributed to $d_{in}$}
      \label{fig:2a}
    \end{subfigure}
    \hfill
    \begin{subfigure}{.45\columnwidth}
      \centering
    \includegraphics[width=\columnwidth]{2b.png}
    \caption{$d_{in}$ and $d_{out}$ contributes to tha total propagation}
    \label{fig:2b}
    \end{subfigure}
    \caption{Tunnel diagrams allows us to reduces the complexity of the optical configuration.}
  \end{figure}

%\nocite{*}
%\bibliographystyle{plain}   % or unsrt, alpha, apalike, etc.
%\bibliography{refs}

\end{document}
