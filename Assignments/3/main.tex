\documentclass[letterpaper,11pt,twoside]{article}
\usepackage{graphicx} % Required for inserting images
\usepackage[table,xcdraw,dvipsnames]{xcolor}
\usepackage{amsmath,amsfonts,amssymb,amsthm}
\usepackage{listings}
\usepackage{lipsum}
\usepackage{hyperref}
\usepackage{enumitem}

\usepackage{tikz}
\usepackage[siunitx, RPvoltages]{circuitikz}
\usetikzlibrary{3d}
\usepackage{comment}
\usepackage{caption,subcaption}
\usepackage{pgfplots}
\pgfplotsset{compat=newest} % or a newer version if available
\usepgfplotslibrary{groupplots}
\usetikzlibrary{pgfplots.groupplots}
\usetikzlibrary{shapes.geometric, arrows}
\tikzstyle{arrow} = [->,>=stealth,shorten >=2pt]
\tikzstyle{darrow}=[<->,shorten >=2pt,shorten <=2pt,>=stealth]
\usepackage{cancel}
\usepackage{bm}
\usepackage{fancyhdr}
\usepackage[utf8x]{inputenc}
\usepackage[T1]{fontenc}
\usepackage[margin=0.8in,top=1in,bottom=1in]{geometry}
%%%%%
\begin{filecontents*}{refs.bib}
@book{bornwolf,
  author    = {Born, M. and Wolf, E.},
  title     = {Principles of Optics},
  publisher = {Pergamon Press},
  edition   = {7},
  year      = {1999}
}
@book{hecht,
  author    = {Hecht, E.},
  title     = {Optics},
  publisher = {Addison-Wesley},
  edition   = {5},
  year      = {2016}
}
\end{filecontents*}
%
\newcommand{\institution}{University of Arizona}
\newcommand{\autor}{Nicolás Hernández Alegría}
\newcommand{\course}{OPTI 502 Optical Design and Instrumentation I}
\newcommand{\assignment}{Assignment 3}
%
\title{\textbf{\assignment}\\\course\\{\Large\institution}}
\author{\autor}
\date{\today}
%
\renewcommand{\sectionmark}[1]{\markright{#1}}
\fancypagestyle{mainstyle}{
    \fancyhf{} % Clear all header and footer fields
    \fancyfoot[C]{\thepage}
    \fancyhead[LE,RO]{\course} % Section name on odd pages
    \fancyhead[LO,RE]{\assignment}
    % Optional: Thin rules
    \renewcommand{\headrulewidth}{0pt} % Header rule
    \renewcommand{\footrulewidth}{0pt} % No footer rule
}
%
\begin{document}

\pagestyle{mainstyle}
\maketitle
%%
\section{Exercise 1}
\begin{enumerate}[itemsep=0pt,topsep=0pt,label=(\alph*)]
  \item
  \begin{enumerate}[itemsep=0pt,topsep=0pt,label=\roman*.]
    \item The ray diagram is illustrated in figure \ref{fig:1a}. We have traced three rays:
    \begin{itemize}[itemsep=0pt,topsep=0pt]
      \item Parallel to the optical axis from the object, then it is refracted with direction to $F'$.
      \item Direct to $F$: it is refracted so that it becomes parallel to the optical axis.
      \item The chief ray, which maintain its direction through its propagation.
    \end{itemize}
    The intersection of these three rays produces the image. We can see that the image is to the left of the lens, but to the right of the object.
    Therefore, it will be a virtual image and demagnified.
    \begin{figure}[h!]
      \centering
      \includegraphics[width=.8\columnwidth]{1a.pdf}
      \caption{Ray diagram of the problem, the position of the image is $z'$ and is located to the right of the object. Dashed lines correspond to 
      virtual rays.}
      \label{fig:1a}
    \end{figure}
    \item Using the thin lens equation, considering that $F'=-100\;mm$ and $z=-50\;mm$ provides
    \begin{align*}
      \frac{1}{z'}&=\frac{1}{F'}+\frac{1}{z}\\
      \frac{1}{z'}&=\frac{1}{-100}+\frac{1}{-50}\\
      z'&=\frac{(-100)(-50)}{-150}=-33.333\;mm.
    \end{align*}
    Because $z'<0$, the image is \textbf{virtual} and will be to the left of the lens. Its magnification is:
    \begin{align*}
      m=\frac{z'}{z}=\frac{-33.333}{-50}=0.667.
    \end{align*}
    The image is then erected ($\text{sgn}(m)=1$), and demagnified ($|m|<1$) making it smaller than the object.
  \end{enumerate}
  \item The Galilean telescope is composed of a positive lens and a negative lens to the right (figure \ref{fig:1b}). In order to make the system afocal, 
  is necessary that the concave lens be located so that $F_2$ is at the back focal plane of the convex lens, that is,
  $F_1'=F_2$.
  \begin{figure}[h!]
    \centering
    \includegraphics[width=.8\columnwidth]{1b.pdf}
    \caption{Galilean telescope illustration.}
    \label{fig:1b}
  \end{figure}
  \begin{enumerate}[itemsep=0pt,topsep=0pt,label=\roman*.]
    \item The distance between the two lenses $t$ is computed if we plot the focal length of each one as illustrated in the figure. 
    We see that $f_1$ is the total distance, which can be thought of as the sum of $f_2$ and $t$. Then, equating them and solving gives the distance $t$:
    \begin{align*}
      f_1&=t-f_2\\
      200&=t-(-50)\longrightarrow t=200-50=150\;mm.
    \end{align*} 
    Lets remember that an afocal system does not have focal points, that is, incoming rays parallel to the optical axis will produce an 
    image ray also parallel to the optical axis.

    On the other hand, the magnification is the ratio of the heights $h'/h$, or in this case $-f_2/f_1$:
    \begin{align*}
      m=\frac{-f_2}{f_1}=\frac{50}{200}=\frac{1}{4}.
    \end{align*}
    Therefore, the image will be demagnified but erected; there will be no parity change.
    \item Figure \ref{fig:1b} also shows the ray diagram that begins parallel to the optical axis. Then, it is refracted to be converged at point $F_1'$. However,
    before reaching the point the negative lens refracts it again so that it goes back to its initial direction, parallel to the optical axis. 
    This effect is because $F_2$ is located at $F_1'$ so that the second lens see an object converging to its focal length.
  \end{enumerate}
\end{enumerate}


\section{Exercise 2}
\begin{enumerate}[itemsep=0pt,topsep=0pt,label=(\alph*)]
  \item What we are asked to plot is the position $z'$ of the image when the position of the object is varied in the range $z\in[-300,300]\;mm$. To plot it as a function,
  we will use the rearranged thin lens equation 
  \begin{align}
    z'=\frac{zf}{z+f}.
    \label{eq:zprime}
  \end{align} 
  \begin{enumerate}[itemsep=0pt,topsep=0pt,label=\roman*.]
    \item For a positive lens $f=+100\;mm$, we have figure \ref{fig:2a1}. We can see that the curve has two asymptotes which corresponds to:
    \begin{itemize}[itemsep=0pt,topsep=0pt]
      \item\textbf{(1)} Object located at $z=-1$, generating a real image at $z'=\infty$.
      \item\textbf{(2)} Real image located at $z'=1$, which is due to an object at $z=-\infty$.
    \end{itemize} 
    \item For $f=-100\;mm$, we have figure \ref{fig:2a2}, where the same behavior is obtained as in the eprevious case. However,
    the asymptotes are displaced so that:
    \begin{itemize}[itemsep=0pt,topsep=0pt]
      \item\textbf{(1)} Object located at $z=1$, generating a real image at $z'=\infty$.
      \item\textbf{(2)} Virtual image located at $z'=-1$, which is due to an object at $z=-\infty$. 
    \end{itemize} 
  \end{enumerate}
  \begin{figure}[h!]
    \centering
    \begin{subfigure}{.45\columnwidth}
      \centering
      \begin{circuitikz}[scale=.7]
        \def\f{1}
        \draw[arrow](-4,0)--(4,0)node[right]{$z$};
        \draw[arrow](0,-4)--(0,4)node[above]{$z'$};
        \draw[very thick,NavyBlue,domain=-3:-1.35,samples=100] plot(\x,{ (\x*\f)/(\x+\f) });
        \draw[very thick,NavyBlue,domain=-.8:3,samples=100] plot(\x,{ (\x*\f)/(\x+\f) });
        \draw[dashed](-4,1)--(4,1)(-1,4)--(-1,-4)(-1,2)node[fill=white]{$(1)$}(-2,1)node[fill=white]{$(2)$};
        \foreach \t in{-3,...,3}{\draw(\t,.1)--(\t,-.1)node[below]{$\t$};}
        \foreach \y in{-3,-2,-1,1,2,3}{\draw(-.1,\y)--(.1,\y)node[right,fill=white]{$\y$};}
        %\draw[](-1/2,-1)node[draw,fill=red,circle,inner sep=1.5]{};
      \end{circuitikz}
      \caption{$f=+100\;mm$}
      \label{fig:2a1}
    \end{subfigure}
    \hfill
    \begin{subfigure}{.45\columnwidth}
      \centering
      \begin{circuitikz}[scale=.7]
        \def\f{-1}
        \draw[arrow](-4,0)--(4,0)node[right]{$z$};
        \draw[arrow](0,-4)--(0,4)node[above]{$z'$};
        \draw[very thick,NavyBlue,domain=-3:.8,samples=100] plot(\x,{ (\x*\f)/(\x+\f) });
        \draw[very thick,NavyBlue,domain=1.35:3,samples=100] plot(\x,{ (\x*\f)/(\x+\f) });
        \draw[dashed](-4,-1)--(4,-1)(1,4)--(1,-4)(1,-2)node[fill=white]{$(1)$}(2,-1)node[fill=white]{$(2)$};
        \foreach \t in{-3,...,3}{\draw(\t,-.1)--(\t,.1)node[above]{$\t$};}
        \foreach \y in{-3,-2,-1,1,2,3}{\draw(.1,\y)--(-.1,\y)node[left,fill=white]{$\y$};}
        %\draw[](1/2,1)node[draw,fill=red,circle,inner sep=1.5]{};
      \end{circuitikz}
      \caption{$f=-100\;mm$}
      \label{fig:2a2}
    \end{subfigure}
    \caption{Plot $z'-z$. Both axis were normalized by $100\;mm$ for clarity.}
  \end{figure}
  %%
  \item The magnification can be described as a function of the object distance $z$'. We use the magnification equation $m=z'/z$ and the equation \eqref{eq:zprime}:
  \begin{align}
    m=\frac{z'}{z}=\frac{f}{z+f}.
    \label{eq:mag}
  \end{align}
  \begin{enumerate}[itemsep=0pt,topsep=0pt,label=\roman*.]
    \item For a positive lens $f=+100\;mm$, we have figure \ref{fig:2b1}. We can see that the curve has two asymptotas which corresponds to:
    \begin{itemize}[itemsep=0pt,topsep=0pt]
      \item\textbf{(1)} Object located at $z=-1$, generating a real image with $m=\infty$.
      \item\textbf{(2)} Real image with $m=0$, which is due to an object at $z=-\infty$.
    \end{itemize} 
    \item For $f=-100\;mm$, we have the figure \ref{fig:2b2}, where the same behavior is obtained as in th eprevious case. In this case,
    the asymptotes are displaced so that:
    \begin{itemize}[itemsep=0pt,topsep=0pt]
      \item\textbf{(1)} Object located at $z=1$, generating a real image with $m=\infty$.
      \item\textbf{(2)} Virtual image with $m=0$, which is due to a object at $z=-\infty$. 
    \end{itemize} 
  \end{enumerate}
  \begin{figure}[h!]
    \centering
    \begin{subfigure}{.45\columnwidth}
      \centering
      \begin{circuitikz}[scale=.7]
        \def\f{1}
        \draw[arrow](-4,0)--(4,0)node[right]{$z$};
        \draw[arrow](0,-4)--(0,4)node[above]{$m$};
        \draw[very thick,NavyBlue,domain=-3:-1.25,samples=100] plot(\x,{ (\f)/(\x+\f) });
        \draw[very thick,NavyBlue,domain=-.75:3,samples=100] plot(\x,{ (\f)/(\x+\f) });
        \draw[dashed](-4,0)--(4,0)(-1,4)--(-1,-4)(-1,1)node[fill=white]{$(1)$}(-3.5,0)node[fill=white]{$(2)$};
        \foreach \t in{-3,...,3}{\draw(\t,.1)--(\t,-.1)node[below]{$\t$};}
        \foreach \y in{-3,-2,-1,1,2,3}{\draw(-.1,\y)--(.1,\y)node[right,fill=white]{$\y$};}
        \draw[](-1/2,2)node[draw,fill=red,circle,inner sep=1.5]{};
      \end{circuitikz}
      \caption{$f=+100\;mm$}
      \label{fig:2b1}
    \end{subfigure}
    \hfill
    \begin{subfigure}{.45\columnwidth}
      \centering
      \begin{circuitikz}[scale=.7]
        \def\f{-1}
        \draw[arrow](-4,0)--(4,0)node[right]{$z$};
        \draw[arrow](0,-4)--(0,4)node[above]{$m$};
        \draw[very thick,NavyBlue,domain=-3:.75,samples=100] plot(\x,{ (\f)/(\x+\f) });
        \draw[very thick,NavyBlue,domain=1.25:3,samples=100] plot(\x,{ (\f)/(\x+\f) });
        \draw[dashed](-4,0)--(4,0)(1,4)--(1,-4)(1,1)node[fill=white]{$(1)$}(-3.5,0)node[fill=white]{$(2)$};
        \foreach \t in{-3,...,3}{\draw(\t,.1)--(\t,-.1)node[below]{$\t$};}
        \foreach \y in{-3,-2,-1,1,2,3}{\draw(.1,\y)--(-.1,\y)node[left,fill=white]{$\y$};}
        \draw[](1/2,2)node[draw,fill=red,circle,inner sep=1.5]{};
      \end{circuitikz}
      \caption{$f=-100\;mm$}
      \label{fig:2b2}
    \end{subfigure}
    \caption{Plot $m-z$. Horizontal axis was normalized by $100\;mm$ for clarity.}
  \end{figure}
  \item For a magnification $m+=2$, the distances of the object and image can be obtained by looking the figures \ref{fig:2b1} and \ref{fig:2b2}, respectively.
  \begin{enumerate}[itemsep=0pt,topsep=0pt,label=\roman*.]
    \item For a positive lens $f=+100\;mm$, this magnification is obtained by looking at figure \ref{fig:2b1} (red circle), indicating that $z<0$.
    Then, we solve the magnification equation \eqref{eq:mag}
    \begin{align*}
      m=\frac{f}{z+f}=2\longrightarrow f&=2z+2f\\ 
      -f&=2z\\
      z&=-f/2.
    \end{align*}
    Substituting the image position in $m=z'/z$ give the object position:
    \begin{align*}
      m=\frac{z'}{z}=\frac{z'}{-f/2}=2\longrightarrow z'=-f.
    \end{align*}
    Thus, both the object and image are located to the left of the lens, the object is real while the image is virtual and to the left of the object.
    \item In the case of a negative lens $f=-100\;mm$, by looking figure \ref{fig:2b2} we know that the object must be to the right of the negative lens, as $z'>0$.
    By doing an analogus procedure as before, we found that $z=f/2$ and $z'=f$. However, after verifying with a ray diagram we concluded that is \textbf{not possible} 
    to obtain a magnified image with a real object. It \textbf{can only happen} when the object is virtual, that is, the rays correspond to the image of a previous optical system, which is the object
    for the lens we are studying. The image will be then real.
  \end{enumerate}
\end{enumerate}


%\nocite{*}
%\bibliographystyle{plain}   % or unsrt, alpha, apalike, etc.
%\bibliography{refs}

\end{document}
