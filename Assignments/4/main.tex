\documentclass[letterpaper,11pt,twoside]{article}
\usepackage{graphicx} % Required for inserting images
\usepackage[table,xcdraw,dvipsnames]{xcolor}
\usepackage{amsmath,amsfonts,amssymb,amsthm}
\usepackage{listings}
\usepackage{lipsum}
\usepackage{hyperref}
\usepackage{enumitem}

\usepackage{tikz}
\usepackage[siunitx, RPvoltages]{circuitikz}
\usetikzlibrary{3d}
\usepackage{comment}
\usepackage{caption,subcaption}
\usepackage{pgfplots}
\pgfplotsset{compat=newest} % or a newer version if available
\usepgfplotslibrary{groupplots}
\usetikzlibrary{pgfplots.groupplots}
\usetikzlibrary{shapes.geometric, arrows}
\tikzstyle{arrow} = [->,>=stealth,shorten >=2pt]
\tikzstyle{darrow}=[<->,shorten >=2pt,shorten <=2pt,>=stealth]
\usepackage{cancel}
\usepackage{bm}
\usepackage{fancyhdr}
\usepackage[utf8x]{inputenc}
\usepackage[T1]{fontenc}
\usepackage[margin=0.8in,top=1in,bottom=1in]{geometry}
%%%%%
\begin{filecontents*}{refs.bib}
@book{bornwolf,
  author    = {Born, M. and Wolf, E.},
  title     = {Principles of Optics},
  publisher = {Pergamon Press},
  edition   = {7},
  year      = {1999}
}
@book{hecht,
  author    = {Hecht, E.},
  title     = {Optics},
  publisher = {Addison-Wesley},
  edition   = {5},
  year      = {2016}
}
\end{filecontents*}
%
\newcommand{\institution}{University of Arizona}
\newcommand{\autor}{Nicolás Hernández Alegría}
\newcommand{\course}{OPTI 502 Optical Design and Instrumentation I}
\newcommand{\assignment}{Assignment 4}
%
\title{\textbf{\assignment}\\\course\\{\Large\institution}}
\author{\autor}
\date{\today}
%
\renewcommand{\sectionmark}[1]{\markright{#1}}
\fancypagestyle{mainstyle}{
    \fancyhf{} % Clear all header and footer fields
    \fancyfoot[C]{\thepage}
    \fancyhead[LE,RO]{\course} % Section name on odd pages
    \fancyhead[LO,RE]{\assignment}
    % Optional: Thin rules
    \renewcommand{\headrulewidth}{0pt} % Header rule
    \renewcommand{\footrulewidth}{0pt} % No footer rule
}
%
\begin{document}

\pagestyle{mainstyle}
\maketitle
%%
\section*{Exercise 1}
The reduced thickness in this case, help us to obtain the equivalent air space of the medium of refractive index $n$. 
The greater $n$ the shorter the equivalent distance, that because the wave propagates slower.
\begin{enumerate}[itemsep=0pt,topsep=0pt,label=\alph*)]
  \item In this case, we have
  \begin{align*}
    \tau=\frac{500\;mm}{1.33}=375.94\;mm.
  \end{align*}
  \item The total distance is the sum of the air thickness and the reduced thickness of the water:
  \begin{align*}
    \tau_{\text{total}}=500\;mm+375.94\;mm=875.94\;mm.
  \end{align*}
  \item In this case, we assume that the thick layer of ice has \textbf{replaced} $100\;m$ of the air distance while the distance from 
  of the water remains the same.
  \begin{itemize}[itemsep=0pt,topsep=0pt]
    \item For the part a), the distance would be:
    \begin{align*}
      \tau_{\text{total}}=\frac{100\;mm}{1.31}+\frac{500\;mm}{1.33}=452.27\;mm.
    \end{align*}
    \item For part b), the total equivalent distance is 
    \begin{align*}
      \tau_{\text{total}}=400\;mm+\frac{100\;mm}{1.31}+\frac{500\;mm}{1.33}=852.27\;mm.
    \end{align*}
    Because $100\;mm$ of aire has now been replaced by the ice.
  \end{itemize}
\end{enumerate}

\section*{Exercise 2}
I will assume that the glass rod is intended to perform the action of the two lenses in each case so that depending on the raddi of curvatures of each surface,
an incoming collimated ray will outputs collimated but scaled. Therefore, the problem reduces to find $R_1$ and $R_2$ of the rod to accomplish both 
conditions: remains collimated (zero total optical power), and perform magnification.
\begin{enumerate}[itemsep=0pt,topsep=0pt,label=\alph*)]
  \item For a magnification of $-0.5$, we have 
  \begin{align*}
    \Phi_{\text{total}}=\Phi_{1}+\Phi_{2}-\frac{t}{n}\Phi_1\Phi_2=0.
  \end{align*}
  Using the magnification formula for afocal systems $m=-f_2/f_1$ and the definition of the optical power $\Phi_i=(n'-n)/R_i=1/f_i$  allow us to related 
  the optical power of each surface as $m=-\Phi_1/\Phi_2$. Equating the magnification if the previous formula
  \begin{align*}
    m=-\Phi_1/\Phi_2=-0.5\longrightarrow\Phi_1=0.5\Phi_2.
  \end{align*}
  If we replace this rsult in the total power and solve for $\Phi_2$ yields
  \begin{align*}
    (0.5\Phi_2)+\Phi_2-\frac{t}{n}(0.5\Phi_2)\Phi_2=0\longrightarrow\Phi_2=\frac{3n}{t}=-\frac{(n-1)}{R_2}.
  \end{align*}
  We can now substitute the refactive index and the length for then solve for $R_2$:
  \begin{align*}
    \Phi_2=30\;mm^{-1}\longrightarrow R_2=-0.016\;mm.
  \end{align*}
  Then, $R_1$ is 
  \begin{align*}
    \Phi_1=0.5(30)=15\;mm^{-1}\longrightarrow R_1=\frac{n-1}{\Phi_1}=0.033\;mm.
  \end{align*}
  \item To achieve a magnification of $+0.5$ we do the same procedure.
  First, the magnification tells us 
  \begin{align*}
    m=-\frac{\Phi_1}{\Phi_2}=0.5\longrightarrow\Phi_2=-2\Phi_1.
  \end{align*}
  Then, substituting in the total power and solving for $\Phi_1$ yields
  \begin{align*}
    \Phi_1+(-2\Phi_1)-\frac{t}{n}\Phi_1(-2\Phi_1)=0\longrightarrow\Phi_1=\frac{n}{2t}=\frac{(n-1)}{R_1}.
  \end{align*}
  The value of the raddii of curvature 1 is 
  \begin{align*}
    \Phi_1=5\;mm^{-1}\longrightarrow R_1=0.10\;mm.
  \end{align*}
  The same for $R_2$ is 
  \begin{align*}
    \Phi_2=-2(5)=-10\;mm^{-1}\longrightarrow R_2=0.050\;mm.
  \end{align*}
\end{enumerate}


%\nocite{*}
%\bibliographystyle{plain}   % or unsrt, alpha, apalike, etc.
%\bibliography{refs}

\end{document}
