\documentclass[letterpaper,11pt,twoside]{article}
\usepackage{graphicx} % Required for inserting images
\usepackage[table,xcdraw,dvipsnames]{xcolor}
\usepackage{amsmath,amsfonts,amssymb,amsthm}
\usepackage{listings}
\usepackage{lipsum}
\usepackage{hyperref}
\usepackage{enumitem}

\usepackage{tikz}
\usepackage[siunitx, RPvoltages]{circuitikz}
\usetikzlibrary{3d}
\usepackage{comment}
\usepackage{caption,subcaption}
\usepackage{pgfplots}
\pgfplotsset{compat=newest} % or a newer version if available
\usepgfplotslibrary{groupplots}
\usetikzlibrary{pgfplots.groupplots}
\usetikzlibrary{shapes.geometric, arrows}
\tikzstyle{arrow} = [->,>=stealth,shorten >=2pt]
\tikzstyle{darrow}=[<->,shorten >=2pt,shorten <=2pt,>=stealth]
\usepackage{cancel}
\usepackage{bm}
\usepackage{fancyhdr}
\usepackage[utf8x]{inputenc}
\usepackage[T1]{fontenc}
\usepackage[margin=0.8in,top=1in,bottom=1in]{geometry}
%%%%%
\begin{filecontents*}{refs.bib}
@book{bornwolf,
  author    = {Born, M. and Wolf, E.},
  title     = {Principles of Optics},
  publisher = {Pergamon Press},
  edition   = {7},
  year      = {1999}
}
@book{hecht,
  author    = {Hecht, E.},
  title     = {Optics},
  publisher = {Addison-Wesley},
  edition   = {5},
  year      = {2016}
}
\end{filecontents*}
%
\newcommand{\institution}{University of Arizona}
\newcommand{\autor}{Nicolás Hernández Alegría}
\newcommand{\course}{OPTI 502 Optical Design and Instrumentation I}
\newcommand{\assignment}{Assignment 4}
%
\title{\textbf{\assignment}\\\course\\{\Large\institution}}
\author{\autor}
\date{\today}
%
\renewcommand{\sectionmark}[1]{\markright{#1}}
\fancypagestyle{mainstyle}{
    \fancyhf{} % Clear all header and footer fields
    \fancyfoot[C]{\thepage}
    \fancyhead[LE,RO]{\course} % Section name on odd pages
    \fancyhead[LO,RE]{\assignment}
    % Optional: Thin rules
    \renewcommand{\headrulewidth}{0pt} % Header rule
    \renewcommand{\footrulewidth}{0pt} % No footer rule
}
%
\begin{document}

\pagestyle{mainstyle}
\maketitle
%%
\section*{Exercise 1}
The reduced thickness in this case, help us to obtain the equivalent air space of the medium of refractive index $n$. 
The greater $n$ the shorter the equivalent distance, that because the wave propagates slower.
\begin{enumerate}[itemsep=0pt,topsep=0pt,label=\alph*)]
  \item In this case, we have
  \begin{align*}
    d_{\text{total}}=\frac{500\;mm}{1.33}=375.94\;mm.
  \end{align*}
  The fish appears to be $377\;mm$ below the surface of the water.
  \item The total distance is the sum of the air thickness in terms of the water and the thickness of the water:
  \begin{align*}
    d_{\text{total}}=1.33\cdot500\;mm+500\;mm=665\;mm+500\;mm=1165\;mm.
  \end{align*}
  The cat appears to be $665\;mm$ above the surface of the water.
  \item In this case, we assume that the thick layer of ice has \textbf{replaced} $100\;m$ of the water while the distance of air remains the same.
  \begin{itemize}[itemsep=0pt,topsep=0pt]
    \item For the part a), the distance would be:
    \begin{align*}
      d_{\text{total}}=(\frac{100\;mm}{1.31}+\frac{400\;mm}{1.33})+500\;mm=377\;mm+500\;mm=877\;mm.
    \end{align*}
    The fish appears to be $377\;mm$ below the surface of the ice.
    \item For part b), the total equivalent distance is the distance of the water, plus the equivalent distance in water of the ice and air: 
    \begin{align*}
      d_{\text{total}}=1.33\cdot(\frac{100\;mm}{1.31}+500\;mm)+400\;mm=767\;mm+400\;mm=1166.53\;mm.
    \end{align*}
    The cat appears to be $767\;mm$ above the water, that is, below the air and the ice.
    We computed first the reduced thickness of ice in order to then convert it to the equivalent in water.
  \end{itemize}
\end{enumerate}

\section*{Exercise 2}
The afocal we have seen in classes are Galilean (positive magnification) and Keplerian (negative magnification).
\begin{enumerate}[itemsep=0pt,topsep=0pt,label=\alph*)]
  \item For a magnification of $-0.5$, we have a Keplerian telescope, composed of two positive lenses.
    The reduced thickness of the glass rodd is then:
    \begin{align*}
      \tau_{\text{air}}=\frac{150\;mm}{1.5}=100\;mm=f_1+f_2.
    \end{align*}
    The magnification is:
    \begin{align*}
      m=-\frac{f_2}{f_1}=-0.5\longrightarrow f_1=2f_2.
    \end{align*}
    We replace $f_1$ in the reduced thickness formula 
    \begin{align*}
      (2f_2)+f_2=f_2=\frac{100\;mm}{3}\longrightarrow f_2=33.3\;mm.
    \end{align*}
    Then,
    \begin{align*}
      f_1=2f_2=66.6\;mm.
    \end{align*}
    The focal length can be related to the optical power through the formula $\Phi_i=1/f_i=(n'-n)/R_i$. We apply it for each lens:
    \begin{align*}
      \Phi_1=\frac{1.5-1}{R_1}=\frac{1}{f_1}=\frac{1}{66.6\;mm}\longrightarrow R_1=+33.3\;mm.
    \end{align*}
    For the second lens, we have:
    \begin{align*}
      \Phi_2=\frac{1-1.5}{R_2}=\frac{1}{f_2}=\frac{1}{33.3\;mm}\longrightarrow R_2=-16.6\;mm.
    \end{align*}
  \item To achieve a magnification of $+0.5$ we use the Galilean telescope, composed of a positive lens followed by a negative one.
  The procedure is similar to above.
  The reduced thickness of the glass rodd is then:
  \begin{align*}
    \tau_{\text{air}}=\frac{150\;mm}{1.5}=100\;mm=f_1+f_2.
  \end{align*}
  The magnification is:
  \begin{align*}
    m=-\frac{f_2}{f_1}=0.5\longrightarrow f_1=-2f_2.
  \end{align*}
  We replace $f_1$ in the reduced thickness formula 
  \begin{align*}
    (-2f_2)+f_2=-f_2=100\;mm\longrightarrow f_2=-100\;mm.
  \end{align*}
  Then,
  \begin{align*}
    f_1=-2f_2=200\;mm.
  \end{align*}
  The focal length can be related to the optical power through the formula $\Phi_i=1/f_i=(n'-n)/R_i$. We apply it for each lens:
  \begin{align*}
    \Phi_1=\frac{1.5-1}{R_1}=\frac{1}{f_1}=\frac{1}{200\;mm}\longrightarrow R_1=+100\;mm.
  \end{align*}
  For the second lens, we have:
  \begin{align*}
    \Phi_2=\frac{1-1.5}{R_2}=\frac{1}{f_2}=\frac{1}{-100\;mm}\longrightarrow R_2=+50\;mm.
  \end{align*}

\end{enumerate}


%\nocite{*}
%\bibliographystyle{plain}   % or unsrt, alpha, apalike, etc.
%\bibliography{refs}

\end{document}
