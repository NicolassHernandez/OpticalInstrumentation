\documentclass[letterpaper,11pt,twoside]{article}
\usepackage{graphicx} % Required for inserting images
\usepackage[table,xcdraw,dvipsnames]{xcolor}
\usepackage{amsmath,amsfonts,amssymb,amsthm}
\usepackage{listings}
\usepackage{lipsum}
\usepackage{hyperref}
\usepackage{enumitem}

\usepackage{tikz}
\usepackage[siunitx, RPvoltages]{circuitikz}
\usetikzlibrary{3d}
\usepackage{comment}
\usepackage{caption,subcaption}
\usepackage{pgfplots}
\pgfplotsset{compat=newest} % or a newer version if available
\usepgfplotslibrary{groupplots}
\usetikzlibrary{pgfplots.groupplots}
\usetikzlibrary{shapes.geometric, arrows}
\tikzstyle{arrow} = [->,>=stealth,shorten >=2pt]
\tikzstyle{darrow}=[<->,shorten >=2pt,shorten <=2pt,>=stealth]
\usepackage{cancel}
\usepackage{bm}
\usepackage{fancyhdr}
\usepackage[utf8x]{inputenc}
\usepackage[T1]{fontenc}
\usepackage[margin=0.8in,top=1in,bottom=1in]{geometry}
%%%%%
\begin{filecontents*}{refs.bib}
@book{bornwolf,
  author    = {Born, M. and Wolf, E.},
  title     = {Principles of Optics},
  publisher = {Pergamon Press},
  edition   = {7},
  year      = {1999}
}
@book{hecht,
  author    = {Hecht, E.},
  title     = {Optics},
  publisher = {Addison-Wesley},
  edition   = {5},
  year      = {2016}
}
\end{filecontents*}
%
\newcommand{\institution}{University of Arizona}
\newcommand{\autor}{Nicolás Hernández Alegría}
\newcommand{\course}{OPTI 502 Optical Design and Instrumentation I}
\newcommand{\assignment}{Assignment 5}
%
\title{\textbf{\assignment}\\\course\\{\Large\institution}}
\author{\autor}
\date{\today}
%
\renewcommand{\sectionmark}[1]{\markright{#1}}
\fancypagestyle{mainstyle}{
    \fancyhf{} % Clear all header and footer fields
    \fancyfoot[C]{\thepage}
    \fancyhead[LE,RO]{\course} % Section name on odd pages
    \fancyhead[LO,RE]{\assignment}
    % Optional: Thin rules
    \renewcommand{\headrulewidth}{0pt} % Header rule
    \renewcommand{\footrulewidth}{0pt} % No footer rule
}
%
\begin{document}

\pagestyle{mainstyle}
\maketitle
%%
\section*{Exercise 1}
\begin{enumerate}[itemsep=0pt,topsep=0pt,label=\alph*)]
  \item In this case, the object is virtual and the image will be real and erected.
  \begin{figure}[h!]
    \centering
    \includegraphics[width=.6\columnwidth]{problem1_a.png}
    \caption{}
  \end{figure}
  \item The object is virtual, and therefore the image will be real and erected.
  \begin{figure}[h!]
    \centering
    \includegraphics[width=.6\columnwidth]{problem1_b.png}
    \caption{}
  \end{figure}
  \item The object is virtual, and the image will be virtual and inverted. 
  \begin{figure}[h!]
    \centering
    \includegraphics[width=.6\columnwidth]{problem1_c.png}
    \caption{}
  \end{figure}
\end{enumerate}
%%
\section*{Exercise 2}
\begin{enumerate}[itemsep=0pt,topsep=0pt,label=\alph*)]
  \item For a single refracting surface, we have that:
  \begin{itemize}
    \item Both nodal points are located at the center of curvature CC.
    \item Front and real principal planes are located at the vortex.
    \item The reduced thickness of the surface is the focal length of its thin lens representation.
  \end{itemize}
  We illustrate these quantities along with the vertex and the focal lengths in the following figure.
  \begin{figure}[h!]
    \centering
    \includegraphics[width=.6\columnwidth]{problem2_a.png}
    \caption{Illustration of cardinal point for a single refractive surface.}
  \end{figure}
  We illustrate also some quantities of this surface:
  \begin{align*}
    &C=\frac{1}{R}=100\;m^{-1},\quad \phi=(n'-n)C=33.3\;m^{-1},\quad f_E=\frac{1}{\phi}=30\;mm,\\
    &f_F=-nf_E=-30\;mm,\quad f'_R=n'f_E=40\;mm.
  \end{align*}
  \item We use the following equation:
  \begin{align*}
    \frac{n'}{z'}=\frac{n}{z}+\frac{1}{f_E}\longrightarrow z'=\frac{n'zf_E}{nf_E+z}.
  \end{align*}
  Replacing the physical values and the EFL:
  \begin{align*}
    z'=\frac{(1.333)(30)(100)}{(1)(30)+100}=+30.7615\;mm.
  \end{align*}
  Its height is determined by the magnification:
  \begin{align*}
    m=\frac{h'}{h}=\frac{z'/n'}{z/n}=\frac{30.7615/1.333}{100/1}=0.231\longrightarrow h'=mh=(0.231)(10\;mm)=2.31\;mm.
  \end{align*}
  \item The cube is divided in equal part by the optical axis, yielding a height of $h=5\;mm$. Its last side is located $100\;mm$ from the principal planes, as all their 
  sides have the same sizes. Its first face is located $110\;mm$ from the principal plane. Now, the area of one side is $100\;mm^2$.
  We want to find the equivalent (area of volume) of its image.
  We can address this problem by considering two lines at each side of the cube as two independent objects with $(z_1,h_1)$ and $(z_2,h_2)$.
  Then, we do imaging of both to get $(z_1',h_1')$ and $(z_2',h_2')$. The difference between positions and heights allow us to construct the image dimension.
  
  \begin{align*}
    z_1:&\quad z_1'=\frac{(1.333)(30)(110)}{(1)(30)+110}=+31.42\;mm\\
    z_2:&\quad z_2'=\frac{(1.333)(30)(100)}{(1)(30)+100}=+30.76\;mm
  \end{align*}
  We do the same for the magnification to compute the corresponding heights:
  \begin{align*}
    m_1&=\frac{31.42/1.333}{110/1}=0.214\longrightarrow h_1'=m_1h_1=(0.214)(5)=1.07\;mm\\
    m_2&=\frac{30.76/1.333}{100/1}=0.231\longrightarrow h_2'=m_2h_2=(0.231)(5)=1.16\;mm.
  \end{align*}
  The other dimension remains the same $10\;mm$ as it is seen at the same distance from the refractive surface. The area, would be 
  the integration from $z_1'$ to $z_2'$ with the respective height which can be used to create a linear function (interpolation). However, we will 
  assum the mean value between them to consider it as a constant value. Then, the are is:
  \begin{align*}
    A=\Delta z'\Delta h'=2(z_2'-z_1')(\frac{h_1'+h_2'}{2})=1.472\;mm^2.
  \end{align*} 
  And the volume is this area multiplied by the remaining dimension:
  \begin{align*}
    \Delta x \cdot A=(10)(1.472)=14.72\;mm^3.
  \end{align*}
\end{enumerate}

%%
\section*{Exercise 3}
For a two positive lens system, we use Gaussian reduction to reduce the effect to a single thin lens.
We first compute the overall optical power with the power of individual lenses:
\begin{align*}
  \phi=\phi_1+\phi_2-\phi_1\phi_2t=\frac{1}{40}+\frac{1}{40}-\frac{1}{40}\frac{1}{40}\cdot20=0.038mm^{-1}\longrightarrow f_E=\frac{1}{\phi}=26.67\;mm.
\end{align*}
  \begin{figure}[h!]
    \centering
    \includegraphics[width=.8\columnwidth]{problem3.png}
    \caption{Gaussian reduction for two positive lenses.}
  \end{figure}
The distances $d$ and $d'$ are the shift in the principal plane of the respective individual elelemnt, to the ones of the total system. In this case where $n=1$,
they are given by
\begin{align*}
  d=\frac{\phi_2}{\phi}t=\frac{0.025}{0.038}20=13.158\;mm,\quad\text{and}\quad d'=-\frac{\phi_1}{\phi}t=-\frac{0.025}{0.038}20=-13.158\;mm.
\end{align*}

I think the distances are just relative, no absolute value can be given, what would be the reference then?


%%
\section*{Exercise 4}

%%
\section*{Exercise 5}


%\nocite{*}
%\bibliographystyle{plain}   % or unsrt, alpha, apalike, etc.
%\bibliography{refs}

\end{document}
