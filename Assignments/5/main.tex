\documentclass[letterpaper,11pt,twoside]{article}
\usepackage{graphicx} % Required for inserting images
\usepackage[table,xcdraw,dvipsnames]{xcolor}
\usepackage{amsmath,amsfonts,amssymb,amsthm}
\usepackage{listings}
\usepackage{lipsum}
\usepackage{hyperref}
\usepackage{enumitem}

\usepackage{tikz}
\usepackage[siunitx, RPvoltages]{circuitikz}
\usetikzlibrary{3d}
\usepackage{comment}
\usepackage{caption,subcaption}
\usepackage{pgfplots}
\pgfplotsset{compat=newest} % or a newer version if available
\usepgfplotslibrary{groupplots}
\usetikzlibrary{pgfplots.groupplots}
\usetikzlibrary{shapes.geometric, arrows}
\tikzstyle{arrow} = [->,>=stealth,shorten >=2pt]
\tikzstyle{darrow}=[<->,shorten >=2pt,shorten <=2pt,>=stealth]
\usepackage{cancel}
\usepackage{bm}
\usepackage{fancyhdr}
\usepackage[utf8x]{inputenc}
\usepackage[T1]{fontenc}
\usepackage[margin=0.8in,top=1in,bottom=1in]{geometry}
%%%%%
\begin{filecontents*}{refs.bib}
@book{bornwolf,
  author    = {Born, M. and Wolf, E.},
  title     = {Principles of Optics},
  publisher = {Pergamon Press},
  edition   = {7},
  year      = {1999}
}
@book{hecht,
  author    = {Hecht, E.},
  title     = {Optics},
  publisher = {Addison-Wesley},
  edition   = {5},
  year      = {2016}
}
\end{filecontents*}
%
\newcommand{\institution}{University of Arizona}
\newcommand{\autor}{Nicolás Hernández Alegría}
\newcommand{\course}{OPTI 502 Optical Design and Instrumentation I}
\newcommand{\assignment}{Assignment 5}
%
\title{\textbf{\assignment}\\\course\\{\Large\institution}}
\author{\autor}
\date{\today}
%
\renewcommand{\sectionmark}[1]{\markright{#1}}
\fancypagestyle{mainstyle}{
    \fancyhf{} % Clear all header and footer fields
    \fancyfoot[C]{\thepage}
    \fancyhead[LE,RO]{\course} % Section name on odd pages
    \fancyhead[LO,RE]{\assignment}
    % Optional: Thin rules
    \renewcommand{\headrulewidth}{0pt} % Header rule
    \renewcommand{\footrulewidth}{0pt} % No footer rule
}
%
\begin{document}

\pagestyle{mainstyle}
\maketitle
%%
\section*{Exercise 1}
\begin{enumerate}[itemsep=0pt,topsep=0pt,label=\alph*)]
  \item In this case, the object is virtual and the image will be real and erected.
  \begin{figure}[h!]
    \centering
    \includegraphics[width=.6\columnwidth]{problem1_a.png}
    \caption{}
  \end{figure}
  \item The object is virtual, and therefore the image will be real and erected.
  \begin{figure}[h!]
    \centering
    \includegraphics[width=.6\columnwidth]{problem1_b.png}
    \caption{}
  \end{figure}
  \item The object is virtual, and the image will be virtual and inverted. 
  \begin{figure}[h!]
    \centering
    \includegraphics[width=.6\columnwidth]{problem1_c.png}
    \caption{}
  \end{figure}
\end{enumerate}
%%
\section*{Exercise 2}
\begin{enumerate}[itemsep=0pt,topsep=0pt,label=\alph*)]
  \item For a single refracting surface, we have that:
  \begin{itemize}
    \item Both nodal points are located at the center of curvature CC.
    \item Front and real principal planes are located at the vortex.
    \item The reduced thickness of the surface is the focal length of its thin lens representation.
  \end{itemize}
  We illustrate these quantities along with the vertex and the focal lengths in the following figure.
  \begin{figure}[h!]
    \centering
    \includegraphics[width=.6\columnwidth]{problem2_a.png}
    \caption{Illustration of cardinal point for a single refractive surface.}
  \end{figure}
  We illustrate also some quantities of this surface:
  \begin{align*}
    &C=\frac{1}{R}=100\;m^{-1},\quad \phi=(n'-n)C=33.3\;m^{-1},\quad f_E=\frac{1}{\phi}=30\;mm,\\
    &f_F=-nf_E=-30\;mm,\quad f'_R=n'f_E=40\;mm.
  \end{align*}
  \item We use the following equation:
  \begin{align*}
    \frac{n'}{z'}=\frac{n}{z}+\frac{1}{f_E}\longrightarrow z'=\frac{n'zf_E}{nf_E+z}.
  \end{align*}
  Replacing the physical values and the EFL:
  \begin{align*}
    z'=\frac{(1.333)(30)(100)}{(1)(30)+100}=+30.7615\;mm.
  \end{align*}
  Its height is determined by the magnification:
  \begin{align*}
    m=\frac{h'}{h}=\frac{z'/n'}{z/n}=\frac{30.7615/1.333}{100/1}=0.231\longrightarrow h'=mh=(0.231)(10\;mm)=2.31\;mm.
  \end{align*}
  \item The cube is divided in equal part by the optical axis, yielding a height of $h=5\;mm$. Its last side is located $100\;mm$ from the principal planes, as all their 
  sides have the same sizes. Its first face is located $110\;mm$ from the principal plane. Now, the area of one side is $100\;mm^2$.
  We want to find the equivalent (area of volume) of its image.
  We can address this problem by considering two lines at each side of the cube as two independent objects with $(z_1,h_1)$ and $(z_2,h_2)$.
  Then, we do imaging of both to get $(z_1',h_1')$ and $(z_2',h_2')$. The difference between positions and heights allow us to construct the image dimension.
  
  \begin{align*}
    z_1:&\quad z_1'=\frac{(1.333)(30)(110)}{(1)(30)+110}=+31.42\;mm\\
    z_2:&\quad z_2'=\frac{(1.333)(30)(100)}{(1)(30)+100}=+30.76\;mm
  \end{align*}
  We do the same for the magnification to compute the corresponding heights:
  \begin{align*}
    m_1&=\frac{31.42/1.333}{110/1}=0.214\longrightarrow h_1'=m_1h_1=(0.214)(5)=1.07\;mm\\
    m_2&=\frac{30.76/1.333}{100/1}=0.231\longrightarrow h_2'=m_2h_2=(0.231)(5)=1.16\;mm.
  \end{align*}
  The other dimension is demagnified by the magnification $m_1$. The area, would be 
  the integration from $z_1'$ to $z_2'$ with the respective height which can be used to create a linear function (interpolation). However, we will 
  assum the mean value between them to consider it as a constant value. Then, the are is:
  \begin{align*}
    A=\Delta z'\Delta h'=2(z_2'-z_1')(\frac{h_1'+h_2'}{2})=1.472\;mm^2.
  \end{align*} 
  And the volume is this area multiplied by the remaining dimension:
  \begin{align*}
    \Delta x\cdot A=(2m_15)\cdot A=(2\cdot 1.07)(1.472)=3.15\;mm^3.
  \end{align*}
\end{enumerate}

%%
\section*{Exercise 3}
For a two positive lens system, we use Gaussian reduction to reduce the effect to a single thin lens.
We first compute the overall optical power with the power of individual lenses:
\begin{align*}
  \phi=\phi_1+\phi_2-\phi_1\phi_2t=\frac{1}{40}+\frac{1}{40}-\frac{1}{40}\frac{1}{40}\cdot20=0.038mm^{-1}\longrightarrow f_E=\frac{1}{\phi}=26.67\;mm.
\end{align*}
The front and real focal lengths are:
\begin{align*}
    f_F=-n_1f_E=(1)(26.67\;mm)=-26.67\;mm,\quad\text{and}\quad f_R'=n_3f_E=(1)(26.67\;mm)=26.67\;mm.
\end{align*}
Then the distances $d$ and $d'$, corresponding to the shift from the front (rear) principal planes $P,P'$ of the equivalent system with respect to $f_F,f_R'$ are given by
\begin{align*}
  d=\frac{\phi_2}{\phi}t=\frac{0.025}{0.038}20=13.158\;mm,\quad\text{and}\quad d'=-\frac{\phi_1}{\phi}t=-\frac{0.025}{0.038}20=-13.158\;mm.
\end{align*}
The front (back) focal distances are then:
The FFD and BFD are therefore,
\begin{align*}
  \text{FFD}&=f_F+d=-26.67\;mm+13.158\;mm=-13.512\;mm.\\
  \text{BFD}&=f_R'+d'=26.67\;mm-13.512\;mm=13.512\;mm.
\end{align*}
The reduction process and the quantities obtained are illustrated in figure \ref{fig:problem3}.
\begin{figure}[h!]
  \centering
  \includegraphics[width=.6\columnwidth]{problem3.pdf}
  \caption{Gaussian reduction for two positive lenses.}
  \label{fig:problem3}
\end{figure}
%%
\section*{Exercise 4}
In this case we have three surface, each with their correspond surface curvature $C$ and index of refraction $n$.
\begin{itemize}[itemsep=0pt,topsep=0pt]
  \item\textbf{Gaussian reduction} 
  The optical power of each surface is:
  \begin{align*}
    \phi_1&=\frac{n_1-n_0}{R_1}=\frac{1.336-1}{7.8\;mm}=0.043\;mm^{-1},\\
    \phi_2&=\frac{n_2-n_1}{R_2}=\frac{1.413-1.336}{10\;mm}=0.007\;mm^{-1},\\
    \phi_3&=\frac{n_3-n_2}{R_3}=\frac{1.336-1.413}{-6\;mm}=0.013\;mm^{-1}.
  \end{align*}
  Now, we combine surface 1 with 2:
  \begin{align*}
    \phi_{12}=\phi_1+\phi_2-\phi_1\phi_2\tau_1=0.043+0.007+0.043\cdot0.007\cdot\frac{3.6}{1.336}=0.046\;mm^{-1}.
  \end{align*}
  The shift of the principal plane are given by
  \begin{align*}
    \delta_{12}&=\frac{\phi_2}{\phi_{12}}\tau_1=\frac{0.007}{0.046}\cdot\frac{3.6}{1.336}=0.410\;mm\longrightarrow d_{12}=\delta_{12}.\\
    \delta_{12}'&=-\frac{\phi_1}{\phi_{12}}\tau_1=-\frac{0.043}{0.046}\cdot\frac{3.6}{1.336}=-2.519\;mm\longrightarrow d_{12}'=n_2\delta_{12}'=-3.560\;mm.
  \end{align*}
  We can see that the front principal plane is displaced from $V_1$ to the left, while the rear principal plane is shifted to the right of $V_2$.
  In addition, the distance $d_{12}'$ considered the index $n_2$ as it belong to that space.
  The distance of propagation through the index $n_2$ must be adjusted due to the shift of the rear principal plane:
  \begin{align*}
    \tau_{12}=\frac{t_2-d_{12}'}{n_3}=\tau_2-\delta_{12}'=3.6+2.519=6.119\;mm.
  \end{align*}
  Now, we compute the total optical power considering the reduction and the third surface:
  \begin{align*}
    \phi=\phi_{12}+\phi_3-\phi_{12}\phi_3\tau_{12}=0.046+0.013-(0.046)(0.013)(6.119)=0.055\;mm^{-1}.
  \end{align*}
  The shifts are:
  \begin{align*}
    d_{123}&=\delta_{123}=\frac{\phi_3}{\phi}\tau_{12}=\frac{0.013}{0.055}=0.236\;mm\\
    d_{123}'&=n_3\delta_{123}'=-n_3\frac{\phi_{12}}{\phi}\tau_{12}=-(1.336)\frac{0.046}{0.055}=-1.117\;mm.
  \end{align*}
  The total shift from the first surface is the sum of individual fron shift computed, while for the last surface is just the shift computed in the last reduction:
  \begin{align*}
    d&=d_{12}+d_{123}=0.410+0.236=0.646\;mm\\
    d'&=d_{123}'=-1.117\;mm.
  \end{align*}
  The front (rear) focal lengths are then
  \begin{align*}
    f_E=\frac{1}{\phi}=18.18\;mm\longrightarrow \begin{array}{l}
      f_F=-n_0f_E=-(1)(18.18)=-18.182\;mm\\
      f_R'=n_3f_E=(1.336)(18.18)=24.289\;mm.
    \end{array}
  \end{align*}
  Finally, the FFD and BFD are:
  \begin{align*}
    \text{FFD}&=f_F+d_{123}=-18.182+0.646=-17.536\;mm\\
    \text{BFD}&=f_R'+d_{123}'=24.289-1.117=23.172\;mm.
  \end{align*}
  The reduction process is shown in figure \ref{fig:problem4_a}. The green quantities are the equivalent of the final reduction.
  \begin{figure}[h!]
    \centering
    \includegraphics[width=.6\columnwidth]{problem4_a.pdf}
    \caption{Gaussian reduction for the three-surfaces object.}
    \label{fig:problem4_a}
  \end{figure}
  \item\textbf{Ray tracing}
  For the ray tracing, we will fill the ynu spreadsheet. We see the object space, three surfaces, two intermediate spaces, and the image space. Therefore,
  the table will have five columns.
  We will trace two rays, one from left to right and other in opposite direction in order to find the front and real focal lengths.

  \begin{table}[h!]
    \centering
    \begin{tabular}{l|l|l|l|l|l|l|l|l|l}
      &\shortstack{Object\\space}&Space 1&Surface 1&Space 2&Surface 2&Space 3&Surface 3&Space 4&\shortstack{Image\\space}\\
      \hline
      $C$&&&0.128&&0.1&&-0.167&&\\
      $t$&&\textcolor{blue}{15.79}&&3.6&&3.6&&\textcolor{red}{12.864}&\\
      $n$&&1&&1.336&&1.413&&1.336&\\
      \hline 
      $-\phi$&&&-0.043&&-0.007&&-0.013&&\\
      $t/n$&&\textcolor{blue}{15.79}&&2.695&&2.547&&\textcolor{red}{12.864}&\\
      \hline 
      $y$&1&1&1&&0.884&&0.759&&0\\
      $nu$&0&0&&-0.043&&-0.0492&&-0.059&\\
      $u$&0&0&&&&&&-0.044&\\
      \hline
      $y$&0&&0.916&&0.967&&1&1&1\\
      $nu$&&0.058&&0.019&&0.013&&0&0\\
      $u$&&0.058&&&&&&0&0
    \end{tabular}
  \end{table}
  \end{itemize}


%%
\section*{Exercise 5}
For this problem, we must consider the reflection which translates to a negative index and a negativa reduced thickness. We will asume that 
oter media is air ($n=1$).
  \begin{table}[h!]
    \centering
    \begin{tabular}{l|l|l|l|l|l|l|l|l|l}
      &\shortstack{Object\\space}&Space 1&Surface 1&Space 2&Surface 2&Space 3&Surface 3&Space 4&\shortstack{Image\\space}\\
      \hline
      $C$&&&-0.01&&-0.006&&-0.01&&\\
      $t$&&\textcolor{blue}{127.667}&&10&&-10&&\textcolor{red}{-25.459}&\\
      $n$&&1&&1.5&&-1.5&&-1&\\
      \hline 
      $-\phi$&&&0.005&&-0.018&&-0.025&&\\
      $t/n$&&\textcolor{blue}{127.667}&&6.667&&6.667&&\textcolor{red}{25.459}&\\
      \hline 
      $y$&1&1&1&&1.033&&0.942&&0\\
      $nu$&0&0&&0.005&&-0.0136&&-0.037&\\
      $u$&0&0&&&&&&-0.037&\\
      \hline
      $y$&0&&0.766&&0.833&&1&1&1\\
      $nu$&&0.006&&0.010&&0.025&&0&0\\
      $u$&&0.006&&&&&&0&0
    \end{tabular}
  \end{table}


%\nocite{*}
%\bibliographystyle{plain}   % or unsrt, alpha, apalike, etc.
%\bibliography{refs}

\end{document}
