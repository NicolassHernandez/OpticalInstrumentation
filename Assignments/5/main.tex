\documentclass[letterpaper,11pt,twoside]{article}
\usepackage{graphicx} % Required for inserting images
\usepackage[table,xcdraw,dvipsnames]{xcolor}
\usepackage{amsmath,amsfonts,amssymb,amsthm}
\usepackage{listings}
\usepackage{lipsum}
\usepackage{hyperref}
\usepackage{enumitem}

\usepackage{tikz}
\usepackage[siunitx, RPvoltages]{circuitikz}
\usetikzlibrary{3d}
\usepackage{comment}
\usepackage{caption,subcaption}
\usepackage{pgfplots}
\pgfplotsset{compat=newest} % or a newer version if available
\usepgfplotslibrary{groupplots}
\usetikzlibrary{pgfplots.groupplots}
\usetikzlibrary{shapes.geometric, arrows}
\tikzstyle{arrow} = [->,>=stealth,shorten >=2pt]
\tikzstyle{darrow}=[<->,shorten >=2pt,shorten <=2pt,>=stealth]
\usepackage{cancel}
\usepackage{bm}
\usepackage{fancyhdr}
\usepackage[utf8x]{inputenc}
\usepackage[T1]{fontenc}
\usepackage[margin=0.8in,top=1in,bottom=1in]{geometry}
%%%%%
\begin{filecontents*}{refs.bib}
@book{bornwolf,
  author    = {Born, M. and Wolf, E.},
  title     = {Principles of Optics},
  publisher = {Pergamon Press},
  edition   = {7},
  year      = {1999}
}
@book{hecht,
  author    = {Hecht, E.},
  title     = {Optics},
  publisher = {Addison-Wesley},
  edition   = {5},
  year      = {2016}
}
\end{filecontents*}
%
\newcommand{\institution}{University of Arizona}
\newcommand{\autor}{Nicolás Hernández Alegría}
\newcommand{\course}{OPTI 502 Optical Design and Instrumentation I}
\newcommand{\assignment}{Assignment 5}
%
\title{\textbf{\assignment}\\\course\\{\Large\institution}}
\author{\autor}
\date{\today}
%
\renewcommand{\sectionmark}[1]{\markright{#1}}
\fancypagestyle{mainstyle}{
    \fancyhf{} % Clear all header and footer fields
    \fancyfoot[C]{\thepage}
    \fancyhead[LE,RO]{\course} % Section name on odd pages
    \fancyhead[LO,RE]{\assignment}
    % Optional: Thin rules
    \renewcommand{\headrulewidth}{0pt} % Header rule
    \renewcommand{\footrulewidth}{0pt} % No footer rule
}
%
\begin{document}

\pagestyle{mainstyle}
\maketitle
%%
\section*{Exercise 1}
\begin{enumerate}[itemsep=0pt,topsep=0pt,label=\alph*)]
  \item In this case, the object is virtual and the image will be real and erected (figure \ref{fig:problem1_a}).
  \begin{figure}[h!]
    \centering
    \includegraphics[width=.6\columnwidth]{problem1_a.png}
    \caption{Real demagnified and erected image generated by a virtual object away from $F'$.}
    \label{fig:problem1_a}
  \end{figure}
  \item The object is virtual, and therefore the image will be real and erected (figure \ref{fig:problem1_b}).
  \begin{figure}[h!]
    \centering
    \includegraphics[width=.6\columnwidth]{problem1_b.png}
    \caption{Real magnified and erected image from a virtual object within $F$.}
    \label{fig:problem1_b}
  \end{figure}
  \item The object is virtual, and the image will be virtual and inverted (figure \ref{fig:problem1_c}). 
  \begin{figure}[h!]
    \centering
    \includegraphics[width=.6\columnwidth]{problem1_c.png}
    \caption{Virtual inverted and magnified image from a virtual object far from $F$.}
    \label{fig:problem1_c}
  \end{figure}
\end{enumerate}
%%
\section*{Exercise 2}
\begin{enumerate}[itemsep=0pt,topsep=0pt,label=\alph*)]
  \item For a single refracting surface, we have that:
  \begin{itemize}
    \item Both nodal points are located at the center of curvature CC.
    \item Front and real principal planes are located at the vortex.
    \item The reduced thickness of the surface is the focal length of its thin lens representation.
  \end{itemize}
  We illustrate these quantities along with the vertex and the focal lengths in the following figure.
  \begin{figure}[h!]
    \centering
    \includegraphics[width=.6\columnwidth]{problem2_a.png}
    \caption{Illustration of cardinal point for a single refractive surface.}
  \end{figure}
  We illustrate also some quantities of this surface:
  \begin{align*}
    &C=\frac{1}{R}=100\;m^{-1},\quad \phi=(n'-n)C=33.3\;m^{-1},\quad f_E=\frac{1}{\phi}=30\;mm,\\
    &f_F=-nf_E=-30\;mm,\quad f'_R=n'f_E=40\;mm.
  \end{align*}
  \item We use the following equation:
  \begin{align*}
    \frac{n'}{z'}=\frac{n}{z}+\frac{1}{f_E}\longrightarrow z'=\frac{n'zf_E}{nf_E+z}.
  \end{align*}
  Replacing the physical values and the EFL:
  \begin{align*}
    z'=\frac{(1.333)(30)(-100)}{(1)(30)-100}=+57.129\;mm.
  \end{align*}
  Its height is determined by the magnification:
  \begin{align*}
    m=\frac{h'}{h}=\frac{z'/n'}{z/n}=\frac{57.129/1.333}{-100/1}=-0.429\longrightarrow h'=mh=(-0.429)(10\;mm)=-4.29\;mm.
  \end{align*}
  \item The cube is divided in equal part by the optical axis, yielding a height of $h=5\;mm$. Its last side is located $100\;mm$ from the principal planes, as all their 
  sides have the same sizes. Its first face is located $110\;mm$ from the principal plane. Now, the area of one side is $100\;mm^2$.
  We want to find the equivalent (area of volume) of its image.
  We can address this problem by considering two lines at each side of the cube as two independent objects with $(z_1,h_1)$ and $(z_2,h_2)$.
  Then, we do imaging of both to get $(z_1',h_1')$ and $(z_2',h_2')$. The difference between positions and heights allow us to construct the image dimension.
  
  \begin{align*}
    z_1:&\quad z_1'=\frac{(1.333)(30)(-110)}{(1)(30)-110}=+54.986\;mm\\
    z_2:&\quad z_2'=\frac{(1.333)(30)(-100)}{(1)(30)-100}=+57.129\;mm
  \end{align*}
  We do the same for the magnification to compute the corresponding heights:
  \begin{align*}
    m_1&=\frac{54.9862/1.333}{-110/1}=-0.375\longrightarrow h_1'=m_1h_1=(-0.375)(5)=-1.875\;mm\\
    m_2&=\frac{57.129/1.333}{-100/1}=-0.429\longrightarrow h_2'=m_2h_2=(-0.429)(5)=-2.145\;mm.
  \end{align*}
  With these two distances and heights, we can construct the object volume. The remaining dimension perpendicular to $z$ (optical axis) is affected in exactly the same way.
  For the difference in $z$, we have 
  \begin{align*}
    \Delta z=z'_2-z'_1=2.143\;mm.
  \end{align*}
  For the heights, we will use the mean value between both to define an equivalent cube (the final volume will not be affected). Also, we include a factor $2$ as 
  we only considered half of the height. Therefore, 
  \begin{align*}
    \Delta h=|h_1+h_2|=4.02\;mm.
  \end{align*} 
  The $x$ dimension, will be affected in the same form as the heights:
  \begin{align*}
    \Delta x=\Delta h.
  \end{align*}
  Finally, the volume of the image is:
  \begin{align*}
    \Delta x\Delta h\Delta z=(4.02)(4.02)(2.143)=34.632\;mm^3.
  \end{align*}
  If we divide this volume by the volume of the object, we will have exactly the product of the magnification for each dimension, that is,
  \begin{align*}
    m_xm_ym_z=\frac{34.623}{10^3}=0.0346.
  \end{align*}
  On the other hand, the illustration of the image dimension is shown in figure \ref{fig:problem2_b}.
  \begin{figure}[h!]
    \centering
    \includegraphics[width=.6\columnwidth]{problem2_b.png}
    \caption{Image form and dimension given the object.}
    \label{fig:problem2_b}
  \end{figure}
\end{enumerate}

%%
\section*{Exercise 3}
For a two positive lens system, we use Gaussian reduction to reduce the effect to a single thin lens.
We first compute the overall optical power with the power of individual lenses:
\begin{align*}
  \phi=\phi_1+\phi_2-\phi_1\phi_2t=\frac{1}{40}+\frac{1}{40}-\frac{1}{40}\frac{1}{40}\cdot20=0.038mm^{-1}\longrightarrow f_E=\frac{1}{\phi}=26.67\;mm.
\end{align*}
The front and real focal lengths are:
\begin{align*}
    f_F=-n_1f_E=(1)(26.67\;mm)=-26.67\;mm,\quad\text{and}\quad f_R'=n_3f_E=(1)(26.67\;mm)=26.67\;mm.
\end{align*}
Then the distances $d$ and $d'$, corresponding to the shift from the front (rear) principal planes $P,P'$ of the equivalent system with respect to $f_F,f_R'$ are given by
\begin{align*}
  d=\frac{\phi_2}{\phi}t=\frac{0.025}{0.038}20=13.158\;mm,\quad\text{and}\quad d'=-\frac{\phi_1}{\phi}t=-\frac{0.025}{0.038}20=-13.158\;mm.
\end{align*}
The front (back) focal distances are then:
The FFD and BFD are therefore,
\begin{align*}
  \text{FFD}&=f_F+d=-26.67\;mm+13.158\;mm=-13.512\;mm.\\
  \text{BFD}&=f_R'+d'=26.67\;mm-13.512\;mm=13.512\;mm.
\end{align*}
The reduction process and the quantities obtained are illustrated in figure \ref{fig:problem3}.
\begin{figure}[h!]
  \centering
  \includegraphics[width=.6\columnwidth]{problem3.pdf}
  \caption{Gaussian reduction for two positive lenses.}
  \label{fig:problem3}
\end{figure}
%%
\section*{Exercise 4}
In this case we have three surface, each with their correspond surface curvature $C$ and index of refraction $n$.
\begin{itemize}[itemsep=0pt,topsep=0pt]
  \item\textbf{Gaussian reduction} 
  The optical power of each surface is:
  \begin{align*}
    \phi_1&=\frac{n_1-n_0}{R_1}=\frac{1.336-1}{7.8\;mm}=0.043\;mm^{-1},\\
    \phi_2&=\frac{n_2-n_1}{R_2}=\frac{1.413-1.336}{10\;mm}=0.008\;mm^{-1},\\
    \phi_3&=\frac{n_3-n_2}{R_3}=\frac{1.336-1.413}{-6\;mm}=0.013\;mm^{-1}.
  \end{align*}
  Now, we combine surface 1 with 2:
  \begin{align*}
    \phi_{12}=\phi_1+\phi_2-\phi_1\phi_2\tau_1=0.043+0.008-0.043\cdot0.008\cdot\frac{3.6}{1.336}=0.050\;mm^{-1}.
  \end{align*}
  The shift of the principal plane are given by
  \begin{align*}
    \delta_{12}&=\frac{\phi_2}{\phi_{12}}\tau_1=\frac{0.008}{0.050}\cdot\frac{3.6}{1.336}=0.431\;mm\longrightarrow d_{12}=\delta_{12}.\\
    \delta_{12}'&=-\frac{\phi_1}{\phi_{12}}\tau_1=-\frac{0.043}{0.050}\cdot\frac{3.6}{1.336}=-2.317\;mm\longrightarrow d_{12}'=n_2\delta_{12}'=-3.274\;mm.
  \end{align*}
  We can see that the front principal plane is displaced from $V_1$ to the left, while the rear principal plane is shifted to the right of $V_2$.
  In addition, the distance $d_{12}'$ considered the index $n_2$ as it belong to that space.
  The distance of propagation through the index $n_2$ must be adjusted due to the shift of the rear principal plane:
  \begin{align*}
    \tau_{12}=\frac{t_2-d_{12}'}{n_3}=\tau_2-\delta_{12}'=\frac{3.6}{1.413}+2.317=4.865\;mm.
  \end{align*}
  Now, we compute the total optical power considering the reduction and the third surface:
  \begin{align*}
    \phi=\phi_{12}+\phi_3-\phi_{12}\phi_3\tau_{12}=0.050+0.013-(0.046)(0.013)(4.865)=0.060\;mm^{-1}.
  \end{align*}
  The shifts are:
  \begin{align*}
    d_{123}&=n_0\delta_{123}=\frac{\phi_3}{\phi}\tau_{12}=\frac{0.013}{0.060}\cdot4.865=1.054\;mm\\
    d_{123}'&=n_3\delta_{123}'=-n_3\frac{\phi_{12}}{\phi}\tau_{12}=-(1.336)\frac{0.050}{0.060}\cdot4.865=-5.416\;mm.
  \end{align*}
  The total shift from the first surface is the sum of individual fron shift computed, while for the last surface is just the shift computed in the last reduction:
  \begin{align*}
    d&=d_{12}+d_{123}=0.431+1.054=1.485\;mm\\
    d'&=d_{123}'=-5.416\;mm.
  \end{align*}
  The front (rear) focal lengths are then
  \begin{align*}
    f_E=\frac{1}{\phi}=16.667\;mm\longrightarrow \begin{array}{l}
      f_F=-n_0f_E=-(1)(16.667)=-16.667\;mm\\
      f_R'=n_3f_E=(1.336)(16.667)=22.267\;mm.
    \end{array}
  \end{align*}
  Finally, the FFD and BFD are:
  \begin{align*}
    \text{FFD}&=f_F+d_{123}=-16.667+1.054=15.613\;mm\\
    \text{BFD}&=f_R'+d_{123}'=22.267-5.416=16.851\;mm.
  \end{align*}
  The reduction process is shown in figure \ref{fig:problem4_a}. The green quantities are the equivalent of the final reduction.
  \begin{figure}[h!]
    \centering
    \includegraphics[width=.6\columnwidth]{problem4_a.pdf}
    \caption{Gaussian reduction for the three-surfaces object.}
    \label{fig:problem4_a}
  \end{figure}
  \item\textbf{Ray tracing}
  For the ray tracing, we will fill the ynu spreadsheet. We will trace two rays, one from left to right and other in opposite direction in order to find 
  the front and real focal lengths.
  \begin{table}[h!]
    \centering
    \begin{tabular}{l|l|l|l|l|l|l|l|l|l}
      &\shortstack{Object\\space}&Space 1&Surface 1&Space 2&Surface 2&Space 3&Surface 3&Space 4&\shortstack{Image\\space}\\
      \hline
      $C$&&&0.128&&0.1&&-0.167&&\\
      $t$&&\textcolor{blue}{15.167}&&3.6&&3.6&&\textcolor{red}{16.856}&\\
      $n$&&1&&1.336&&1.413&&1.336&\\
      \hline 
      $-\phi$&&&-0.043&&-0.008&&-0.013&&\\
      $t/n$&&\textcolor{blue}{15.167}&&2.695&&2.548&&\textcolor{red}{12.617}&\\
      \hline 
      $y$&1&1&1&&0.884&&0.757&&0\\
      $nu$&0&0&&-0.043&&-0.05&&-0.060&\\
      $u$&0&0&&&&&&-0.045&\\
      \hline
      $y$&0&&0.910&&0.967&&1&1&1\\
      $nu$&&0.060&&0.021&&0.013&&0&0\\
      $u$&&0.060&&&&&&0&0
    \end{tabular}
  \end{table}
  \end{itemize}
  We must compare the $t$ in blue with the FFD and the red $t$ with the BFD. The differences are due to the approximation in intermediate computations. 
  We can see that both methods yield the same answer, despite that ynu raytracing is way faster than Gaussian reduction.

  The effective focal length is defined considering the magnification $nu$ divided by the input ray:
  \begin{align*}
    f_E'=\frac{1}{\phi'}=-\frac{y_1}{nu'}=\frac{1}{0.060}=16.667\;mm\longrightarrow f_R'=n_3f_E'=22.267\;mm.
  \end{align*}
  Similarly,
  \begin{align*}
    f_E'=\frac{1}{\phi}=\frac{y_2}{nu}=\frac{1}{0.060}=16.667\;mm\longrightarrow f_F=-n_0f_E=-16.667\;mm.
  \end{align*}
  The focal lengths match exactly as the ones computed by Gaussian reduction. We can also compute the principal planes shifts, but we will not do it as we already know the answer.
%%
\section*{Exercise 5}
\begin{enumerate}[itemsep=0pt,topsep=0pt,label=\alph*)]
  \item 
For this problem, we must consider the reflection with a negative index and a negative distante $t$. We will asume that $n=1$.
The first media after the object space is the one with $n=1.5$ which has a negative curvature $C_1$. Then, a mirror with negative $C_2$ reflects the ray where 
a negative distance must be considered. After refracts again in the surface $1$, the air propagation extend to infinity. 
The following table illustrates the raytracing, where a parallel ray is propagated and the distance from the last refraction is found by setting $y=0$.
\begin{table}[h!]
  \centering
  \begin{tabular}{l|l|l|l|l|l|l|l|l|l}
    &\shortstack{Object\\space}&Space 1&Surface 1&Space 2&Surface 2&Space 3&Surface 3&Space 4&\shortstack{Image\\space}\\
    \hline
    $C$&&&-0.01&&-0.006&&-0.01&&\\
    $t$&&\textcolor{blue}{84.182}&&10&&-10&&\textcolor{red}{-84.182}&\\
    $n$&&1&&1.5&&-1.5&&-1&\\
    \hline 
    $-\phi$&&&0.005&&-0.02&&0.005&&\\
    $t/n$&&\textcolor{blue}{84.182}&&6.667&&6.667&&\textcolor{red}{84.182}&\\
    \hline 
    $y$&1&1&1&&1.033&&0.926&&0\\
    $nu$&0&0&&0.005&&-0.016&&-0.011&\\
    $u$&0&0&&&&&&-0.011&\\
    \hline
    $y$&0&&0.926&&1.033&&1&1&1\\
    $nu$&&0.011&&0.016&&-0.005&&0&0\\
    $u$&&0.011&&&&&&0&0
  \end{tabular}
\end{table}

From the forward ray, we see that 
\begin{align*}
  f'_{E}=\frac{1}{\phi'}=-\frac{y_1}{nu'}=-\frac{1}{-0.011}=90.909\;mm.
\end{align*}
Therefore, the rear focal length is:
\begin{align*}
  f_{R'}=nf'_{E}=(-1)(90.909)=-90.909\;mm.
\end{align*}
The shift from the rear principal plane $P'$ and the last surface is computed as the following substraction:
\begin{align*}
  d'=\text{BFD}-f'_R=-84.182+90.909=6.727\;mm.
\end{align*}
Therefore, $P'$ is to the right of the last surface.

\item With the backward ray, we have the same properties for the front cardinal points:
\begin{align*}
  \text{FFD}&=-84.182\;mm\\
  f_E&=\frac{1}{\phi}=\frac{y_2}{nu}=\frac{1}{0.011}=90.909\;mm.\\
  f_F&=-nf_E=-90.909\;mm.\\
  d&=\text{FFD}-f_F=-84.182+90.909=6.727\;mm.
\end{align*}
The FFD is located to the left of the first surface and the fron principal plane is shifted to the right by $d$.
\end{enumerate}

%\nocite{*}
%\bibliographystyle{plain}   % or unsrt, alpha, apalike, etc.
%\bibliography{refs}

\end{document}
