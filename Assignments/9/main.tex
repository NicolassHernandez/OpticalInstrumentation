\documentclass[letterpaper,11pt,twoside]{article}
\usepackage{graphicx} % Required for inserting images
\usepackage[table,xcdraw,dvipsnames]{xcolor}
\usepackage{amsmath,amsfonts,amssymb,amsthm}
\usepackage{listings}
\usepackage{lipsum}
\usepackage{hyperref}
\usepackage{enumitem}
\usepackage{pdflscape}
\usepackage{rotating}
\usepackage{tikz}
\usepackage[siunitx, RPvoltages]{circuitikz}
\usetikzlibrary{3d}
\usepackage{comment}
\usepackage{caption,subcaption}
\usepackage{pgfplots}
\pgfplotsset{compat=newest} % or a newer version if available
\usepgfplotslibrary{groupplots}
\usetikzlibrary{pgfplots.groupplots}
\usetikzlibrary{shapes.geometric, arrows}
\tikzstyle{arrow} = [->,>=stealth,shorten >=2pt]
\tikzstyle{darrow}=[<->,shorten >=2pt,shorten <=2pt,>=stealth]
\usepackage{cancel}
\usepackage{bm}
\usepackage{fancyhdr}
\usepackage[utf8x]{inputenc}
\usepackage[T1]{fontenc}
\usepackage[margin=0.8in,top=1in,bottom=1in]{geometry}
%%%%%
\begin{filecontents*}{refs.bib}
@book{bornwolf,
  author    = {Born, M. and Wolf, E.},
  title     = {Principles of Optics},
  publisher = {Pergamon Press},
  edition   = {7},
  year      = {1999}
}
@book{hecht,
  author    = {Hecht, E.},
  title     = {Optics},
  publisher = {Addison-Wesley},
  edition   = {5},
  year      = {2016}
}
\end{filecontents*}
%
\newcommand{\institution}{University of Arizona}
\newcommand{\autor}{Nicolás Hernández Alegría}
\newcommand{\course}{OPTI 502 Optical Design and Instrumentation I}
\newcommand{\assignment}{Assignment 9}
%
\title{\textbf{\assignment}\\\course\\{\Large\institution}}
\author{\autor}
\date{\today}
%
\renewcommand{\sectionmark}[1]{\markright{#1}}
\fancypagestyle{mainstyle}{
    \fancyhf{} % Clear all header and footer fields
    \fancyfoot[C]{\thepage}
    \fancyhead[LE,RO]{\course} % Section name on odd pages
    \fancyhead[LO,RE]{\assignment}
    % Optional: Thin rules
    \renewcommand{\headrulewidth}{0pt} % Header rule
    \renewcommand{\footrulewidth}{0pt} % No footer rule
}
%
\begin{document}

\pagestyle{mainstyle}
\maketitle
%%
\section*{Exercise 1}
We use raytracing with the chief and marginal ray and the information given. Recall that the stop is placed at 
the primary mirror, so the chief ray must have 0 height at that location. Also, we put a \textbf{dummy surface} at the primary mirror 
to use it for vignetting, and working
\begin{table}[h!]
  \centering
  \resizebox{\columnwidth}{!}{%
  \begin{tabular}{ll|l|l|l|l|l|l|l|l|l|l|l|l}
    &&\shortstack{Object\\space}&EP&&$M_1$ (stop)&&$M_2$&&Dummy&&XP&&\shortstack{Image\\space}\\
    \hline
    &$R$&&&&$-500$&&$-125$&&$\infty$&&&&\\
    &$t$&&&$z_{EP}=$&&$-200$&&$200$&&$=$&&&\\
    &$n$&1&&1&&-1&&1&&1&&1\\
    \hline 
    &$-\phi$&&&&$-0.004$&&$0.016$&&$0$&&$0$&&\\
    &$t/n$&&&$\tau_{EP}=$&&$200$&&$200$&&$-247.619$&&$297.625$&\\
    \hline 
    &$y$&&&&\textcolor{red}{$0$}&&$20$&&$104$&&\textcolor{red}{$0$}&&\\
    CR&$nu$&&&\textcolor{red}{$0.1$}&&$0.1$&&$0.42$&&$0.42$&&&\\
    &$u$&&&&&&&&&&&&\\
    \hline
    &$y$&&$R_{EP}=$&&\textcolor{red}{$1$}&&$0.2$&&$0.04$&&$R_{XP}=0.2381$&&\textcolor{red}{$0$}\\
    MR&$nu$&&&\textcolor{red}{$0$}&&$-0.004$&&$-0.0008$&&$-0.0008$&&$-0.0008$&\\
    &$u$&&&&&&&&&&&&
  \end{tabular}}
  \caption{Raytrace, with CR=Chief ray, MR=Marginal ray.}
\end{table}

From the marginal ray, we know that the effective focal length is:
\begin{align*}
  f=-\frac{y_1}{\omega_k}=-\frac{1}{-0.0008}=1250\;mm.
\end{align*}
The working distance is the sum of the distance from the dummy surface to the exit pupil and the distance from the exit pupil to the image location:
\begin{align*}
  \text{WD}=-247.619+297.625\approx50\;mm.
\end{align*}
By looking the distance of the dumy to the exit pupil, we say that the exit pupil is located $247.619\;mm$ to the left of the primary mirror.
The diameter of the pupils are obtained through the f-number requirement.
\begin{align*}
  f/\#=4=\frac{f}{D_{EP}}\longrightarrow D_{EP}=\frac{1250}{4}=312.5\;mm.
\end{align*}



%%
\section*{Exercise 2}
A relaxed eye means that the intermediate image must be at the front focal plane $f_{EP}$ of the eyepiece so that the 
rays output parallel. The magnifying power is:
\begin{align*}
  MP=\frac{250\;mm}{f_{EP}}=10\longrightarrow f_{EP}=25\;mm.
\end{align*}
\begin{enumerate}[itemsep=0pt,topsep=0pt,label=\alph*)]
  \item For a simple eyepiece, the $f_{EP}$ corresponds to the front focal length of the eyepiece: $f_{eye}=f_{EP}$.
  Because the stop is $200\;mm$ to the left of the intermediate image, which is $25\;mm$ to the left of the eyepiece, the total distance from this lens is:
  \begin{align*}
    z=-200-f_{eye}=-200-25=-225\;mm.
  \end{align*}
  Using the thin-lens equation, we find the distance of the object and therefore the eye relief:
  \begin{align*}
    \frac{1}{ER}=\frac{1}{z}+\frac{1}{f_{eye}}\longrightarrow ER=\frac{1}{\frac{1}{-225}+\frac{1}{25}}=28.125\;mm.
  \end{align*}
  \item In the case of a compound eyepiece, the field lens is placed at one front focal length from the eye lens. This implies that only the 
  rear principal plane and the ER change from the simple eyepiece discussed previously. It is shifted by the following amount:
  \begin{align*}
    d'=-\frac{\phi_F}{\phi_{EP}}t=-\frac{f^2_{eye}}{f_{F}}=-\frac{25^2}{40}=-15.625\;mm.
  \end{align*}
  The eyerelief is then shifted by the same amount to the left:
  \begin{align*}
    ER=28.125-15.625=12.5\;mm.
  \end{align*}
  \item In the Ramsden eyepiece, we need to have the same eye relief of part b) knowing that the field lens is located $12\;mm$ to the right of the intermediate image plane.
  Recall also that we have to achieve the magnyfying power and the image location, so we have three conditions to meet.
  \begin{align*}
    f_{EP}=25\;mm,\quad d'=-\frac{\phi_F}{\phi_{eye}}t,\quad f_{EP}=12+d=25\;mm,\quad \frac{1}{z'}=\frac{1}{z}+\frac{1}{f_{EP}}.
  \end{align*}
  First, the focal length $f_{EP}$ allows us to obtain the shift of the front principal plane:
  \begin{align*}
    f_{EP}=12+d=25\longrightarrow d=13\;mm.
  \end{align*}
  This shifting is equal to the ratio of powers, which enables us to solve for the power of the eye:
  \begin{align*}
    d=13=\frac{\phi_{eye}}{\phi_{EP}}t\longrightarrow \phi_{eye}t=\frac{d}{f_{EP}}=\frac{13}{25}=0.52.
  \end{align*}
  The distance of the stop to the front principal plane is:
  \begin{align*}
    z=-200-23-d=-225\;mm.
  \end{align*}
  Using the thin-lens equation:
  \begin{align*}
    \frac{1}{z'}=\frac{1}{z}+\frac{1}{f_{EP}}\longrightarrow z'=28.215\;mm.
  \end{align*}
  This distance is from the rear principal plane to the exit pupil so we can extract the shifrt $d'$:
  \begin{align*}
    ER-d'=28.215\longrightarrow d'=-15.625\;mm.
  \end{align*}
  We can relate it to tits formula and get the term $\phi_Ft$:
  \begin{align*}
    d'=-\frac{\phi_F}{\phi_{EP}}t\longrightarrow \phi_Ft=-\frac{d'}{f_{EP}}=0.625.
  \end{align*}
  Using the overal power multiplied by $t$ allow us to get this distance,
  \begin{align*}
    \phi_{EP}t=\phi_Ft+\phi_{eye}t-\phi_F\phi_{eye}t^2=0.625+0.52-0.625\cdot0.52=0.82.
  \end{align*}
  Therefore, 
  \begin{align*}
    t=\frac{0.82}{\phi_{EP}}=0.82f_{EP}\longrightarrow t=20.5\;mm.
  \end{align*}
  With this distance, we now have the focal length of the eye and the field lens:
  \begin{align*}
    \phi_{eye}&=\frac{0.52}{20.5}=\frac{1}{f_{eye}}\longrightarrow f_{eye}=39.42\;mm\\
    \phi_{F}&=\frac{0.625}{20.5}=\frac{1}{f_{F}}\longrightarrow f_F=32.8\;mm.
  \end{align*}
\end{enumerate}


%\nocite{*}
%\bibliographystyle{plain}   % or unsrt, alpha, apalike, etc.
%\bibliography{refs}

\end{document}