\section*{Formula sheet}
%%
\columnratio{.3}
\begin{paracol}{2}
\subsection*{$z'$ and $m$ curves}
\begin{figure}[h!]
\centering
\begin{subfigure}{\columnwidth}
    \centering
    \begin{circuitikz}[scale=.6]
    \def\f{1}
    \draw[arrow](-4,0)--(4,0)node[right]{$z$};
    \draw[arrow](0,-4)--(0,4)node[above]{$z'/m$};
    \draw[very thick,NavyBlue,domain=-3:-1.35,samples=100] plot(\x,{ (\x*\f)/(\x+\f) });
    \draw[very thick,NavyBlue,domain=-.8:3,samples=100] plot(\x,{ (\x*\f)/(\x+\f) });
    \draw[very thick,OliveGreen,domain=-3:-1.25,samples=100] plot(\x,{ (\f)/(\x+\f) });
    \draw[very thick,OliveGreen,domain=-.75:3,samples=100] plot(\x,{ (\f)/(\x+\f) });
    \draw[dashed](-4,1)--(4,1)(-1,4)--(-1,-4);
    \foreach \t in{-3,...,3}{\draw(\t,.1)--(\t,-.1)node[below]{$\t$};}
    \foreach \y in{-3,-2,-1,1,2,3}{\draw(-.1,\y)--(.1,\y)node[right,fill=white]{$\y$};}
    \begin{scope}[shift={(1.5,3.5)}] % move legend to top-right
        \draw[very thick,NavyBlue](0,0)--(0.8,0)node[right,black]{$z'$};
        \draw[very thick,OliveGreen](0,-0.5)--(0.8,-0.5)node[right,black]{$m$};
    \end{scope}
    \end{circuitikz}
    \caption{$f>0$}
    \label{fig:2a1}
\end{subfigure}
\hfill
\begin{subfigure}{\columnwidth}
    \centering
    \begin{circuitikz}[scale=.6]
    \def\f{-1}
    \draw[arrow](-4,0)--(4,0)node[right]{$z$};
    \draw[arrow](0,-4)--(0,4)node[above]{$z'$};
    \draw[very thick,NavyBlue,domain=-3:.8,samples=100] plot(\x,{ (\x*\f)/(\x+\f) });
    \draw[very thick,NavyBlue,domain=1.35:3,samples=100] plot(\x,{ (\x*\f)/(\x+\f) });
    \draw[very thick,OliveGreen,domain=-3:.75,samples=100] plot(\x,{ (\f)/(\x+\f) });
    \draw[very thick,OliveGreen,domain=1.25:3,samples=100] plot(\x,{ (\f)/(\x+\f) });
    \draw[dashed](-4,-1)--(4,-1)(1,4)--(1,-4);
    \foreach \t in{-3,...,3}{\draw(\t,-.1)--(\t,.1)node[above]{$\t$};}
    \foreach \y in{-3,-2,-1,1,2,3}{\draw(.1,\y)--(-.1,\y)node[left,fill=white]{$\y$};}
    \begin{scope}[shift={(1.5,3.5)}] % move legend to top-right
        \draw[very thick,NavyBlue](0,0)--(0.8,0)node[right,black]{$z'$};
        \draw[very thick,OliveGreen](0,-0.5)--(0.8,-0.5)node[right,black]{$m$};
    \end{scope}
    \end{circuitikz}
    \caption{$f<0$}
    \label{fig:2a2}
\end{subfigure}
\end{figure}
\switchcolumn
\subsection*{Useful formulas}
\begin{table}[h!]
    \centering
    \renewcommand{\arraystretch}{1.2}
    \begin{tabular}{l|l|l}
        $\nu=\frac{V}{\lambda}=\frac{1}{T}=\frac{V}{\lambda}$&$n=\frac{c}{V}$&$\text{OPL}=\int_a^b\bm{n}(s)\cdot d\bm{s}$\\
        $\frac{1}{z'}=\frac{1}{z}+\frac{1}{f}$&$m=\frac{z'}{z}=\frac{h'}{h}$&$m_{\text{total}}=\prod_im_i$\\
        $n_1\sin\theta_1=n_2\sin\theta_2$&$\theta_2=-\theta_1$&$\theta_i>\theta_c=\sin^{-1}n_2/n_1$\\
        $d=\frac{n-1}{n}t$&$D\approx-t\theta\frac{n-1}{n}$&$\tau=t/n$\\
        \multicolumn{3}{l}{($f>0$) $|z|\gg f\Longrightarrow z'\approx f\land m\approx f/z\land L=z'-z\approx -z$}\\
        \multicolumn{3}{l}{($f>0$) $|z'|\gg f\Longrightarrow z\approx-f\land m\approx -z'/f\land L\approx z'$}\\
        Afocal $m=\frac{h'}{h}=\frac{-f_2}{f_1}$&Keplerian $m<0$&Galilean $m>0$\\
        $\phi=(n'-n)C$&$C=1/R$&$n'u'=nu-y\phi$\\
        $f=f_E=1/\phi$&$f_F=-nf_E$&$f'_R=n'f_E$\\
        $\frac{n'}{z'}=\frac{n}{z}+\phi$&\multicolumn{2}{l}{$\frac{1}{f}=(n-1)(1/R_1-1/R_2)$}\\
        \multicolumn{3}{l}{$\frac{1}{f}=(n-1)[1/R_1+1/R_2-\frac{(n-1)}{R_1R_2}\frac{t}{n}]$}
    \end{tabular}
\end{table}
\end{paracol}

\begin{figure}[h!]
    \centering
    \includegraphics[width=.2\columnwidth]{figures/snell.png}
    \includegraphics[width=.2\columnwidth]{figures/ppp1.png}
    \includegraphics[width=.2\columnwidth]{figures/ppp2.png}
    \includegraphics[width=.15\columnwidth]{figures/imagerotations.png}
    \includegraphics[width=.4\columnwidth]{figures/paraxialgeometry.png}
\end{figure}



\subsection*{Key points}
\begin{itemize}[itemsep=0pt,topsep=0pt]
    \item $n$ tells us how much light slows down compared to the vacuum. Frecuency doesnt change but wavelength does.
    \item Fermat's principle states that the path is given by $\text{OPL}'(\text{path})=0$.
    \item Reflection is a refraction with negative index $n'=-n$.
    \item Sign convention is: up-right, counter clockwise, vertex-radius of curvature.
    \item Parity change is preserved only for an \textbf{even} number of reflections. It is determined by looking backwards to the object.
    \item Wherever we have a roof mirror, denoted by a V, we must account for two reflections.
    \item Reduced thickness is the air-equivalent distance of a medium. All objects are therefore reduced.
    \item In negative lenses, the rear $f_R'$ and front $f_F$ focal points are reversed from positive lenses.
    \item The $\text{FOV}=2\text{HFOV}$ has several definitions, but all are related each other: solid arc can be measured.
    \item Newtonian equations measure the object and image distances from the focal planes, while Gaussian equations from the principal planes.
    \item Nodal points $N$ and $N'$ preserves magnification of 1.
\end{itemize}