\section{Gaussian imagery}
Gaussian optics is a system of treating imaging as a mapping from object into image space. It is a special case of a 
\textbf{collinear transformation} applied to rotationally symmetric systems, and it maps points to points, lines to lines, and places to planes.
The corresponding object and image elements are called \bfemph{conjugate elements}. 

The cardinal points and planes completely describes the focal mapping. They are defined by specific magnifications:
\begin{align*}
    \begin{array}{lll}
        F&\text{Front focal point/plane}&m=\infty\\
        F'&\text{Rear focal point/plane}&m=0\\
        P&\text{Front principal plane}&m=1\\
        P'&\text{Rear principal plane}&m=1
    \end{array}\quad m=\frac{h'}{h}.
\end{align*}
\begin{figure}[h!]
    \centering
    \begin{subfigure}{.45\columnwidth}
        \centering
        \includegraphics[width=\columnwidth]{PartOne/ChapterOne/figures/rearcardinalpoint.png}
        \caption{Rear cardinal point/plane}
    \end{subfigure}
    \hfill
    \begin{subfigure}{.45\columnwidth}
        \centering
        \includegraphics[width=\columnwidth]{PartOne/ChapterOne/figures/frontcardinalpoint.png}
        \caption{Front cardinal point/plane}
    \end{subfigure}
\end{figure}

Also, there exists \bfemph{nodal points} $N$ and $N'$ that define the location of unit angular magnification for a focal system.
A ray passing through one is mapped to a ray passsing through the other having the same angle.
\begin{figure}[h!]
    \centering
    \includegraphics[width=.5\columnwidth]{PartOne/ChapterOne/figures/nodalpoints.png}
    \caption{}
\end{figure}
\begin{align}
    z_{PN}=z'_{PN}=f_F+f_R'=(n'-n)f_E,\quad m_N=-\frac{f_F}{f'_R}=\frac{n}{n'}.
\end{align}
\begin{itemize}[itemsep=0pt,topsep=0pt]
    \item Both nodal points of a single surface are located at the center of curvature of the surface: $z_{PN}=z'_{PN}=R$.
    \item If $n=n'$, then $z_{PN}=z'_{PN}=0$ and the nodal points are coincident with the respective principal planes.
    \item The angular substense of an image seen from $N'$ equals to the one seen from $N$: $m=h'/h=z'_N/z_N$.
\end{itemize}
%
\subsection{Representation of an optical system}
An optical system can be represented as a set of principal planes and a set of focal points.
\begin{figure}[h!]
    \centering
    \begin{subfigure}{.45\columnwidth}
        \centering
        \includegraphics[width=\columnwidth]{PartOne/ChapterOne/figures/problem1_a.png}
        \caption{Positive system}
    \end{subfigure}
    \hfill
    \begin{subfigure}{.45\columnwidth}
        \centering
        \includegraphics[width=\columnwidth]{PartOne/ChapterOne/figures/problem1_b.png}
        \caption{Negative system}
    \end{subfigure}
\end{figure}
    
\begin{emphasizer}
    Remember that refraction for $F$ must happen at $P$ while for $F'$ at $P'$.
\end{emphasizer}

%
\subsection{Newtonian equations}
\bfemph{Newtonian equations} measure object and image distances form the \textbf{focal planes}
\begin{figure}[h!]
    \centering
    \includegraphics[width=.6\columnwidth]{PartOne/ChapterOne/figures/newtonianequations.png}
    \caption{Newtonian equations.}
\end{figure}
\begin{table}[h!]
    \centering
    \begin{tabular}{l|l|l|l|l|l}
        $z=-\dfrac{f_F}{m}$&$z'=-mf_R'$&$zz'=f_Ff_R'$&$\dfrac{z}{n}=\dfrac{f_E}{m}$&$\dfrac{z'}{n'}=-mf_E$&$\dfrac{z}{n}\dfrac{z'}{n'}=-f_E^2$
    \end{tabular}
\end{table}
%
\subsection{Gaussian equations}
\bfemph{Gaussian equations} measure object and image distances 
from the \textbf{principal planes}.
\begin{figure}[h!]
    \centering
    \includegraphics[width=.6\columnwidth]{PartOne/ChapterOne/figures/gaussianequations.png}
    \caption{Gaussian equations.}
\end{figure}
\begin{table}[h!]
    \centering
    \begin{tabular}{l|l|l|l|l}
        $z=-\dfrac{(1-m)}{m}f_F$&$z'=(1-m)f_R'$&$m=-\dfrac{z'}{z}\dfrac{f_F}{f_R'}$&$\dfrac{f_R'}{z'}+\dfrac{f_F}{z}=1$&$\dfrac{z}{n}=\dfrac{(1-m)}{m}f_E$\\
        $\dfrac{z'}{n'}=(1-m)f_E$&$m=\dfrac{z'/n'}{z/n}$&$\dfrac{n'}{z'}=\dfrac{n}{z}+\dfrac{1}{f_E}$&&
    \end{tabular}
\end{table}

A ray angle multiplied by the refractive indesx of its optical space is called \bfemph{optical angle}:
\begin{align}
    \text{Optical angle}\qquad\highlight{\omega=nu\quad(-)}.
\end{align}

\subsection{Longitudinal magnification}
The \bfemph{longitudinal magnification} relates the distances between pairs of conjugate planes.
\begin{figure}[h!]
    \centering
    \includegraphics[width=.6\columnwidth]{PartOne/ChapterOne/figures/longitudinalmagnification.png}
    \caption{Longitudinal magnification allows you to have the thickness of the object or image.}
\end{figure}

\begin{align}
    \Delta z=z_2-z_1,\quad\Delta z'=z_2'-z_1',\quad m_1=\frac{h_1'}{h_1},\quad m_2=\frac{h_2'}{h_2},\quad \frac{\Delta z'}{\Delta z}=-\frac{f_R'}{f_F}m_1m_2,\quad \frac{\Delta z'/n'}{\Delta z/n}=m_1m_2
\end{align}
As the plane separation approaches zero, $m_1\approx m_2\approx m$ and the local magnification $\bar{m}$ is obtained:
\begin{align}
    \bar{m}=\frac{n'}{n}m^2,\quad \frac{\Delta z'/n'}{\Delta z/n}=m^2.
\end{align}


%
\subsection{Gaussian properties of a single refracting surface}
The radius of curvature $R$ is defined to be the distance from its vertex to the center of curvature CC.
The front and rear principal planes are coincident and located at the surface vertex. In addition, both nodal points are located at the center 
of the curvature (CC) of the optical surface.
\begin{figure}[h!]
    \centering
    \begin{subfigure}{.5\columnwidth}
        \centering
        \includegraphics[width=\columnwidth]{PartOne/ChapterOne/figures/singlesurface.png}
        \caption{Single surface}
    \end{subfigure}
    \hfill
    \begin{subfigure}{.45\columnwidth}
        \centering
        \includegraphics[width=\columnwidth]{PartOne/ChapterOne/figures/thinlensreduction}
        \caption{Thin lens reduction}
    \end{subfigure}
\end{figure}

\begin{emphasizer}
    The use of reduced distances and optical angles allows a system to be represented as an air-equivalent system with thin lenses of the same power $\phi$.   
\end{emphasizer}

\begin{example}{Gaussian imagery of a single refractive surface}
    \begin{enumerate}[itemsep=0pt,topsep=0pt,label=\alph*)]
        \item For a single refracting surface, we have that:
        \begin{itemize}
            \item Both nodal points are located at the center of curvature CC.
            \item Front and real principal planes are located at the vortex.
            \item The reduced thickness of the surface is the focal length of its thin lens representation.
        \end{itemize}
        We illustrate these quantities along with the vertex and the focal lengths in the following figure.
        \begin{figure}[h!]
            \centering
            \includegraphics[width=.5\columnwidth]{PartOne/ChapterOne/figures/problem2_a.png}
            \caption{Illustration of cardinal point for a single refractive surface.}
        \end{figure}
        We illustrate also some quantities of this surface:
        \begin{align*}
            &C=\frac{1}{R}=100\;m^{-1},\quad \phi=(n'-n)C=33.3\;m^{-1},\quad f_E=\frac{1}{\phi}=30\;mm,\\
            &f_F=-nf_E=-30\;mm,\quad f'_R=n'f_E=40\;mm.
        \end{align*}
        \item We use the following equation:
        \begin{align*}
            \frac{n'}{z'}=\frac{n}{z}+\frac{1}{f_E}\longrightarrow z'=\frac{n'zf_E}{nf_E+z}.
        \end{align*}
        Replacing the physical values and the EFL:
        \begin{align*}
            z'=\frac{(1.333)(30)(-100)}{(1)(30)-100}=+57.129\;mm.
        \end{align*}
        Its height is determined by the magnification:
        \begin{align*}
            m=\frac{h'}{h}=\frac{z'/n'}{z/n}=\frac{57.129/1.333}{-100/1}=-0.429\longrightarrow h'=mh=(-0.429)(10\;mm)=-4.29\;mm.
        \end{align*}
    \end{enumerate}
\end{example}
%%
\subsection{Generalized afocal systems}
An afocal system is formed by the combination of two focal systems. The rear focal point of the first one is coincident with the front 
focal points of the second system. Common afocal systems are telescopes, binoculars, and beam expanders.
\begin{figure}[h!]
    \centering
    \includegraphics[width=.6\columnwidth]{PartOne/ChapterOne/figures/afocalsystems.png}
    \caption{Generalized afocal system.}
\end{figure}
The transverse and longitudinal magnification are constant. Due to this, the cardinal points are not defined, and the Gaussian and 
Newton equations \textbf{cannot} be used to determine cnojugate planes. However, any pair of conjugate planes coupled with $\bar{m}$ 
can be employed.
\begin{align}
    m=\frac{f_{F2}}{f_{R1}'}=-\frac{f_{E2}}{f_{E1}}=-\frac{f_2}{f_1},\quad \bar{m}=\frac{n'}{n}m^2,\quad \frac{\Delta z'/n'}{\Delta z/n}=m^2.
\end{align}





