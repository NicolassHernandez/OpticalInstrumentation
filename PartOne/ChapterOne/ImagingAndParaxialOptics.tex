\section{Imaging and paraxial optics}
An optical system is a collection of optical elements. The first-order properties of the system is charectized
by a \textbf{single} focal length, or magnification.
First-order optics is the optics of perfect imaging systems: no aberrations, where the object is \textbf{mapped} to its image. 

A small number of system properties can completely define and determine the mapping of first-order imaging properties. These are 
known as the \bfemph{cardinal points} of the imaging system.


Each time a rafecting or reflecting surface is encountered, a new optical space is entered. In general, if a system coantains $N$ surfaces, 
there will be $N+1$ optical spaces. The first space is calles the \bfemph{object space} and the last \bfemph{image space}.
%%
\subsection{General system}
A black box is a convenient way to treat the optical system and analyze the refractions. 
\begin{itemize}[itemsep=0pt,topsep=0pt]
    \item An infinite object from left is effectively refracted by the system at the \bfemph{rear principal plane} $P'$.
    The distance from $P'$ to the \bfemph{rear focal point} $F'$ is the \bfemph{rear focal length} $f_R'$.
    \item An object starting at $F$ is effectively refracted by the system at the \bfemph{front principal plane} $P$.
    The distance from $P$ to the \bfemph{front focal point} $F$ is the \bfemph{front focal length} $f_F$.
\end{itemize}
\begin{figure}[h!]
    \centering
    \begin{subfigure}{.45\columnwidth}
        \centering
        \includegraphics[width=\columnwidth]{PartOne/ChapterOne/figures/realprincipalplane.png}
        \caption{Refraction at $P'$ ($f_R'>0$)}
    \end{subfigure}
    \hfill
    \begin{subfigure}{.45\columnwidth}
        \centering
        \includegraphics[width=\columnwidth]{PartOne/ChapterOne/figures/frontprincipalplane.png}
        \caption{Refraction at $P$ ($f_F<0$)}
    \end{subfigure}
\end{figure}
The system can now be trated as a thin lens, with the difference that object and image distances 
are from their respective principal planes.

%
\subsection{Paraxial optics}
The first-order properties of the system can be found using \bfemph{paraxial rays}.
\begin{definition}[Paraxial ray]
    \begin{enumerate}[itemsep=0pt,topsep=0pt,label=\alph*)]
        \item Rays are nearly parallel to the optical axis.
        \item The amount a ray is bent at surfaces is assumed to be small: $\cos u\approx\cos u'$.
        \item The sag of surfaces is ignored: $|\text{sag}|\ll|R|,|z|,|z'|$. Rays refracts at the vertex.
        \item Rays are traced using the slopes of the rays instead of ray angles: $u=y/-z$.
    \end{enumerate}
\end{definition}
\begin{figure}[h!]
    \centering
    \begin{subfigure}{.3\columnwidth}
        \centering
        \includegraphics[width=\columnwidth]{PartOne/ChapterOne/figures/sag.png}
        \caption{Sag is ignored}
    \end{subfigure}
    \hfill
    \begin{subfigure}{.3\columnwidth}
        \centering
        \includegraphics[width=\columnwidth]{PartOne/ChapterOne/figures/sloperay.png}
        \caption{Ray slopes instead of ray angles}
    \end{subfigure}
    \hfill
    \begin{subfigure}{.3\columnwidth}
        \centering
        \includegraphics[width=.8\columnwidth]{PartOne/ChapterOne/figures/singlerefractingsurface.png}
        \caption{Ray slopes instead of ray angles}
    \end{subfigure}
\end{figure}
%
\subsubsection{Single refractive surface}
Single refracting surface are the fundamental object from which all other are composed of. They are defined by two 
refractive indices at the object and image space $n$ and $n'$, respectively, and the curvature of the surface:
\begin{align}
    \text{Radius of curvature}\qquad\highlight{R=\frac{1}{C}}.
\end{align}
\begin{figure}[h!]
    \centering
    \includegraphics[width=.8\columnwidth]{PartOne/ChapterOne/figures/singlerefractingsurface.png}
    \caption{SIngle refracting surface.}
\end{figure}
The \bfemph{optical power} $\phi$ is a measure of the benting power of the surface:
\begin{align}
    \text{Optical power}\qquad\highlight{\phi=(n'-n)C\quad(m^{-1})(\text{diopters})}.
\end{align}

Then, the \bfemph{paraxial raytrace equation} describes how the ray will travel after hitting the refracting surface:
\begin{align}
    \text{Paraxial raytrace equation}\qquad\highlight{n'u'=nu-y\phi}.
\end{align}

There are others useful equations, illustrated as follows 
\begin{align}
    \frac{n'}{z'}=\frac{n}{z}+\phi,\quad f_E=\frac{1}{\phi},\quad f_F=-nf_E,\quad f_R'=n'f_E.
\end{align}
\begin{emphasizer}
    A reflective surface is a special case with $n'=-n$.
\end{emphasizer}

With all variables defined, the scheme is illustrated as follows:
\begin{figure}[h!]
    \centering
    \includegraphics[width=.8\columnwidth]{PartOne/ChapterOne/figures/generalparaxialsystem.png}
    \caption{General paraxial system, with parameters defined previously.}
\end{figure}

\begin{emphasizer}
    The focal length $f_E$ is not a physical distance, but the front and real focal lengths are physical distances.
\end{emphasizer}
