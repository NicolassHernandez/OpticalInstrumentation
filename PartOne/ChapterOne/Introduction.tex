\section{Introduction}
%
\subsection{Light propagation}
Geometrical optics is the study of light in the limit of short wavelenegths. We treat light as propagating rays.
Geometrical optics usually ignores interferences, diffraction, polarization and quantum effects.

It often includes:
\begin{itemize}[itemsep=0pt,topsep=0pt]
    \item Reflection, refraction
    \item Optical design
    \item Imaging properties 
    \item Aberrations
    \item Radiometry
\end{itemize}

Light is a self-propagating EM wave where electric and magnetic fields are perpendicular or transverse to direction of propagation.
In a vacuum, light propagates at the speed of light $c$, which is 
\begin{align}
    c=\frac{1}{\sqrt{\mu_0\epsilon_0}}=2.99792458\times10^8\quad(m/s).
\end{align}
The \bfemph{wavelength} $\lambda$ is the distance between two peaks or two valleys on the wave.

A \bfemph{wavefront} is a surface of constant propagtion time from the source. It begins from a point source in spherical form, 
and as it propagates away, a given solid arc tend to behaves a a planar wavefront. 
\begin{figure}[h!]
    \centering
    \includegraphics[width=.7\columnwidth]{PartOne/ChapterOne/figures/wavefront.png}
    \caption{We can treat the wavefront as planar when assuming a distant object.}
\end{figure}

The time for one wavelength to pass is knwon as the \bfemph{period} $T$:
\begin{align}
    T=\frac{\lambda}{V}\quad(s),
\end{align}
where $V$ is the velocity of propagation. The number of wavelengths to pass in one second is the \bfemph{frequency} $\nu$:
\begin{align}
    \nu=\frac{1}{T}\quad(s^{-1})(Hz).
\end{align}

\subsection{Sign convention}
We define the sign convention for which the light propagates. It allows us to keep track of physical quantities and multiple reflections when analyzing 
complex optical systems.
\begin{figure}[h!]
    \centering
    \includegraphics[width=\columnwidth]{PartOne/ChapterOne/figures/signconvention.png}
    \caption{}
\end{figure}

%
\subsection{Electromagnetic spectrum}
The light can be of various wavelengths (frequencies) which translates to the color of the light. The range of 
the wavelengths is called the \bfemph{electromagnetic spectrum}.
\begin{figure}[h!]
    \centering
    \includegraphics[width=.7\columnwidth]{PartOne/ChapterOne/figures/electromagneticspectrum.jpg}
    \caption{}
\end{figure}
The \bfemph{index of refraction} tells how much the light is slown down in a medium with respecto to vacuum.
\begin{align}
    \text{Index of refraction}\qquad\highlight{n=\frac{\text{Speed of light in vacuum}}{\text{Speed of light in medium}}=\frac{c}{V}\geq1}.
\end{align}
From one medium to another, the frequency remains unchanged but only the wavelength is modified. The index of refraction is 
a function of the wavelength and of the temperature.

\begin{emphasizer}[Vaccum equal to air]
    In geometrical optics, vacuum and air are used interchangable as the index of refraction of air is $n\approx1$.
\end{emphasizer}
%
\subsection{Optical path length}
The \bfemph{optical path length} (OPL) is the equivalent distance in vacuum that light would cover in the same time as it takes to cross the actual medium.
\begin{align}
    \text{Optical path length}\qquad\highlight{\text{OPL}=\int_a^b\bm{n}(s)\cdot d\bm{s}}.
\end{align}
When the medium is homogeneous, the index $n$ reduces to a constant value. Consequently, the ray travels in \textbf{straight lines}. 

\bfemph{Fermat's principle} states that the path taken by the light from one point to another is the path for which the OPL is stationary:
\begin{align}
    \text{Fermat's principle}\qquad\highlight{\frac{d\text{OPL}}{d\text{path}}=0}.
\end{align}

\subsection{Snell's laws of reflection and refraction}
Snell's law can be obtained from Fermat's postulate. They governs the dynamics of the ray when passing thourgh an interface of different index of 
refractions:
\begin{align}
    \text{Snell's laws}\qquad\highlight{\begin{array}{rl}
        n_1\sin\theta_1&=n_2\sin\theta_2\\
        \theta_1&=-\theta_2
    \end{array}}
\end{align}
The angles are measured relative to the surface normal.
\begin{figure}[h!]
    \centering
    \begin{subfigure}{.45\columnwidth}
        \centering
        \includegraphics[width=.6\columnwidth]{PartOne/ChapterOne/figures/snellrefraction.png}
        \caption{Refraction}
    \end{subfigure}
    \hfill
    \begin{subfigure}{.45\columnwidth}
        \centering
        \includegraphics[width=.6\columnwidth]{PartOne/ChapterOne/figures/snellreflection.png}
        \caption{Reflection}
    \end{subfigure}
    \caption{In refraction and reflection, the angles are taken respective to the surface normal.}
\end{figure}
\begin{emphasizer}
    The reflection is equal to refraction with a negative index: $n=-n$.
\end{emphasizer}

%
\subsection{Total internal reflection (TIR)}
\bfemph{Total internal reflection} occurs when the light propagating from a medium $n_1$ to another $n_2$, with $n1>n_2$, exceed a critical incident angle 
\begin{align}
    \text{Total internal reflection}\qquad\highlight{\theta_i>\theta_c=\sin^{-1}\frac{n_2}{n_1}}.
\end{align} 
Under this condition, 100\% of the light is reflected into $n_1$, and no refracted light is present.
