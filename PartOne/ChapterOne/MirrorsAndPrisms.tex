\section{Mirrors and prisms}

\subsection{Parallel mirrors}
Plane mirrors are used to:
\begin{itemize}[itemsep=0pt,topsep=0pt]
    \item Produce a deviation
    \item Fold the optical path
    \item Change image parity
\end{itemize}

\subsection{Parity change}
A reflection from the plane mirror will cause a parity change in the image.
\begin{figure}[h!]
    \centering
    \begin{subfigure}{.45\columnwidth}
        \centering
        \includegraphics[width=\columnwidth]{PartOne/ChapterOne/figures/planemirrorparitychange.png}
        \caption{}
    \end{subfigure}
    \hfill
    \begin{subfigure}{.45\columnwidth}
        \centering
        \includegraphics[width=.6\columnwidth]{PartOne/ChapterOne/figures/paritychange.png}
        \caption{}        
    \end{subfigure}
\end{figure}
An inversion (reversion) is a parity change about the horizontal (vertical) line, whereas a $180^\circ$ rotation has no parity change and is rotated about the optical axis.
An inversion and a reversion is equivalent to a $180^\circ$ rotation.
\begin{emphasizer}[Parity change]
    Only an \textbf{odd} number of reflections changes parity.
\end{emphasizer}
Parity is determined by looking back agains the propagation towards the obejct. Compare looking directly the objet vs at the reflection.

\begin{emphasizeb}
    A lens adds inversion and reversion to the object, so that the image has no parity change, only rotation.
\end{emphasizeb}

%%
\subsection{Non-parallel plane mirror}
The \bfemph{dihedral line} is the line of intersection of two non-parallel plane mirrors. The ray is deviated twice the angle between the mirrors.
\begin{align}
    \gamma=2\alpha=\begin{cases}
        \text{Input-Output rays cross},&\alpha<90^\circ\\
        \text{Input-Output rays diverge},&\alpha>90^\circ\\
        \text{Input-Output rays anti-parallel},&\alpha=90^\circ
    \end{cases}
\end{align}
\begin{figure}[h!]
    \centering
    \includegraphics[width=.4\columnwidth]{PartOne/ChapterOne/figures/nonparallelmirrors.png}
    \caption{}
\end{figure}

There are several mirrors,
\begin{itemize}[itemsep=0pt,topsep=0pt]
    \item\textbf{Roof mirror} Two plane mirrors with a dihedral angle of $90^\circ$. It is used to insert two reflection in the propagation. The presence of this mirror
    is indicated by a "V".
\end{itemize}

\subsection{Prisms and tunnel diagrams}
\bfemph{Prisms} can be considered systems of plane mirrors. The reflection may be due to TIR, or by reflective coating.

A \bfemph{tunnel diagram} \textbf{unfolds} the optical path at each reflection so that the ray is maintained straight through the propagation in the prism.
\begin{figure}[h!]
    \centering
    \includegraphics[width=.2\columnwidth]{PartOne/ChapterOne/figures/RightAnglePrism.png}
    \includegraphics[width=.3\columnwidth]{PartOne/ChapterOne/figures/PentaPrism.png}
    \includegraphics[width=.25\columnwidth]{PartOne/ChapterOne/figures/DovePrism.png}
    \caption{}
\end{figure}


\subsection{Reduced thickness}
The \bfemph{reduced thickness} is the vacuum equivalent distance of the medium that has the same propagation effect.
\begin{align}
    \text{Reduced thickness}\qquad\highlight{\tau=\frac{t}{n}}.
\end{align}
Expressing all distances in $\tau$ is equivalent to propagates the light in only vacuum (or air).
This quantity is implicitly in the optical propagation and will be present in equations. When a reflection takes place,
both $n$ and $t$ are negative, but $\tau$ remains positive. 
\begin{figure}[h!]
    \centering
    \includegraphics[width=.35\columnwidth]{PartOne/ChapterOne/figures/reducedthickness_a.png}
    \includegraphics[width=.35\columnwidth]{PartOne/ChapterOne/figures/reducedthickness_b.png}
    \caption{Reduced thickness is the vacuum (air) equivalent distance.}
\end{figure}


Tunnel diagrams are afected by the reduced thickness along the 
propagation distance. If the total distance is $L$, then the reduced is $L/n$. 

\begin{figure}[h!]
    \centering
    \includegraphics[width=.4\columnwidth]{PartOne/ChapterOne/figures/rtunneldiagram_a.png}
    \includegraphics[width=.35\columnwidth]{PartOne/ChapterOne/figures/rtunneldiagram_b.png}
    \caption{The diagram is only shortened along the direction of the propagation.}
\end{figure}

In a plate parallel plate (PPP), the beam is shifted horizontally a distance proportional to $\tau$ when is placed perpendicular ot the optical axis:
\begin{align}
    d=\frac{n-1}{n}t.
\end{align}
If it is disposed with an angle $\theta$, then the ray will be shifted vertically
\begin{align}
    D\approx -t\theta\frac{n-1}{n}.
\end{align}

\begin{example}{Reduced thickness and aparent distance}
\begin{enumerate}[itemsep=0pt,topsep=0pt,label=\alph*)]
    \item The fish is $500\;mm$ beneath the surface of the water ($n=1.33$). For the cat observing in air, how far below the water's surface does
    the fish appear to be?\\
    In this case, we have
    \begin{align*}
        d_{\text{total}}=\frac{500\;mm}{1.33}=375.94\;mm.
    \end{align*}
    The fish appears to be $377\;mm$ below the surface of the water.
    \item The cat is $500\;mm$ above the surface of the water. For the fish observing in water, how far abothe the water's surface
    does the cat appear to be?\\
    The total distance is the sum of the air thickness in terms of the water and the thickness of the water:
    \begin{align*}
        d_{\text{total}}=1.33\cdot500\;mm+500\;mm=665\;mm+500\;mm=1165\;mm.
    \end{align*}
    The cat appears to be $665\;mm$ above the surface of the water.
    \item Several months later the cat return to watch the fish again. This time, there is a $100\;mm$ thick layer of ice ($n=1.31$) on the surface.
    The fish is still a total physical distance of $500\;mm$ below the surface. Repeat parts a) and b).\\
    In this case, we assume that the thick layer of ice has \textbf{replaced} $100\;m$ of the water while the distance of air remains the same.
    \begin{itemize}[itemsep=0pt,topsep=0pt]
        \item For the part a), the distance would be:
        \begin{align*}
        d_{\text{total}}=(\frac{100\;mm}{1.31}+\frac{400\;mm}{1.33})+500\;mm=377\;mm+500\;mm=877\;mm.
        \end{align*}
        The fish appears to be $377\;mm$ below the surface of the ice.
        \item For part b), the total equivalent distance is the distance of the water, plus the equivalent distance in water of the ice and air: 
        \begin{align*}
        d_{\text{total}}=1.33\cdot(\frac{100\;mm}{1.31}+500\;mm)+400\;mm=767\;mm+400\;mm=1166.53\;mm.
        \end{align*}
        The cat appears to be $767\;mm$ above the water, that is, below the air and the ice.
        We computed first the reduced thickness of ice in order to then convert it to the equivalent in water.
    \end{itemize}
\end{enumerate}
\end{example}

