\section{Paraxial raytrace}
%
\subsection{Introduction}
\bfemph{Paraxial optics} is a method of determining the first-order properties of an optical system that assumess all ray angles are small.
It follows the same assumptions of \bfemph{paraxial optics} regime seen.

\begin{figure}[h!]
    \centering
    \includegraphics[width=.5\columnwidth]{PartOne/ChapterOne/figures/paraxialopticsscheme.png}
    \caption{A paraxial raytrace is linear with respect to ray angle and heights.}
\end{figure}

It composes of iterative \bfemph{refraction} and \bfemph{Transfer} processes. These type of raytrace are called \bfemph{YNU raytrace}:
\begin{align}
    \text{Object $\longrightarrow$ Image}&\quad\left\{
    \begin{array}{rl}
    \text{Refraction (reflection)}&\qquad\highlight{n'u'=nu-y\phi\qquad \omega'=\omega-y\phi}\\
    \text{Transfer}&\qquad\highlight{y'=y+u't'\qquad y'=y+\omega'\tau'}    
    \end{array}\right.\\
    \text{Image $\longrightarrow$ Object}&\quad\left\{
    \begin{array}{rl}
    \text{Refraction (reflection)}&\qquad\highlight{nu=n'u'+y\phi\qquad \omega=\omega'+y\phi}\\
    \text{Transfer}&\qquad\highlight{y=y'-u't'\qquad y=y'-\omega'\tau'}    
    \end{array}\right.
\end{align}
%
\subsection{Procedure}
The procedure is always the same:
\begin{itemize}[itemsep=0pt,topsep=0pt]
    \item You set the optical propoerties of the system.
    \item\textbf{Rear cardinal points} Trace a forward ray from object to image, and at the image space, you look for $t$ that satisfies $y=0$ to get the BFD.
    \begin{align}
        \phi=-\frac{\omega'_k}{y_1},\quad f_E=\frac{1}{\phi},\quad f_R'=n'f_E,\quad \text{BFD}=-\frac{y_k}{u'_k},\quad d'=\text{BFD}-f_R'.
    \end{align}
    \item\textbf{Front cardinal points} Trace a backward ray from image to object, and at object space, you look for $t$ that satisfies $y=0$ to het the FFD.
    \begin{align}
        \phi=\frac{\omega_1}{y_k},\quad f_E=\frac{1}{\phi},\quad f_F=-nf_E,\quad \text{FFD}=-\frac{y_1}{u_1},\quad d=\text{FFD}-f_F.
    \end{align}
\end{itemize}

\begin{figure}[h!]
    \centering
    \begin{subfigure}{.45\columnwidth}
        \centering
        \includegraphics[width=\columnwidth]{PartOne/ChapterOne/figures/rearcardinalpoints_raytrace.png}
        \caption{Finding rear cardinal points}
    \end{subfigure}
    \hfill
    \begin{subfigure}{.45\columnwidth}
        \centering
        \includegraphics[width=\columnwidth]{PartOne/ChapterOne/figures/frontcardinalpoints_raytrace.png}
        \caption{Finding front cardinal points}
    \end{subfigure}
\end{figure}

Once you have the rays traced, you construct a reduced table, where 
\begin{itemize}[itemsep=0pt,topsep=0pt]
    \item List out all the surfaces: the first part of the big table. Parameters vertically in-line with the surface are associated with optical surfaces. 
    Parameters sandwiched between refer to the optical spaces.
    \item Another box below contains the information about the rays traced to find the cardinal points.
\end{itemize}

\begin{example}{Cassegrain telescope}
    \begin{figure}[h!]
        \centering
        \includegraphics[width=.49\columnwidth]{PartOne/ChapterOne/figures/cassegraintelescope.png}
        \includegraphics[width=.49\columnwidth]{PartOne/ChapterOne/figures/cassegraintelescope_b.png}
    \end{figure}
\end{example}
\begin{example}{Thin lens}
    \begin{figure}[h!]
        \centering
        \includegraphics[width=.8\columnwidth]{PartOne/ChapterOne/figures/thinlens_raytrace.png}
    \end{figure}    
\end{example}
\begin{example}{}
    In this case we have three surface, each with their correspond surface curvature $C$ and index of refraction $n$.
\begin{itemize}[itemsep=0pt,topsep=0pt]
  \item\textbf{Gaussian reduction} 
  The optical power of each surface is:
  \begin{align*}
    \phi_1&=\frac{n_1-n_0}{R_1}=\frac{1.336-1}{7.8\;mm}=0.043\;mm^{-1},\\
    \phi_2&=\frac{n_2-n_1}{R_2}=\frac{1.413-1.336}{10\;mm}=0.008\;mm^{-1},\\
    \phi_3&=\frac{n_3-n_2}{R_3}=\frac{1.336-1.413}{-6\;mm}=0.013\;mm^{-1}.
  \end{align*}
  Now, we combine surface 1 with 2:
  \begin{align*}
    \phi_{12}=\phi_1+\phi_2-\phi_1\phi_2\tau_1=0.043+0.008-0.043\cdot0.008\cdot\frac{3.6}{1.336}=0.050\;mm^{-1}.
  \end{align*}
  The shift of the principal plane are given by
  \begin{align*}
    \delta_{12}&=\frac{\phi_2}{\phi_{12}}\tau_1=\frac{0.008}{0.050}\cdot\frac{3.6}{1.336}=0.431\;mm\longrightarrow d_{12}=\delta_{12}.\\
    \delta_{12}'&=-\frac{\phi_1}{\phi_{12}}\tau_1=-\frac{0.043}{0.050}\cdot\frac{3.6}{1.336}=-2.317\;mm\longrightarrow d_{12}'=n_2\delta_{12}'=-3.274\;mm.
  \end{align*}
  We can see that the front principal plane is displaced from $V_1$ to the left, while the rear principal plane is shifted to the right of $V_2$.
  In addition, the distance $d_{12}'$ considered the index $n_2$ as it belong to that space.
  The distance of propagation through the index $n_2$ must be adjusted due to the shift of the rear principal plane:
  \begin{align*}
    \tau_{12}=\frac{t_2-d_{12}'}{n_3}=\tau_2-\delta_{12}'=\frac{3.6}{1.413}+2.317=4.865\;mm.
  \end{align*}
  Now, we compute the total optical power considering the reduction and the third surface:
  \begin{align*}
    \phi=\phi_{12}+\phi_3-\phi_{12}\phi_3\tau_{12}=0.050+0.013-(0.046)(0.013)(4.865)=0.060\;mm^{-1}.
  \end{align*}
  The shifts are:
  \begin{align*}
    d_{123}&=n_0\delta_{123}=\frac{\phi_3}{\phi}\tau_{12}=\frac{0.013}{0.060}\cdot4.865=1.054\;mm\\
    d_{123}'&=n_3\delta_{123}'=-n_3\frac{\phi_{12}}{\phi}\tau_{12}=-(1.336)\frac{0.050}{0.060}\cdot4.865=-5.416\;mm.
  \end{align*}
  The total shift from the first surface is the sum of individual fron shift computed, while for the last surface is just the shift computed in the last reduction:
  \begin{align*}
    d&=d_{12}+d_{123}=0.431+1.054=1.485\;mm\\
    d'&=d_{123}'=-5.416\;mm.
  \end{align*}
  The front (rear) focal lengths are then
  \begin{align*}
    f_E=\frac{1}{\phi}=16.667\;mm\longrightarrow \begin{array}{l}
      f_F=-n_0f_E=-(1)(16.667)=-16.667\;mm\\
      f_R'=n_3f_E=(1.336)(16.667)=22.267\;mm.
    \end{array}
  \end{align*}
  Finally, the FFD and BFD are:
  \begin{align*}
    \text{FFD}&=f_F+d_{123}=-16.667+1.054=15.613\;mm\\
    \text{BFD}&=f_R'+d_{123}'=22.267-5.416=16.851\;mm.
  \end{align*}
  The reduction process is shown in figure \ref{fig:problem4_a}. The green quantities are the equivalent of the final reduction.
  \begin{figure}[h!]
    \centering
    \includegraphics[width=.5\columnwidth]{PartOne/ChapterOne/figures/problem4_a.pdf}
    \caption{Gaussian reduction for the three-surfaces object.}
    \label{fig:problem4_a}
  \end{figure}
  \item\textbf{Ray tracing}
  For the ray tracing, we will fill the ynu spreadsheet. We will trace two rays, one from left to right and other in opposite direction in order to find 
  the front and real focal lengths.
  \begin{table}[h!]
    \centering
    \resizebox{.8\columnwidth}{!}{
    \begin{tabular}{l|l|l|l|l|l|l|l|l|l}
      &\shortstack{Object\\space}&Space 1&Surface 1&Space 2&Surface 2&Space 3&Surface 3&Space 4&\shortstack{Image\\space}\\
      \hline
      $C$&&&0.128&&0.1&&-0.167&&\\
      $t$&&\textcolor{blue}{15.167}&&3.6&&3.6&&\textcolor{red}{16.856}&\\
      $n$&&1&&1.336&&1.413&&1.336&\\
      \hline 
      $-\phi$&&&-0.043&&-0.008&&-0.013&&\\
      $t/n$&&\textcolor{blue}{15.167}&&2.695&&2.548&&\textcolor{red}{12.617}&\\
      \hline 
      $y$&1&1&1&&0.884&&0.757&&0\\
      $nu$&0&0&&-0.043&&-0.05&&-0.060&\\
      $u$&0&0&&&&&&-0.045&\\
      \hline
      $y$&0&&0.910&&0.967&&1&1&1\\
      $nu$&&0.060&&0.021&&0.013&&0&0\\
      $u$&&0.060&&&&&&0&0
    \end{tabular}}
  \end{table}
  \end{itemize}
  We must compare the $t$ in blue with the FFD and the red $t$ with the BFD. The differences are due to the approximation in intermediate computations. 
  We can see that both methods yield the same answer, despite that ynu raytracing is way faster than Gaussian reduction.

  The effective focal length is defined considering the magnification $nu$ divided by the input ray:
  \begin{align*}
    f_E'=\frac{1}{\phi'}=-\frac{y_1}{nu'}=\frac{1}{0.060}=16.667\;mm\longrightarrow f_R'=n_3f_E'=22.267\;mm.
  \end{align*}
  Similarly,
  \begin{align*}
    f_E'=\frac{1}{\phi}=\frac{y_2}{nu}=\frac{1}{0.060}=16.667\;mm\longrightarrow f_F=-n_0f_E=-16.667\;mm.
  \end{align*}
  The focal lengths match exactly as the ones computed by Gaussian reduction. We can also compute the principal planes shifts, but we will not do it as we already know the answer.
\end{example}
%
\newpage
\subsection{Table worksheet}
\begin{figure}[h!]
    \centering
    \includegraphics[width=.6\columnwidth]{PartOne/ChapterOne/figures/raytraceworksheet.png}
\end{figure}



