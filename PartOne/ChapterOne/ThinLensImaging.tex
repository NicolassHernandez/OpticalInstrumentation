\section{Thin lens Imaging}

\subsection{Introduction}
A \bfemph{thin lens} is an idealization of an optical system with:
\begin{itemize}[itemsep=0pt,topsep=0pt]
    \item Zero thickness $\tau$.
    \item Refracting power $\phi$.
    \item Characterized by its focal length $f$.
\end{itemize}

An object at infiniity is imaged to the \bfemph{rear focal point} $F'$, whereas an object ar the \bfemph{front focal point} $F$ is 
iamged to infinity. The respective distances from the center of the thin lens to $F$ is $f_F$ and to $F'$ is $f_R'$. 
\begin{figure}
    \centering
    \begin{subfigure}{.45\columnwidth}
        \centering
        \includegraphics[width=\columnwidth]{PartOne/ChapterOne/figures/frthinlens.png}
        \includegraphics[width=.9\columnwidth]{PartOne/ChapterOne/figures/frnegativethinlens.png}
        \caption{Rear focal length $f_R'$}
    \end{subfigure}
    \hfill
    \begin{subfigure}{.45\columnwidth}
        \centering
        \includegraphics[width=\columnwidth]{PartOne/ChapterOne/figures/ffthinlens.png}
        \includegraphics[width=.9\columnwidth]{PartOne/ChapterOne/figures/ffnegativethinlens.png}
        \caption{Front focal length $f_F$}
    \end{subfigure}
\end{figure}
\begin{emphasizer}[Positive vs negative thin lenses]
    A positive lens has a positive focal length $f>0$, a positive $f_R'>0$ but a negative $f_F<0$. In the negative lens,
    \textbf{all} is opposite. 
\end{emphasizer}

\begin{emphasizer}
    These points and lenghts are purely geometric properties of the lens.
\end{emphasizer}

\bfemph{Real rays} are rays that are physically present, they can be touched. On the other hand, \bfemph{virtual rays} are 
rays that are projection of real rays, and cannot be touched. Both type of rays are useful for imaging.

%
\subsection{Imaging relationships}
The imaging property of a thin lens relates the position of the object with that of the image.
\begin{figure}
    \centering
    \includegraphics[width=.7\columnwidth]{PartOne/ChapterOne/figures/imagingscheme.png}
    \caption{Imaging scheme}
\end{figure}

The \bfemph{thin lens equation} is 
\begin{align}
    \text{Thin lens equation}\qquad\highlight{\frac{1}{z'}=\frac{1}{z}+\frac{1}{f}}.
\end{align}
The \bfemph{transverse magnification} is the ratio of the heights:
\begin{align}
    \text{Transverse magnitication}\qquad\highlight{m=\frac{h'}{h}=\frac{z'}{z}}.
\end{align}
These two equations are the most fundamental for imaging. They are used extensivelty through geometrical optics.

Intersecting at elast 2 rays is enough to map a point from object to image. The following are the trivial rays used:
\begin{itemize}
    \item Parallel ray from the object, emerges (diverges) through the rear focal point.
    \item Ray from object throught the front focal point, emerging parallel (antiparallel).
    \item A ray that goes directly from the object through the center of the lens which is not refracted.
\end{itemize}

\begin{example}{Imaging with a negative lens}
The ray diagram is illustrated in figure \ref{fig:1a}. We have traced three rays:
    \begin{itemize}[itemsep=0pt,topsep=0pt]
      \item Parallel to the optical axis from the object, then it is refracted with direction to $F'$.
      \item Direct to $F$: it is refracted so that it becomes parallel to the optical axis.
      \item The chief ray, which maintain its direction through its propagation.
    \end{itemize}
    The intersection of these three rays produces the image. We can see that the image is to the left of the lens, but to the right of the object.
    Therefore, it will be a virtual image and demagnified.
    \begin{figure}[h!]
      \centering
      \includegraphics[width=.6\columnwidth]{PartOne/ChapterOne/figures/imagingnegativethinlens.pdf}
      \caption{Ray diagram of the problem, the position of the image is $z'$ and is located to the right of the object. Dashed lines correspond to 
      virtual rays.}
    \end{figure}
    Using the thin lens equation, considering that $F'=-100\;mm$ and $z=-50\;mm$ provides
    \begin{align*}
      \frac{1}{z'}&=\frac{1}{F'}+\frac{1}{z}\\
      \frac{1}{z'}&=\frac{1}{-100}+\frac{1}{-50}\\
      z'&=\frac{(-100)(-50)}{-150}=-33.333\;mm.
    \end{align*}
    Because $z'<0$, the image is \textbf{virtual} and will be to the left of the lens. Its magnification is:
    \begin{align*}
      m=\frac{z'}{z}=\frac{-33.333}{-50}=0.667.
    \end{align*}
    The image is then erected ($\text{sgn}(m)=1$), and demagnified ($|m|<1$) making it smaller than the object.
\end{example}

%
\subsection{Optical spaces}
Any optical object creates two optical spaces: the object space and the image space. Each optical space extends from $-\infty$ 
to $\infty$ and has an associated index of refraction. The connection of both spaces is through the optical object.
\begin{figure}[h!]
    \centering
    \includegraphics[width=.6\columnwidth]{PartOne/ChapterOne/figures/opticalspace.png}
    \caption{}
\end{figure}
A \bfemph{real object} is to the left of the object while a \bfemph{virtual object} is to the right. 
A \bfemph{real image} is to the right of the object and a \bfemph{virtual image} to the left.
In an optical space with negative index, left and right are reversed in these descriptions.

A thin lens creates two optical spaces:
\begin{itemize}[itemsep=0pt,topsep=0pt]
    \item\textbf{Object space} contains the object and the front focal point $F$.
    \item\textbf{Image space} contains the image and the rear focal point $F'$.
\end{itemize}
\begin{figure}[h!]
\centering
\begin{subfigure}{.45\columnwidth}
    \centering
    \begin{circuitikz}[scale=.6]
    \def\f{1}
    \draw[arrow](-4,0)--(4,0)node[right]{$z$};
    \draw[arrow](0,-4)--(0,4)node[above]{$z'/m$};
    \draw[very thick,NavyBlue,domain=-3:-1.35,samples=100] plot(\x,{ (\x*\f)/(\x+\f) });
    \draw[very thick,NavyBlue,domain=-.8:3,samples=100] plot(\x,{ (\x*\f)/(\x+\f) });
    \draw[very thick,OliveGreen,domain=-3:-1.25,samples=100] plot(\x,{ (\f)/(\x+\f) });
    \draw[very thick,OliveGreen,domain=-.75:3,samples=100] plot(\x,{ (\f)/(\x+\f) });
    \draw[dashed](-4,1)--(4,1)(-1,4)--(-1,-4);
    \foreach \t in{-3,...,3}{\draw(\t,.1)--(\t,-.1)node[below]{$\t$};}
    \foreach \y in{-3,-2,-1,1,2,3}{\draw(-.1,\y)--(.1,\y)node[right,fill=white]{$\y$};}
    \begin{scope}[shift={(1.5,3.5)}] % move legend to top-right
        \draw[very thick,NavyBlue](0,0)--(0.8,0)node[right,black]{$z'$};
        \draw[very thick,OliveGreen](0,-0.5)--(0.8,-0.5)node[right,black]{$m$};
    \end{scope}
    \end{circuitikz}
    \caption{$f>0$}
    \label{fig:2a1}
\end{subfigure}
\hfill
\begin{subfigure}{.45\columnwidth}
    \centering
    \begin{circuitikz}[scale=.6]
    \def\f{-1}
    \draw[arrow](-4,0)--(4,0)node[right]{$z$};
    \draw[arrow](0,-4)--(0,4)node[above]{$z'$};
    \draw[very thick,NavyBlue,domain=-3:.8,samples=100] plot(\x,{ (\x*\f)/(\x+\f) });
    \draw[very thick,NavyBlue,domain=1.35:3,samples=100] plot(\x,{ (\x*\f)/(\x+\f) });
    \draw[very thick,OliveGreen,domain=-3:.75,samples=100] plot(\x,{ (\f)/(\x+\f) });
    \draw[very thick,OliveGreen,domain=1.25:3,samples=100] plot(\x,{ (\f)/(\x+\f) });
    \draw[dashed](-4,-1)--(4,-1)(1,4)--(1,-4);
    \foreach \t in{-3,...,3}{\draw(\t,-.1)--(\t,.1)node[above]{$\t$};}
    \foreach \y in{-3,-2,-1,1,2,3}{\draw(.1,\y)--(-.1,\y)node[left,fill=white]{$\y$};}
    \begin{scope}[shift={(1.5,3.5)}] % move legend to top-right
        \draw[very thick,NavyBlue](0,0)--(0.8,0)node[right,black]{$z'$};
        \draw[very thick,OliveGreen](0,-0.5)--(0.8,-0.5)node[right,black]{$m$};
    \end{scope}
    \end{circuitikz}
    \caption{$f<0$}
\end{subfigure}
\caption{Plot of $z'$ and $m$ for positive and negative lenses. We can directly see when an object (image) is real or virtual.}
\end{figure}

Virtual objects occur when an image is projected into the lens by a previous optical system. 
\begin{figure}
    \centering
    \includegraphics[width=.6\columnwidth]{PartOne/ChapterOne/figures/virtualobject.png}
    \caption{A virtual object is the projection of the image of a previous optical system.}
\end{figure}


\subsection{Object-image approximations}
We can make further appproximations for extreme situations:
\begin{itemize}[itemsep=0pt,topsep=0pt]
    \item\textbf{Distant object} When the magnitude of the object distance $z$ is more then a few times the magnitude of the system focal length, the image distance $z'$ is approximately equal to the real focal length.
    \begin{align}
        |z|\gg|f|\Longrightarrow a)\;z'\approx f\quad b)\;L=z'-z\approx f-z\approx-z\quad c)\;m=\frac{z'}{z}\approx\frac{f}{z}.
    \end{align} 
    \item\textbf{Distant image} Similarly,
    \begin{align}
        |z'|\gg|f|\Longrightarrow a)\;z\approx-f\quad b)\;L=z'-z\approx z'+f\approx z'\quad c)\;m=\frac{z'}{z}\approx-\frac{z'}{f}.
    \end{align}
\end{itemize}
This fraction error of these approximations is about $|f|/|z|$ so they are very useful for distant object/image from over $4f$.

\subsection{Field of view}
The half \bfemph{field of view} (HFOV) is often expressed as:
\begin{itemize}[itemsep=0pt,topsep=0pt]
    \item maximum object height $h$.
    \item maximum image height $h'$.
    \item maximum angular size of the object as seen from the optical system $\theta_{1/2}$.
    \item maximum angular size of the image as seen from the optical system $\theta_{1/2}'$.
\end{itemize}
\begin{figure}[h!]
    \centering
    \includegraphics[width=.6\columnwidth]{PartOne/ChapterOne/figures/fov.png}
    \caption{}
\end{figure}
\begin{align}
    \text{FOV}=2\text{HFOV},\quad \text{HFOV}=\theta_{1/2}=\theta_{1/2}',\quad \tan\theta_{1/2}=\frac{h}{z}=\frac{h'}{z'}.
\end{align}
In many situations, the FOV is determines by the detector size, which impose the maximum spatial dimensions. 

%%
\subsection{Afocal systems}
An \bfemph{afocal system} does not have focal points. Parallel rays will produce parallel images. The only change is in the transverse magnification.
\begin{align}
    m=\frac{h'}{h}=\frac{-f_2}{f_1}.
\end{align}
\begin{figure}
    \centering
    \begin{subfigure}{.45\columnwidth}
        \includegraphics[width=\columnwidth]{PartOne/ChapterOne/figures/keplerian.png}
        \caption{Keplerian telescope ($m<0$).}        
    \end{subfigure}
    \hfill
    \begin{subfigure}{.45\columnwidth}
        \includegraphics[width=\columnwidth]{PartOne/ChapterOne/figures/galilean.png}
        \caption{Galilean telescope ($m>0$).}        
    \end{subfigure}
\end{figure}
