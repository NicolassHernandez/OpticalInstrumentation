\section{Chromatic effects}
\subsection{Chromatic aberration}
\bfemph{Axial chromatic aberration} or \bfemph{axial color} is a variation of the system focal lenfgth with wavelength. This 
aberration derives from the dispersion of the glass as the index changes with wavelength $n(\lambda)$.

\begin{figure}[h!]
    \centering
    \begin{subfigure}{.6\columnwidth}
        \centering
        \includegraphics[width=.8\columnwidth]{PartOne/ChapterThree/figures/chromaticaberration.png}
        \caption{Chromatic aberration, $f>0$}
    \end{subfigure}
    \hfill
    \begin{subfigure}{.35\columnwidth}
        \centering
        \includegraphics[width=.8\columnwidth]{PartOne/ChapterThree/figures/dispersioncurve.png}
        \caption{Disperison curve}
    \end{subfigure}
    \hfill
    \begin{subfigure}{.6\columnwidth}
        \centering
        \includegraphics[width=.8\columnwidth]{PartOne/ChapterThree/figures/negativelenschromaticaberration.png}
        \caption{Chromatic aberration, $f<0$}
    \end{subfigure}
\end{figure}

Because of the higher index for F light, blue light is bent more and therefore the blue focus is closest to the lens.
\begin{emphasizer}[How much does the focal length change for the F and C wavelengths?]
    We look at the difference in power between these two wavelengths:
\begin{align*}
    \delta\phi&=\phi_F-\phi_C=(n_F-1)(C_1-C_2)-(n_C-1)(C_1-C_2)=(n_F-n_C)(C_1-C_2)\\
    \delta\phi&=\underbrace{\frac{n_F-n_C}{n_d-1}}_{1/\nu}\underbrace{(n_d-1)(C_1-C_2)}_{\phi_d}=\frac{\phi_d}{\nu}.
\end{align*}
Similarly, for the focal length:
\begin{align*}
    \delta f=f_C-f_F=\frac{1}{\phi_C}-\frac{1}{\phi_F}=\frac{\phi_F-\phi_C}{\phi_C\phi_F}=\frac{\delta\phi}{\phi_C\phi_F}\approx\frac{\delta\phi}{\phi_d^2}=\frac{\phi_d}{\nu\phi_d^2}=\frac{f_d}{\nu}.
\end{align*}
\end{emphasizer}

The foci of F, d and C are not evenly spaced due to the shape of the dispersion curve. The relative order of the foci is reversed for a negative lens.
\begin{align}
    \text{+ lens chromatic aberration}&\qquad\highlight{\delta f_{CF}=f_C-f_F,\quad\delta\phi_{FC}=\phi_F-\phi_C,\quad\frac{\delta f_{CF}}{f_d}=\frac{\delta\phi_{FC}}{\phi_d}=\frac{1}{\nu}}\\
    \text{- lens chromatic aberration}&\qquad\highlight{\text{same, with }f_d<0,\quad\phi_d<0,\quad\delta f_{CF}<0,\quad\delta\phi_{FC}<0}.
\end{align}
%
\subsection{Type of chromatic aberrations}
\subsubsection{Longitudinal chromatic aberration}
The blur associated with the chromatic aberration of the objective lens limits the performance of an objective.
To reduce the blur, a small diameter objective lens is required.
\begin{align*}
    \text{Blur}\propto\text{Lens diameter}.
\end{align*}
\begin{figure}[h!]
    \centering
    \begin{subfigure}{.45\columnwidth}
        \centering
        \includegraphics[width=.8\columnwidth]{PartOne/ChapterThree/figures/axialchromaticaberration.png}
        \caption{Axial longitudinal chromatic aberration}
    \end{subfigure}
    \hfill
    \begin{subfigure}{.45\columnwidth}
        \centering
        \includegraphics[width=\columnwidth]{PartOne/ChapterThree/figures/axialchromaticaberration1.png}
        \caption{Small diameter to reduce aberration}
    \end{subfigure}
\end{figure}
%
\subsubsection{Transverse axial chromatic aberration}
\bfemph{Transverse axial chromatic aberration} measures the image blur size due to axial chromatic aberration. It depends only on the glass and the pupil radius $r_P$ (stop at the lens):
The rays from the edge of the pupil are approximately parallel in the vecinity of the focus ($f\gg\delta f$).
\begin{align*}
    \tan U'&=-\frac{\text{TA}_{CH}}{\delta f}=-\frac{r_p}{f}\longrightarrow\frac{\text{TA}_{CH}}{\delta f}=\frac{r_P}{f}\longrightarrow\frac{\text{TA}_{CH}}{r_p}=\frac{\delta f}{f}=\frac{1}{\nu}.
\end{align*}
Therefore,
\begin{align}
    \text{Transverse axial chromatic aberration}\qquad\highlight{\text{TA}_{CH}=\frac{r_P}{\nu}}.
\end{align}
\begin{figure}[h!]
    \centering
    \begin{subfigure}{.6\columnwidth}
        \centering
        \includegraphics[width=.8\columnwidth]{PartOne/ChapterThree/figures/transversechromaticaberration.png}
        \caption{Transverse axial chromatic aberration}
    \end{subfigure}
    \hfill
    \begin{subfigure}{.3\columnwidth}
        \centering
        \includegraphics[width=\columnwidth]{PartOne/ChapterThree/figures/transversechromaticaberration1.png}
        \caption{Color swaped, should be blue first}
    \end{subfigure}
\end{figure}
%
\subsubsection{Lateral chromatic aberration}
\bfemph{Lateral chromatic aberration} or \bfemph{lateral color} is the chromatic aberration of the marginal ray and is caused by dispersion of the chief ray. The edge of the lens behaves 
like a thin prism. Off-axis image points will exhibit a radial color smear. The blur length increases linearly with the image height. Each color has a different
lateral magnification.
\begin{figure}[h!]
    \centering
    \includegraphics[width=.3\columnwidth]{PartOne/ChapterThree/figures/lateralchromaticaberration.png}
    \caption{Lateral chromatic aberration}
\end{figure}
%
\subsection{Achromatic doublet}
The thin lens \bfemph{achromatic doublet} corrects longtiduinal chromatic aberration by combining a positive element with a negatvie one.
Two different glasses $(\nu_1,P_1)$ and $(\nu_2,P_2)$ are used. The nominal powers and focal lengths are for d light.
\begin{figure}[h!]
    \centering
    \begin{subfigure}{.4\columnwidth}
        \centering
        \includegraphics[width=.8\columnwidth]{PartOne/ChapterThree/figures/achromaticdoublet.png}
        \caption{Achromatic doublet}
    \end{subfigure}
    \hfill
    \begin{subfigure}{.15\columnwidth}
        \centering
        \includegraphics[width=.7\columnwidth]{PartOne/ChapterThree/figures/achromaticdoublet1.png}
        \caption{Composition}
    \end{subfigure}
    \hfill
    \begin{subfigure}{.4\columnwidth}
        \centering
        \includegraphics[width=\columnwidth]{PartOne/ChapterThree/figures/secondaryachromaticdoublet.png}
        \caption{Secondary aberration}
    \end{subfigure}
\end{figure}

\begin{emphasizer}[How do we design the individual powers of the achromatic doublet?]
    Red and blue light are made to focus at the same location ($\Delta f=0$):
    \begin{align*}
        &\phi=\phi_1+\phi_2\Longrightarrow\delta\phi_{FC}=\delta\phi_{FC1}+\delta\phi_{FC2}=\frac{\phi_1}{\nu_1}+\frac{\phi_2}{\nu_2}=\phi_F-\phi_C=\textcolor{red}{0}\longrightarrow\highlight{\frac{\phi_1}{\nu_1}=-\frac{\phi_2}{\nu_2}}.\\
        &\phi=\phi_2-\frac{\nu_1}{\nu_2}\phi_2=\frac{\nu_2-\nu_1}{\nu_2}\phi_2\longrightarrow=\highlight{\frac{\phi_2}{\phi}=-\frac{\nu_2}{\nu_1-\nu_2}\land\frac{\phi_1}{\phi}=\frac{\nu_1}{\nu_1-\nu_2}}.
    \end{align*}
\end{emphasizer}

The doublet design places excess power in the positive element that is cancelleed by the negative element. Both contribute equal, but opposite, amounts of longitudinal chromatic aberration.
Large differences in the Abbe numbers minimize the excess power and provide better performance.
\begin{figure}[h!]
    \centering
    \includegraphics[width=.4\columnwidth]{PartOne/ChapterThree/figures/excesspower.png}
    \caption{Excess power in achromatic doublet}
\end{figure}
%
\subsection{Secondary chromatic aberration}
We have ensured the F and C lines are at the same longitudinal distance, but the d line have a different focal position.
\begin{figure}[h!]
    \centering
    \includegraphics[width=.5\columnwidth]{PartOne/ChapterThree/figures/secondarychromaticaberration.png}
    \caption{Secondary chromatic aberration.}
\end{figure}

First, we have:
{\small
\begin{align*}
    \delta\phi_{dC}&=\delta\phi_{dC1}+\delta\phi_{dC2}
\end{align*}}
We derive each term:
{\small
\begin{align*}
    \delta\phi_{dC1}&=(n_{d1}-1)(C_1-C_2)-(n_{c1}-1)(C_1-C_2)=(n_{d1}-n_{C1})(C_1-C_2)\\
   \delta\phi_{dC1}&=\frac{(n_{d1}-n_{C1})}{(n_{F1}-n_{C1})}(n_{F1}-n_{C1})(C_1-C_2)=P_1\delta\phi_{FC1}.
\end{align*}}
Similarly,
\begin{align*}
    \delta\phi_{dC1}=(n_{d1}-n_{C1})(C_1-C_2),\quad\delta\phi_{FC1}=(n_{F1}-n_{C1})(C_1-C_2),\quad P_1=\frac{n_{d1}-n_{C1}}{n_{F1}-n_{C1}},\quad P_2=\frac{n_{d2}-n_{C2}}{n_{F2}-n_{C2}}.
\end{align*}
We go back to the initial equation:
\begin{align*}
    \delta\phi_{dC}&=\delta\phi_{dC1}+\delta\phi_{dC2}\\
    &=P_1\delta\phi_{FC1}+P_2\delta\phi_{FC2}\\
    &=P_1\frac{\phi_1}{\nu_1}+P_2\frac{\phi_2}{\nu_2}\\
    &=\frac{P_1-P_2}{\nu_1-\nu_2}\phi\quad\text{Achromat eq: }\frac{\phi_1}{\nu_1}=-\frac{\phi_2}{\nu_2}\\
    \delta\phi_{dC}&=\frac{\Delta P}{\Delta \nu}\phi.
\end{align*}
\begin{emphasizer}
    In order to obtain zero secondary achromatic aberration with a doublet, the partial dispersion ratio $P$ of the two glasses must be zero:
    \begin{align}
        \Delta P=0\Longrightarrow\delta f_{Cd}=\frac{\Delta P}{\Delta\nu}f=0.
    \end{align}
    To correct chromatic aberration at additional wavelength, more than two glasses are used.
\end{emphasizer}
%
\subsection{Partial dispersion ratio and Abbe number}
On a plot of P versus $\nu$, most glasses lie on a straight line.
\begin{figure}[h!]
    \centering
    \includegraphics[width=.4\columnwidth]{PartOne/ChapterThree/figures/pversusnu.png}
    \caption{Partial dispersion ratio versus Abbe number.}
\end{figure}
For a singlet and doublet, we have:
\begin{align}
    \delta f_{CF}\approx\frac{f}{50}\quad\text{vs}\quad\delta f_{Cd}\approx\frac{f}{2200}.
\end{align}
%
\begin{example}{Design of an achromatic doublet}
    Design a $160\;mm$ focal length thin-lens ahcromatic doublet using the following glases. Provide the focal lengths and indices of refraction of the two thin lenses.
    \begin{align*}
        \text{Glass 1: Fused Silica, 458678},\quad\text{Glass 2: SF6, 805254}.
    \end{align*}
    \subsubsection{Solution}
    \begin{align*}
        \text{Glass 1: $n_1=1458,\;\nu_1=67.8$},\quad\text{Glass 2: $n_2=1.805,\;\nu_2=25.4$}.
    \end{align*}
    \begin{align*}
        &\frac{1}{f_2}=-\frac{\nu_2}{\nu_1-\nu_2}\frac{1}{f}=-\frac{25.4}{67.8-25.4}\frac{1}{160}=-0.00374\;mm^{-1}\longrightarrow f_2=-267.380\;mm\\
        &\frac{1}{f_1}=\frac{\nu_1}{\nu_1-\nu_2}\frac{1}{f}=\frac{67.8}{67.8-25.4}\frac{1}{160}=0.0010\;mm^{-1}\longrightarrow f_1=100\;mm.
    \end{align*}
\end{example}