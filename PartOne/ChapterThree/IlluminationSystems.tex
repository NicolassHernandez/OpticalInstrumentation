\section{Illumination systems}
\subsection{Illumination systems and types}
The illumination system provides light for the optical system. Important considrations are the amount of light,
the uniformity, and the angular spread of the light as seen by the object.

A \bfemph{projector} is the general term for an imaging system that also provudes the illumination for the object.
\begin{figure}[h!]
    \centering
    \includegraphics[width=.6\columnwidth]{PartOne/ChapterThree/figures/projector.png}
    \caption{Projector as the illumination system.}
\end{figure}

There are three basic classifications of illumination systems:
\begin{itemize}[itemsep=0pt,topsep=0pt]
    \item\textbf{Diffuse illumination} Light with a large angular spread is incident on the object. There is no attempt to image the source into the imaging system.
    It provides uniform illumination but is light inefficient.
    \begin{align*}
        \text{No source coupling}
    \end{align*}
    \item\textbf{Specular illumination} The light source is imagesd by the condenser optics into the EP of the imaging optics. As it is good light eficient, is used for 
    most optical systems with an integral light source.
    \begin{align*}
        \text{Source to pupil coupling}
    \end{align*}
    \item\textbf{Critical illumination} The light source is imaged directly onto the object.
    \begin{align*}
        \text{Source to object coupling}
    \end{align*}
\end{itemize}

