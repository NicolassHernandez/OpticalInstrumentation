\section{Thin prisms and dispersing prisms}


\subsection{Thin prisms}
\bfemph{Thin prisms} introduce small angular beam deviations $\delta$ that is approximately independen of the incident angle:
\begin{align}
    \text{Thin prism deviation}\qquad\highlight{\delta\approx-(n-1)\alpha}.
\end{align}
The deviation is measured in prism diopters. A prisms of 1 diopter deviates a beam by $1\;cm$ at $1\;m$.
The beam deviation is parallel to a principal section of the prism and towards the thick end of the prism. 
\begin{figure}[h!]
    \centering
    \begin{subfigure}{.25\columnwidth}
        \centering
        \includegraphics[width=.8\columnwidth]{PartOne/ChapterThree/figures/thinprism.png}
        \caption{Thin prism}
    \end{subfigure}
    \hfill
    \begin{subfigure}{.25\columnwidth}
        \centering
        \includegraphics[width=.9\columnwidth]{PartOne/ChapterThree/figures/thinprismderivation.png}
        \caption{Thin prism derivation}
    \end{subfigure}
    \hfill
    \begin{subfigure}{.2\columnwidth}
        \centering
        \includegraphics[width=.8\columnwidth]{PartOne/ChapterThree/figures/beamdeviation.png}
        \caption{Beam deviation}
    \end{subfigure}
    \hfill
    \begin{subfigure}{.25\columnwidth}
        \centering
        \includegraphics[width=\columnwidth]{PartOne/ChapterThree/figures/risleyprism.png}
        \caption{Risley prism}
    \end{subfigure}
\end{figure}
\begin{emphasizer}[Proof of the thin prism deviation]
    We tip the prism by $\theta$ so that the front face is perpendicular to the input ray (no refraction):
    \begin{align}
    \begin{array}{l}
        \theta=-\alpha\\
        \theta'=\delta-\alpha
    \end{array}\longrightarrow\begin{array}{rl}
        n\theta&=\theta'\quad(\text{small angle})\\
        -n\alpha&=\delta-\alpha\\
        \delta&\approx-(n-1)\alpha
    \end{array}.
\end{align}
\end{emphasizer}

The magnitude and direction of the deviation defines a vector perpendicular to the optical axis (xy plane). The net deviation vector for a series of 
thin prisms is then the vector sum of the copmonent vectors:
\begin{align*}
    \text{Algeba of deviation}\qquad\highlight{\bm{\delta}=\bm{\delta}_1+\bm{\delta}_2}.
\end{align*}
%%
\subsection{Risley prism}
A \bfemph{Risley prism} consists of a pair of identical, but opposing, thin prisms. The prisms are counter-rotated by $\pm\beta$ to obtain 
a variable net deviation in a fixed direction. The Risley prism allows the fine angular alignment for an optical system by adjusting the prism 
orientation $\beta$.
%%
\subsection{Thin prism dispersion}
\subsubsection{Thin prism}
The dispersion of a thin prism $\Delta$ measures the total angular spread from $C$ to $F$ light, and the \bfemph{secondary dispersion} $\varepsilon$ gives
the spread from the $C$ to $d$ wavelengths. The results depend on the index $n_d$, the Abbe number $\nu$ and the partial dispersion ratio $P$ of the glass:
\begin{align}
    \text{Deviation}&\quad\highlight{\delta=-(n_d-1)\alpha}\\
    \text{Dispersion}&\quad\highlight{\Delta=}(n_F-1)(-\alpha)-(n_C-1)(-\alpha)=\highlight{-(n_F-n_C)\alpha}\\
    \text{Secondary dispersion}&\quad\highlight{\varepsilon=}(n_d-1)(-\alpha)-(n_C-1)(-\alpha)=\highlight{-(n_d-n_C)\alpha}.
\end{align}
From above and the definition of the Abbe number, we can also get the following relations
\begin{align}
    \frac{\delta}{\Delta}=\frac{n_d-1}{n_F-n_C}=\nu\longrightarrow\highlight{\Delta=\frac{\delta}{\nu}},\quad\text{and}\quad\frac{\varepsilon}{\Delta}=\frac{n_d-n_C}{n_F-n_C}=P\longrightarrow\highlight{\varepsilon=P\frac{\delta}{\nu}}.
\end{align}
\begin{figure}[h!]
    \centering
    \begin{subfigure}{.25\columnwidth}
        \centering
        \includegraphics[width=.8\columnwidth]{PartOne/ChapterThree/figures/prismdispersion.png}
        \caption{Thin prism dispersion}
    \end{subfigure}
    \hfill
    \begin{subfigure}{.45\columnwidth}
        \centering
        \includegraphics[width=\columnwidth]{PartOne/ChapterThree/figures/edgethinprism.png}
        \caption{Edge of a thin prism}
    \end{subfigure}
    \hfill
    \begin{subfigure}{.25\columnwidth}
        \centering
        \includegraphics[width=.8\columnwidth]{PartOne/ChapterThree/figures/achromaticwedge.png}
        \caption{Achromatic thin prism}
    \end{subfigure}
\end{figure}

The edge of a thin lens can be considered to be a thin prism. The angle $\alpha$ depend on the power and diameter of the lens. Each wavelength will be deviated by a different angle. 
The focal length of the lens is defined for d light. From the above figure, we have the following relation.
\begin{emphasizer}[How does the difference in focal length is related with the Abbe number?]
    \begin{align*}
        &\delta(\lambda)=-(n_\alpha-1)\alpha,\quad f(\lambda)=-\frac{r}{\delta(\lambda)},\quad f_d=-\frac{r}{\delta_d}\\
        &\delta f=f_C-f_F=-\frac{r}{\delta_C}+\frac{r}{\delta_F}=-r\frac{\delta_F-\delta_C}{\delta_F\delta_C}\approx-r\frac{\delta_F-\delta_C}{\delta_d^2}.\\
        &\frac{\delta f}{f_d}=\frac{-r\frac{\delta_F-\delta_C}{\delta_d^2}}{-\frac{r}{\delta_d}}=\frac{\delta_F-\delta_C}{\delta_d}=\frac{-(n_F-1)\alpha+(n_C-1)\alpha}{-(n_d-1)\alpha}=\frac{n_F-n_C}{n_d-1}=\frac{1}{\nu}\longrightarrow\highlight{\frac{\delta f}{f_d}=\frac{1}{\nu}}.
    \end{align*}
\end{emphasizer}

An inverted prism deviates a ray up and has a negative vertex angle $\alpha$.

Deviations and dispersions adds:
\begin{align}
    \text{Deviation}\quad\delta=\sum_i\delta_i,\quad\text{Dispersion}\quad\Delta=\sum_i\Delta_i,\quad\text{Seoncdary dispersion}\quad\varepsilon=\sum_i\varepsilon_i.
\end{align}
%
\subsubsection{Achromatic thin prism}
An \bfemph{achromatic thin prism} or \bfemph{achromatic wedge} provides deviation without dispersion. Opposite prisms made from two 
different glasses $(n_{d1},\nu_1,P_1)$ and $(n_{d2},\nu_2,P_2)$ are combined to force the dispersion between the $F$ and $C$ wavelengths to be 
zero ($\Delta=0$):
\begin{align*}
    \Delta=\Delta_1+\Delta_2=\frac{\delta_1}{\nu_1}+\frac{\delta_2}{\nu_2}=0\longrightarrow\highlight{\delta_2=-\frac{\nu_2}{\nu_1}\delta_1}.
\end{align*}
A deviation of $\delta$ is maintained for $d$ light:
\begin{align*}
    \delta=\delta_1+\delta_2=\delta_1-\frac{\nu_2}{\nu_1}\delta_1=(\nu_1-\nu_2)\frac{\delta_1}{\nu_1}=-(\nu_1-\nu_2)\frac{(n_{d1}-1)\alpha_1}{\nu_1}.
\end{align*}
Doing $\alpha_1/\delta$ and $\alpha_2/\delta$ gives the deviation ratio for each element
\begin{align}
    \text{Achromatic deviation ratio}\qquad\highlight{\frac{\alpha_1}{\delta}=\frac{1}{\nu_2-\nu_1}\frac{\nu_1}{n_{d1}-1},\quad\frac{\alpha_2}{\delta}=-\frac{1}{\nu_2-\nu_1}\frac{\nu_2}{n_{d2}-1}}.
\end{align}

The high-dispersion prism is inverted to obtain an opposing deviation. While the F and C wavelegths are corrected, a residual secondary dispersion remains.
The residual dispersion is the secondary dispersion $\varepsilon$:
\begin{align*}
    \varepsilon=\varepsilon_1+\varepsilon_2=P_1\frac{\delta_1}{\nu_1}+P_2\frac{\delta_2}{\nu_2}.
\end{align*}
From the achromatic condition $\delta_2=-(\nu_2/\nu_1)\delta_1$ we have:
\begin{align*}
    \nu_1\delta_2&=-\nu_2\delta_1\\
    \nu_1(\delta-\delta_1)&=-\nu_2\delta_1\\
    \nu_1\delta&=(\nu_1-\nu_2)\delta_1\\
    \frac{\delta_1}{\nu_1}&=\frac{\delta}{\nu_1-\nu_2},\quad\frac{\delta_2}{\nu_2}=-\frac{\delta}{\nu_1-\nu_2}.
\end{align*}
Putting these two ratios back in the secondary dispersion and rearranging yields:
\begin{align}
    \text{Secondary dispersion in the thin prism}\qquad\highlight{\frac{\varepsilon}{\delta}=\frac{P_2-P_1}{\nu_2-\nu_1}=\frac{\Delta P}{\Delta\nu}}.
\end{align}

For most glass pairs, d light will be bent more than the F and C wavelengths and also $\frac{\varepsilon}{\delta}>0$. The shape of the curve of $|\delta|-\lambda$ 
is concave and the maximum dispersion \textbf{does not occur} at d light.
\begin{figure}[h!]
    \centering
    \begin{subfigure}{.45\columnwidth}
        \centering
        \includegraphics[width=.8\columnwidth]{PartOne/ChapterThree/figures/secondarydispersion.png}
        \caption{Secondary disperion curve}
    \end{subfigure}
    \hfill
    \begin{subfigure}{.45\columnwidth}
        \centering
        \includegraphics[width=.5\columnwidth]{PartOne/ChapterThree/figures/directvisionprism.png}
        \caption{Direct vision prism}
    \end{subfigure}
\end{figure}

The achromatic thin prism has about 40 less secondary dispersion compared to a simple thin prism:
{\small
\begin{align*}
    \text{Siple thin prism ($\nu\approx[30,70]$)}\quad\frac{\Delta P}{\Delta\nu}=\frac{1}{\nu}\approx[1.5,3]\%\quad\text{vs}\quad\text{Achromatic thin prism}\quad\frac{\Delta P}{\Delta\nu}\approx0.045\%.
\end{align*}}
%
\subsubsection{Direct vision prism}
A \bfemph{direct vision prism} uses opposing prisms to provide dispersion without deviation of the d light ($\delta=\delta_1+\delta_2=0$).
For a desired dispersion $\Delta$, we have 
{\small
\begin{align*}
    \Delta=\Delta_1+\Delta_2=\frac{\delta_1}{\nu_1}+\frac{\delta_2}{\nu_2}=-\frac{\nu_1-\nu_2}{\nu_1\nu_2}\delta_1.
\end{align*}}
Consequently,
\begin{align}
    \text{Dispersion in a direct vision prism}\qquad\begin{array}{l}
    \highlight{\displaystyle\frac{\alpha_1}{\Delta}=\left(\frac{\nu_1\nu_2}{\nu_1-\nu_2}\right)\left(\frac{1}{n_{d1}-1}\right)}\\
    \highlight{\displaystyle\frac{\alpha_2}{\Delta}=-\left(\frac{\nu_1\nu_2}{\nu_1-\nu_2}\right)\left(\frac{1}{n_{d2}-1}\right)}
    \end{array}.
\end{align}
In order to get dufficient dispersion, it is ioften necessary to use more than two prisms.
%
\subsection{Dispersing prism}
The total deviation $\delta$ in the \bfemph{dispersion prism} is the sum of the deviations at the two surfaces:
\begin{align*}
    \text{Total deviation}\qquad\highlight{\delta=\alpha-\sin^{-1}[\sqrt{n^2-\sin^2\theta}\sin\alpha-\cos\alpha\sin\theta]-\theta}.
\end{align*}
There is a minimum deviation angle $\delta_{\text{min}}$, at which the ray path through the prism is symetric $\theta'=-\theta$. 
The ray is bent an equal amount at each surface. The deviation is negative for the orientation of the prism in the figure. 
The \bfemph{angle of minimum deviation} is 
\begin{align}
    \delta_{\text{min}}=\alpha-2\sin^{-1}[n\sin(\alpha/2)].
\end{align}
\begin{figure}[h!]
    \centering
    \begin{subfigure}{.25\columnwidth}
        \centering
        \includegraphics[width=\columnwidth]{PartOne/ChapterThree/figures/dispersingprism.png}
        \caption{Dispersing prism}
    \end{subfigure}
    \hfill
    \begin{subfigure}{.2\columnwidth}
        \centering
        \includegraphics[width=\columnwidth]{PartOne/ChapterThree/figures/deviationinputangle.png}
        \caption{Deviation in $\theta$}
    \end{subfigure}
    \hfill
    \begin{subfigure}{.2\columnwidth}
        \centering
        \includegraphics[width=.7\columnwidth]{PartOne/ChapterThree/figures/disersingprismtable.png}
        \caption{Table for $\delta_{\text{min}}$}
    \end{subfigure}
    \hfill
    \begin{subfigure}{.25\columnwidth}
        \centering
        \includegraphics[width=.8\columnwidth]{PartOne/ChapterThree/figures/dispersingprismdispersion.png}
        \caption{Dispersion of the prism}
    \end{subfigure}
\end{figure}

The measurement of the index depends only on $\delta_{\text{min}}$ and the prism apex angle $\alpha$:
\begin{align}
    n=\frac{\sin[(\alpha-\delta_{\text{min}})/2]}{\sin(\alpha/2)}.
\end{align}
Minimum deviation provides an extremely accurate method of measuring the index of refraction of a material.
Prism spectrometers can obtain accuracies of one part in $10^6$.
\begin{align}
    \delta_{\text{min}}=I_2-I_1.
\end{align}
\begin{figure}[h!]
    \centering
    \includegraphics[width=.5\columnwidth]{PartOne/ChapterThree/figures/prismspectrometer.png}
    \caption{Prism spectrometer to measure the index of refraction.}
\end{figure}

\begin{itemize}[itemsep=0pt,topsep=0pt]
    \item Telescope and collimator mounted in a rotation stage.
    \item Use the telescope as autocollimator to measure the angle $\alpha$.
    \item Measure the straight through angle $I_1$ without prism.
    \item Insert the prism an observe the angle of refracted beam $\delta$.
    \item Rotate the prism to obtain the minimum deviation and measure this angle $I_2$.
    \item Make $\delta_{\text{min}}=I_2-I_1$.
    \item Calculate the index with the formula above.
\end{itemize}