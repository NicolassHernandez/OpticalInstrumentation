\section{Thin prisms and dispersing prisms}


\subsection{Thin prisms}
\bfemph{Thin prisms} introduce small angular beam deviations $\delta$ that is approximately independen of the incident angle:
\begin{align}
    \delta\approx-(n-1)\alpha.
\end{align}
The deviation is measured in prism diopters. A prisms of 1 diopter deviates a beam by $1\;cm$ at $1\;m$.
The beam deviation is parallel to a principal section of the prism and towards the thick end of the prism. De magnitude and 
direction of the deviation defines a vector perpendicular to the optical axis (xy plane). The net deviation vector for a series of thin prisms 
is then the vector sum of the copmonent vectors:
\begin{align*}
    \bm{\delta}=\bm{\delta}_1+\bm{\delta}_2.
\end{align*}
\begin{figure}[h!]
    \centering
    \begin{subfigure}{.3\columnwidth}
        \centering
        \includegraphics[width=.8\columnwidth]{PartOne/ChapterThree/figures/thinprism.png}
        \caption{Thin prism}
    \end{subfigure}
    \hfill
    \begin{subfigure}{.3\columnwidth}
        \centering
        \includegraphics[width=.8\columnwidth]{PartOne/ChapterThree/figures/beamdeviation.png}
        \caption{Beam deviation}
    \end{subfigure}
    \hfill
    \begin{subfigure}{.3\columnwidth}
        \centering
        \includegraphics[width=.8\columnwidth]{PartOne/ChapterThree/figures/risleyprism.png}
        \caption{Risley prism}
    \end{subfigure}
\end{figure}
\begin{emphasizer}[Proof of the deviation]
    We tip the prism by $\theta$ so that the front face is perpendicular to the input ray (no refraction):
    \begin{align}
    \delta=-\alpha,\delta'=\delta-\alpha\longrightarrow\begin{array}{l}
        n\theta=\theta'\\
        -n\alpha=\delta-\alpha\\
        \delta\approx-(n-1)\alpha
    \end{array}.
\end{align}
\end{emphasizer}
%%
\subsection{Risley prism}
A \bfemph{Risley prism} consists of a pair of identical, but opposing, thin prisms. The prisms are counter-rotated by $\pm\beta$ to obtain 
a variable net deviation in a fixed direction. The Risley prism allows the fine angular alignment for an optical system by adjusting the prism 
orientation $\beta$.
%%
\subsection{Thin prism dispersion}
\subsubsection{Thin prism}
The \bfemph{dispersion of a thin prism} $\Delta$ measures the total angular spread from $C$ to $F$ light, and the \bfemph{secondary dispersion} $\epsilon$ gives
the spread from the $C$ to $d$ wavelengths. The results depend on the index $n_d$, Abbe number $\nu$ and partial dispersion ratio $P$ of the glass:
\begin{align}
    \text{Deviation}&\quad\delta=-(n_d-1)\alpha\\
    \text{Dispersion}&\quad\Delta=-(n_F-n_C)\alpha,\quad\Delta=\frac{\delta}{\nu}\\
    \text{Secondary dispersion}&\quad\varepsilon=-(n_d-n_C)\alpha,\quad\varepsilon=P\Delta=P\frac{\delta}{\nu}.
\end{align}
\begin{figure}[h!]
    \centering
    \begin{subfigure}{.3\columnwidth}
        \centering
        \includegraphics[width=.7\columnwidth]{PartOne/ChapterThree/figures/prismdispersion.png}
        \caption{Thin prism dispersion}
    \end{subfigure}
    \hfill
    \begin{subfigure}{.4\columnwidth}
        \centering
        \includegraphics[width=\columnwidth]{PartOne/ChapterThree/figures/edgethinprism.png}
        \caption{Edge of a thin prism}
    \end{subfigure}
    \hfill
    \begin{subfigure}{.25\columnwidth}
        \centering
        \includegraphics[width=.8\columnwidth]{PartOne/ChapterThree/figures/achromaticwedge.png}
        \caption{Achromatic thin prism}
    \end{subfigure}
\end{figure}

\begin{emphasizer}[How does the difference in focal length is related with the Abbe number?]
    \begin{align*}
        &\delta f=f_C-f_F=-\frac{r}{\delta_C}+\frac{r}{\delta_F}=-r\frac{\delta_F-\delta_C}{\delta_F\delta_C}\approx-r\frac{\delta_F-\delta_C}{\delta_d^2}.\\
        &\frac{\delta f}{f_d}=\frac{-r\frac{\delta_F-\delta_C}{\delta_d^2}}{-\frac{r}{\delta_d}}=\frac{\delta_F-\delta_C}{\delta_d}=\frac{-i(n_F-1)\alpha+(n_C-1)\alpha}{-(n_d-1)\alpha}=\frac{n_F-n_C}{n_d-1}=\frac{1}{\nu}\longrightarrow\frac{\delta f}{f_d}=\frac{1}{\nu}.
    \end{align*}
\end{emphasizer}

An inverted prism deviates a ray up and has a negative vertex angle $\alpha$.

Deviations and dispersions adds.
\begin{align}
    \delta=\sum_i\delta_i,\quad\Delta=\sum_i\Delta_i,\quad\varepsilon=\sum_i\varepsilon_i.
\end{align}
%
\subsubsection{Achromatic thin prism}
An \bfemph{achromatic thin prism} or \bfemph{achromatic wedge} provides deviation without dispersion. Opposite prisms made from two 
different glasses $(n_{d1},\nu_1,P_1)$ and $(n_{d2},\nu_2,P_2)$ are combined to force the dispersion between the $F$ and $C$ wavelengths to be 
zero. A deviation of $\delta$ is maintained for $d$ light:
\begin{align}
    \text{Achromatic relations}\qquad\highlight{\frac{\alpha_1}{\delta}=\frac{1}{\nu_2-\nu_1}\frac{\nu_1}{n_{d1}-1},\quad\frac{\alpha_2}{\delta}=-\frac{1}{\nu_2-\nu_1}\frac{\nu_2}{n_{d2}-1}}.
\end{align}

\begin{emphasizer}[Proof of the above relations]
    We force the dispersion of F and C to be zero:
\begin{align*}
    \Delta=\Delta_1+\Delta_2=\frac{\delta_1}{\nu_1}+\frac{\delta_2}{\nu_2}=0\longrightarrow\delta_2=-\frac{\nu_2}{\nu_1}\delta_1.
\end{align*}
A deviation $\delta$ for d light is maintained:
\begin{align*}
    \delta=\delta_1+\delta_2=\delta_1-\frac{\nu_2}{\nu_1}\delta_1=(\nu_1-\nu_2)\frac{\delta_1}{\nu_1}=-(\nu_1-\nu_2)\frac{(n_{d1}-1)\alpha_1}{\nu_1}.
\end{align*}
Doing $\alpha_1/\delta$ and $\alpha_2/\delta$ yields the above results.
\end{emphasizer}

The high-dispersion prism is inverted to obtain an opposing deviation. While the F and C wavelegths are corrected, a residual secondary dispersion remains.
For most glass pairs, d light will be bent more than the F and C wavelengths:
\begin{align}
    \text{Secondary aberration}\qquad\highlight{\frac{\varepsilon}{\delta}=\frac{P_2-P_1}{\nu_2-\nu_1}=\frac{\Delta P}{\Delta\nu}}.
\end{align}

For most glases, $\frac{\varepsilon}{\delta}>0$. The shape of the curve of $|\delta|-\lambda$ is concave and the maxium dispersion does not occur at d ligth.
The achromatic thin prism has about 40 less secondary dispersion compared to a simple thin prism. 
\begin{figure}[h!]
    \centering
    \begin{subfigure}{.45\columnwidth}
        \centering
        \includegraphics[width=\columnwidth]{PartOne/ChapterThree/figures/secondarydispersion.png}
        \caption{Secondary disperion curve}
    \end{subfigure}
    \hfill
    \begin{subfigure}{.45\columnwidth}
        \centering
        \includegraphics[width=.6\columnwidth]{PartOne/ChapterThree/figures/directvisionprism.png}
        \caption{Direct vision prism}
    \end{subfigure}
\end{figure}
%
\subsubsection{Direct vision prism}
A \bfemph{direct vision prism} uses opposing prisms to provide dispersion without deviation of the d light.

We first set the total dispersion to zero: $\delta=\delta_1+\delta_2=0$. Then,
{\small
\begin{align*}
    \Delta=\Delta_1+\Delta_2=\frac{\delta_1}{\nu_1}+\frac{\delta_2}{\nu_2}=-\frac{\nu_1-\nu_2}{\nu_1\nu_2}\delta_1\longrightarrow\highlight{\frac{\alpha_1}{\Delta}=\left(\frac{\nu_1\nu_2}{\nu_1-\nu_2}\right)\left(\frac{1}{n_{s1}-1}\right)}\land\highlight{\frac{\alpha_2}{\Delta}=-\left(\frac{\nu_1\nu_2}{\nu_1-\nu_2}\right)\left(\frac{1}{n_{d2}-1}\right)}.
\end{align*}}


\subsection{Dispersing prism}
The total deviation $\delta$ in the \bfemph{dispersion prism} is the sum of the deviations at the two surfaces:
\begin{align*}
    \text{Total deviation}\qquad\highlight{\delta=\alpha-\sin^{-1}[\sqrt{n^2-\sin^2\theta}\sin\alpha-\cos\alpha\sin\theta]-\theta}.
\end{align*}
There is a minimum deviation angle $\delta_{\text{min}}$, at which the ray path through the prism is symetric $\theta'=-\theta$. 
The ray is bent an equal amount at each surface. The deviation is negative for the orientation of the prism in the figure. 
The \bfemph{angle of minimum deviation} is 
\begin{align}
    \delta_{\text{min}}=\alpha-2\sin^{-1}[n\sin(\alpha/2)].
\end{align}
The following table shows $\delta_{\text{min}}$ for several $n$.
\begin{figure}[h!]
    \centering
    \begin{subfigure}{.25\columnwidth}
        \centering
        \includegraphics[width=\columnwidth]{PartOne/ChapterThree/figures/dispersingprism.png}
        \caption{Dispersing prism}
    \end{subfigure}
    \hfill
    \begin{subfigure}{.2\columnwidth}
        \centering
        \includegraphics[width=\columnwidth]{PartOne/ChapterThree/figures/deviationinputangle.png}
        \caption{Deviation in $\theta$}
    \end{subfigure}
    \hfill
    \begin{subfigure}{.2\columnwidth}
        \centering
        \includegraphics[width=.7\columnwidth]{PartOne/ChapterThree/figures/disersingprismtable.png}
        \caption{Table for $\delta_{\text{min}}$}
    \end{subfigure}
    \hfill
    \begin{subfigure}{.25\columnwidth}
        \centering
        \includegraphics[width=.8\columnwidth]{PartOne/ChapterThree/figures/dispersingprismdispersion.png}
        \caption{Dispersion of the prism}
    \end{subfigure}
\end{figure}

The measurement of the index depends only on $\delta_{\text{min}}$ and the prism apex angle $\alpha$:
\begin{align}
    n=\frac{ \sin\dfrac{\alpha-\delta_{\text{min}}}{2} }{\sin(\alpha/2)}.
\end{align}
Prism spectrometers can obtain accuracies of one part in $10^6$.
