\section{Magnifiers and Telescopes}
%%
\subsection{Magnifiers}
The largest image magnification possible with the unaided eye occurs when the object is placed at the \bfemph{near point} of the eye, by convention, $250\;mm$ or $10\;in$.
A \bfemph{magnifier} is a single lens that provides an enlarged \textbf{erect virtual image} of a nearby object for visual observation. The object must be placed 
within the fron focal length of the lens.
\begin{figure}[h!]
    \centering
    \includegraphics[width=.6\columnwidth]{PartOne/ChapterTwo/magnifier.png}
\end{figure}

The \bfemph{magnifying power} MP is defined as (stop at the eye):
\begin{align*}
    \text{MP}&=\frac{\text{Angular size of the image (with lens)}}{\text{Angular size of the object at the near point}}\\
    &=\frac{\bar{u}_M}{\bar{u}_U}=\frac{h'/(z'-s)}{h/d_{NP}},\quad d_{NP}=-250\;mm\\
    &=\frac{250\;mm(z'-f)}{f(z'-s)}\approx\frac{250\;mm}{f}.
\end{align*}
The approximation relation is the most common definition of the MP. It assumes that the lens is close to the eye and the image is presented to a relaxed eye ($z'=\infty$).

The angular substense $\theta$ of the image $h'$ at the eye is 
\begin{align}
    \theta=\frac{h\text{MP}}{250\;mm}.
\end{align}
The resolution of the human eye is about $1\;arcmin$, or $(1/60)^\circ$. In order to resolve an object of size $h$, th required MP is then 
\begin{align*}
    \text{MP}\geq\frac{0.075\;mm}{h}.
\end{align*}
Magnifiers up t about 25X are practical; 10X is common.

%%
\subsection{Telescopes}
Telescopes are afocal systems used for visual observation of distant objects. The image through the telescope subtends an angle $\theta'$ different from the angle subtend by the object $\theta$.
The magnifying power of a telescope is:
\begin{align}
    \text{MP}=\frac{\theta'}{\theta}=\begin{cases}
        |\text{MP}|>1,&\text{Telescope magnifies}\\
        |\text{MP}|<1,&\text{Telescope minifies}
    \end{cases}.
\end{align}

In Keplerian and Galilean telescope, the lateral magnification $m$ is given by:
\begin{align}
    \text{Lateral magnification of telescope}\qquad m=\frac{1}{\text{MP}}=-\frac{f_{\text{EYE}}}{f_{\text{OBJ}}}.
\end{align}
It is important to notice the reciprocal relation between the magnification and the magnifying power. For instance an image smaller than the object, also have the image much closer, so that the 
apparent size is much larger.
\begin{figure}[h!]
    \centering
    \includegraphics[width=.4\columnwidth]{PartOne/ChapterTwo/magnificationmagnifyingpower.png}
    \caption{In this case, $m<1$ while also $\text{MP}>1$: height and distance are important for these quantities.}
\end{figure}
%%
\subsubsection{Keplerian telescope}
A Keplerian telescope or astronomical telescope consists of an objective that creates an \bfemph{aerial image} (real image in the air) followed by a magnifier
separated by $f_1+f_2$. The system stop is usually at or near the objective lens. 

The Keplearian has a negative MP: $\text{MP}<1$ and the image presented to the eye is inverted and reverted (rotated $180^\circ$). The eye should be placed at the real XP to couple the eye to the telescope and see the entire FOV, if not vignetting may occur.
The XP position from the last surface is called the \bfemph{eye relief} ER.  The magnification of the telescope related the diameters of EP and XP:
\begin{align}
    \text{ER}=z'=(1-m)f_{\text{EYE}},\quad D_{\text{XP}}=|m|D_{\text{EP}}.
\end{align}
The XP of a visual instrument is also known as the \bfemph{eye circle} or the \bfemph{Ramsden circle}.
%
\subsubsection{Galilean telescope}
The Galilean telescope uses a positive lens followed by a negative lens to obtain an erect image and a positive MP $\text{MP}>1$.
In this case, the XP is internal o virtual and not accesible to the eye. The FOV of the system is therefore small. There is no intermediate image plane.

\begin{emphasizer}
For a given $|\text{MP}|$, the Galilean telescope is shorter and the FOV smaller than the corresponding Keplerian telescope.
\end{emphasizer}

\begin{figure}[h!]
    \centering
    \begin{subfigure}{.49\columnwidth}
        \centering
        \includegraphics[width=\columnwidth]{PartOne/ChapterTwo/kepleariantelescope.png}
        \caption{Keplerian telescope}
    \end{subfigure}
    \hfill
    \begin{subfigure}{.49\columnwidth}
        \centering
        \includegraphics[width=\columnwidth]{PartOne/ChapterTwo/galileantelescope.png}
        \caption{Galilean telescope}
    \end{subfigure}
\end{figure}

We describe several important points:
\begin{itemize}[itemsep=0pt,topsep=0pt]
    \item Usually in telescope, the objective is the stop so that this lens is also the EP. Keplerian have real XP to the right of the eye lens.
    \item A \bfemph{reversed Galilean telescope} provides a minified erect image $0<\text{MP}<1$ and the eye is usually the system stop.
    \item \bfemph{Binoculars} are a pair of parallel telescope for each eye.
    \item The specification provided on telescope and binoculars is of the form 
    \begin{align}
        \text{AXB}\Longrightarrow A=|\text{MP}|,\quad\text{and}\quad B=\text{Objective diameter in mm}.
    \end{align} 
\end{itemize}

\begin{example}{Eye relief of a Keplerian telescope}
    \begin{figure}[h!]
        \centering
        \includegraphics[width=.7\columnwidth]{PartOne/ChapterTwo/eyereliefkeplerian.png}
    \end{figure}
\end{example}

%%
\subsection{Field lenses}
The FOV of the Keplerian telescope is limited by vignetting at the eye lens. A \bfemph{field lens} placed at the intermediate image plane 
increases the FOV by bending the ray bundle into the aperture of the eye lens.
\begin{figure}[h!]
    \centering
    \begin{subfigure}{.45\columnwidth}
        \centering
        \includegraphics[width=\columnwidth]{PartOne/ChapterTwo/fieldlend.png}
        \caption{Field lens}
    \end{subfigure}
    \hfill
    \begin{subfigure}{.45\columnwidth}
        \centering
        \includegraphics[width=.8\columnwidth]{PartOne/ChapterTwo/fieldlens2.png}
        \caption{Reduction of the ER}
    \end{subfigure}
\end{figure}

This combination is called an \bfemph{eyepiece}. If we assume the separation between them is equal to the focal length of the eye lens, then the overall power is:
\begin{align*}
    \phi=\phi_{\text{EYE}}+\phi_{\text{FIELD}}-\phi_{\text{EYE}}\phi_{\text{FIELD}}\cdot\frac{1}{\phi_{\text{EYE}}}=\phi_{\text{EYE}}.
\end{align*}
Thus, the eyepiece maintain its focal length and therefore its MP. The shifts of the principal planes are:
\begin{align}
    d=\frac{\phi_{\text{EYE}}}{\phi_{\text{EYE}}}\frac{1}{\phi_{\text{EYE}}}=f_{\text{EYE}},\quad\text{and}\quad d'=-\frac{\phi_{\text{FIELD}}}{\phi_{\text{EYE}}}\frac{1}{\phi_{\text{EYE}}}=-\frac{f^2_{\text{EYE}}}{f_{\text{FIELD}}}.
\end{align}
To summarize, the inclusion of the field lens:
\begin{itemize}[itemsep=0pt,topsep=0pt]
    \item Decrease vigneting incresing in the same way the FOV.
    \item Does not change the MP of the telescope or the size of the XP.
    \item The front principal plane is unchanged, but the rear principal plane and ER are shifted by $d'$.
    \item The field lens is usually also displaced from the image plane to minimize the inclusion of information of the field lens to the image through defocus.
    \item MP of the eyepiece is the same as the MP of a magnifier.
\end{itemize}

%%
\subsection{Eyepieces}
An \bfemph{eyepiece} or \bfemph{ocular} is the combination of the field lens and the eye lens. A simple eye piece does not have a field lens. A compound eyepiece has both of them.
A field stop can be placed at the intermediate image plane to restrict the system FOV. This aperture serves to correct the vignetting. 

Two special eyepiece configuations displace the field lens from the intermediate image:
\begin{itemize}[itemsep=0pt,topsep=0pt]
    \item \textbf{Huygens eyepiece} has the intermediate image between the two elements.
    \item \textbf{Ramsden eyepiece} has intermediate image in front of the field lens. This configuration has about $50\%$ more ER than the Huygens eyepiece.
    \item \textbf{Kellner eyepiece} replaces the singlet eye lens of the Ramsden eyepiece with a doublet for color correciton.
\end{itemize} 
\begin{figure}[h!]
    \centering
    \begin{subfigure}{.45\columnwidth}
        \centering
        \includegraphics[width=.8\columnwidth]{PartOne/ChapterTwo/huygenseyepiece.png}
        \caption{Huygens eyepiece}
    \end{subfigure}
    \hfill
    \begin{subfigure}{.45\columnwidth}
        \centering
        \includegraphics[width=.8\columnwidth]{PartOne/ChapterTwo/ramsdeneyepiece.png}
        \caption{Ramsden eyepiece}
    \end{subfigure}
\end{figure}

Hand-held instruments should have $15-20\;mm$ of eye relief. Microscopes may have as little as $2-3\;mm$. Other systems, such as riflescopes, shoud have a very long eye relief.
The human eye pupil diameter varies from $2-8\;mm$, with a diameter of about $4\;mm$ under ordinary lighting conditions. When overfiled, the eye becomes the system stop.
%
\subsection{Mirror-based Telescopes}
The imaging properties of conic surfaces are used in the design of \bfemph{mirror-based telescopes}.
\begin{itemize}[itemsep=0pt,topsep=0pt]
    \item \textbf{Newtonian telescope} uses a parabola with a fold flat. It is analogous to a Keplerian refracting telescope.
    \item \textbf{Gregorian telescope} uses a parabola followed by an ellipse to relay the intermediate image. It prooduces an erect image.
    \item \textbf{Cassegrain telescope} uses a parabola combined with a hyperbolic secondary mirror to reduce the system length. It is equivalent to a telephoto objective. The two conic surfaces 
    correct the spherical aberration.
    \item \textbf{Ritchey-Chretien telescope} is identical to Cassegrain telescope in layout, except that it uses two hyperbolic mirrors to correct coma as well as spherical aberration.
\end{itemize}
\begin{figure}[h!]
    \centering
    \begin{subfigure}{.3\columnwidth}
        \centering
        \includegraphics[width=.7\columnwidth]{PartOne/ChapterTwo/newtonian}
        \caption{Newtonian}
    \end{subfigure}
    \hfill
    \begin{subfigure}{.3\columnwidth}
        \centering
        \includegraphics[width=\columnwidth]{PartOne/ChapterTwo/gregorian}
        \caption{Gregorian}
    \end{subfigure}
    \hfill
    \begin{subfigure}{.3\columnwidth}
        \centering
        \includegraphics[width=\columnwidth]{PartOne/ChapterTwo/cassegrain}
        \caption{Cassegrain}
    \end{subfigure}
\end{figure}


\begin{example}{Vignetting in a Keplerian telescope}
    \begin{figure}
        \centering
        \includegraphics[width=.3\columnwidth]{PartOne/ChapterTwo/keplerianvignetting1.png}
        \includegraphics[width=.3\columnwidth]{PartOne/ChapterTwo/keplerianvignetting2.png}
        \includegraphics[width=.3\columnwidth]{PartOne/ChapterTwo/keplerianvignetting3.png}
    \end{figure}
\end{example}

\begin{example}{Vignetting in a Galilean telescope}
    \begin{figure}
        \centering
        \includegraphics[width=.45\columnwidth]{PartOne/ChapterTwo/galileanvignetting1.png}
        \includegraphics[width=.45\columnwidth]{PartOne/ChapterTwo/galileanvignetting2.png}
    \end{figure}    
\end{example}