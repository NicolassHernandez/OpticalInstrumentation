\section{Objectives}
\subsection{Type of objectives}
\bfemph{Objectives} are lens element combinations used to image distance objects. To classify them, separated group of lens elements are modeled sa thin lenses.

\begin{itemize}[itemsep=0pt,topsep=0pt]
    \item \textbf{Simple objective} consists of a positive thin lens.
    \item \textbf{Collimator} is a reversed simple objective. It creates a collimated beam from a source at the system focal point.
    \item \textbf{Petzval objective} consists of two separated positive groups of elements. The rear principal plane is located between the groups.
    \item \textbf{Telephoto objective} produces a system focal length longer than the BFD. It consists of a positive element followed by a negative group.
    \item \textbf{Reverse telephoto objective} consists of a negative group followed a positive group. Used to produce a system with BFD larger than the system focal length.
\end{itemize}
\begin{figure}
    \centering
    \begin{subfigure}{.24\columnwidth}
        \centering
        \includegraphics[width=\columnwidth]{PartOne/ChapterTwo/simpleobjective.png}
        \caption{Simple objective}
    \end{subfigure}
    \hfill
    \begin{subfigure}{.24\columnwidth}
        \centering
        \includegraphics[width=\columnwidth]{PartOne/ChapterTwo/petzvalobjective.png}
        \caption{Petzval objective}
    \end{subfigure}
    \hfill
    \begin{subfigure}{.24\columnwidth}
        \centering
        \includegraphics[width=\columnwidth]{PartOne/ChapterTwo/telephotoobjective.png}
        \caption{Telephoto objective}
    \end{subfigure}
    \hfill
    \begin{subfigure}{.24\columnwidth}
        \centering
        \includegraphics[width=\columnwidth]{PartOne/ChapterTwo/reversetelephoto.png}
        \caption{Reverse Telephoto}
    \end{subfigure}
    \caption{Different type of objectives.}
\end{figure}

%%
\subsection{Depth of focus and field}
There is often some allowable image blut that defines the performance requirement of an optical system. No diffraction or aberrations are included.
%%
\subsubsection{Depth of focus}
The \bfemph{depth of focus} DOF describes the amount the detector can be shifted from the nominal image position (focal plane) for a fixed object before 
the resulting blur exceed the blur diameter criterion $B'$.
\begin{figure}[h!]
    \centering
    \begin{subfigure}{.45\columnwidth}
        \centering
        \includegraphics[width=.8\columnwidth]{PartOne/ChapterTwo/depthoffocus.png}
        \caption{Depth of focus}
    \end{subfigure}
    \hfill
    \begin{subfigure}{.45\columnwidth}
        \centering
        \includegraphics[width=\columnwidth]{PartOne/ChapterTwo/dofrelationfnumber.png}
        \caption{Depth of focus vs F-number}
    \end{subfigure}
\end{figure}

The $\pm b'$ transverse distance from the nominal point is used to define the DOF as the whole distance is $2b'$:
\begin{align}
    b'=\frac{B'L_0'}{D_{XP}}\approx\frac{B'z'}{D_{EP}},\quad \text{DOF}\approx\pm b'\approx\pm B'f/\#_W\approx\frac{B'}{2\text{NA}}.
\end{align}
The figure illustrates how the DOF is changed as the diameter of the lens is varied; they are inversely proportional.
%%
\subsubsection{Depth of field}
The \bfemph{depth of field} is the maximum distance, from $L_{\text{near}}$ to $L_{\text{far}}$, the object can move before exceed the aceptable blur $B'$ at the fixed image plane.
\begin{figure}[h!]
    \centering
    \includegraphics[width=.6\columnwidth]{PartOne/ChapterTwo/depthoffield.png}
\end{figure}
The following relations are given
\begin{align}
    L_{\text{far}}\approx\frac{L_0fD}{fD+L_0B'},\quad L_{\text{near}}\approx\frac{L_0fD}{fD-L_0B'}.
\end{align}
Al objects positiones between these distances will produces images on the detector that have geometrical blurs less than the blur criterion $B'$.
%%
\subsection{Hyperfocal distance}
When the far point of the depth of field $L_{\text{far}}$ is at infinity, the optical system is focused at the \bfemph{hyperfocal distance} $L_H$, and all objects from $L_{\text{near}}$ to 
infinity meet the image plane blur criterion.
\begin{align*}
    L_{\text{far}}=\infty\Longrightarrow fD+L_HB'=0.
\end{align*}
Solving for $L_H$,
\begin{align}
    \text{Hyperfocal distance}\qquad\highlight{L_H=-\frac{fD}{B'}=-\frac{f^2}{(f/\#)B'}}.
\end{align}
Substituting of $L_H$ in $L_{\text{near}}$ yields
\begin{align}
    L_{\text{near}}\approx-\frac{fD}{2B'}=\frac{L_H}{2}.
\end{align}
%%
\subsubsection{Hyperfocal distance and depth of}
The relation between the depth of focus and the hyperfocal distance where the detector is placed is, by the thin-lens equations,
\begin{align}
    L_H'\approx f+B'f=f+\text{DOF}.
\end{align}
\begin{emphasizer}
    \begin{itemize}[itemsep=0pt,topsep=0pt]
        \item Focusing at the hyperfocal distance \textbf{ensures} that any greater distance meets the blur criteria.
        \item As $f/\#$ increases (lens stopped down), the hyperfocal distance moves closer to the lens. 
    \end{itemize}
\end{emphasizer}
\begin{figure}[h!]
    \centering
    \begin{subfigure}{.45\columnwidth}
        \centering
        \includegraphics[width=.8\columnwidth]{PartOne/ChapterTwo/hyperfocal1.png}
    \end{subfigure}
    \hfill
    \begin{subfigure}{.45\columnwidth}
        \centering
        \includegraphics[width=\columnwidth]{PartOne/ChapterTwo/hyperfocal2.png}
    \end{subfigure}
\end{figure}

\begin{example}{Fixed-focus camera}
    \begin{figure}[h!]
        \centering
        \includegraphics[width=.6\columnwidth]{PartOne/ChapterTwo/fixedfocuscamera.png}
    \end{figure}
\end{example}

\begin{example}{Fixed-focus digital camera}
    \begin{figure}[h!]
        \centering
        \includegraphics[width=.6\columnwidth]{PartOne/ChapterTwo/fixedfocusdigital.png}
    \end{figure}    
\end{example}

%%
\subsection{Zoom lenses}
A \bfemph{zoom lens} is a variable focal length objective with a \textbf{fixed} image plane. The simplest system is composed of two groups with powers $\phi_1$ and $\phi_2$
where both the focal length $f$ and the BFD vary with the element spacing $t$.
\begin{align*}
    \phi=\frac{1}{f}=\phi_1+\phi_2-\phi_1\phi_2t\quad \text{BFD}=f+d'=f-\frac{\phi_1}{\phi}t.
\end{align*}
To vary the focal length with a fixed image plane, we:
\begin{itemize}[itemsep=0pt,topsep=0pt]
    \item Move the lens $L_1$ a distance $L=t+\text{BFD}$ from the image plane.
    \item Displace lens $L_2$ a distance $t$ from $L_1$. 
\end{itemize}
\begin{figure}[h!]
    \centering
    \begin{subfigure}{.49\columnwidth}
        \centering
        \includegraphics[width=\columnwidth]{PartOne/ChapterTwo/telephotozoom.png}
        \caption{Telephoto zoom (limited by BFD)}
    \end{subfigure}
    \hfill
    \begin{subfigure}{.49\columnwidth}
        \centering
        \includegraphics[width=\columnwidth]{PartOne/ChapterTwo/reversetelephotozoom.png}
        \caption{Reverse telephoto zoom (commonly used)}
    \end{subfigure}
\end{figure}

As the separation approaches the sum of the individual focal lengths ($f_1+f_2$), the system becomes afocal ($f\to\infty$).



A mechanical cam provides the complicated lens motion required for these \bfemph{mechanical compoensated} zoom lenses. A common three group configuration uses 
a fixed front element and moving second and third groups.
\begin{figure}[h!]
\centering
\includegraphics[width=.5\columnwidth]{PartOne/ChapterTwo/mechanicalcompensated.png}
\end{figure}


\begin{example}{Digital camera}
    You are using a digital camera to take a picture of a house against a distant
    mountain backdrop. The camera is focused on the house. You have a wide selection of
    camera lenses and 16:9 (width = 12.48mm, height = 7.02 mm) CCD detectors at your
    disposal. The CCD detectors available to you have pixels with sizes ranging between 3 µm
    and 6 µm, however, cameras with pixel sizes of exactly 3 µm and 6 µm are not available.
    You want to make sure that you can take a high-quality picture of the house and the
    mountains and therefore, want to achieve a diffraction-limited image based on the pixel
    pitch of the CCD detector you selected. The location allows you to place the camera
    anywhere between 10 and 30 meters (not exactly 10 nor 30 m) away from the house.
    Assume that the lens is a thin lens with the stop at the lens.
    \begin{enumerate}[itemsep=0pt,topsep=0pt,label=\alph*)]
        \item What are the hyperfocal distance and acceptable blur for your camera system?
        \item Determine the focal length, f, and the diameter, D, of the lens.
        \item How far behind the lens must the detector be located? What is the maximum size (height and width) 
        of the house in meters you would be able to image on the detector?
        \item What is the horizontal and vertical Field of View of this photographic system (in degrees)?
        \item What is the f/\# of this camera? What is the closest object that will be considered to be in focus?
        \item It is an overcast day (outdoor illuminance=103 lux=103 lm/m2). The reflectance of the house is r = 0.5 
        and the mountains have an average reflectance of r=0.18. What are the image plane illuminances for the house and the
        mountains individually?
    \end{enumerate}
    \subsubsection{Solution}
    \begin{enumerate}[itemsep=0pt,topsep=0pt,label=\alph*)]
        \item CCD pixel pitch is $4\;\mu m$. And camera at $20\;m$ away from the house.
        \item The hyperfocal distance is $L_H=20\;mm$. The blur is $B'\approx4\;\mu m\to f/\#=4$.
        \item The hyperfocal idstance can be related to focal length:
        \begin{align*}
            |L_H|=\frac{f^2}{B'f/\#}\longrightarrow f=\sqrt{L_HB'f/\#}=\sqrt{20\cdot4\;\mu m\cdot4}=17.889\;mm.
        \end{align*}
        The diameter is then:
        \begin{align*}
            D=\frac{f}{f/\#}=\frac{17.889\;mm}{4}=4.472\;mm.
        \end{align*}
        We can think os a single thin lens that images an object at $-20\;mm$ to $L_H$. 
        The magnification is:
        \begin{align*}
            m=\frac{L_H'}{L_H}=frac{z'}{z}=1\cdot1^{-3}.
        \end{align*}
        The field of view for each dimension is:
        \begin{align*}
            \text{HFOV}_H&=\tan^{-1}\frac{7.02}{L_H'}=\\
            \text{HFOV}_W&=\tan^{-1}\frac{12.48}{L_H'}=
        \end{align*}
        \item The near distance is half of the hyperfocal distance 
        \begin{align*}
            L_{\text{near}}=\frac{L_H}{2}=10\;mm.
        \end{align*}
        \item For the house, 
        \begin{align*}
            E'=\frac{M}{4(f/\#_W)^2}=\frac{\rho E}{4(f/\#_W)^2}=\frac{\rho E}{4(f/\#)^2(1-m)^2}.
        \end{align*}
        For the mountain (located at infinity):
        \begin{align*}
            E'=\frac{\rho E}{4(f/\#)^2}.
        \end{align*}
    \end{enumerate}
\end{example}