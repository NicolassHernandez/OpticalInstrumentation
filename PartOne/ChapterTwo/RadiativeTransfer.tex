\section{Radiative transfer}
\subsection{Radiometry}
\bfemph{Radiometry} characterizes the propagation of radiant energy through an optical system.
The basic unit is the watt W. Radiometric terminology and units are:
\begin{table}[h!]
    \centering
    \begin{tabular}{l|l|l|l}
        Quantity&Symbol&Units&Units description\\
        \hline
        Energy&$Q$&$J$&\\
        Flux&$\Phi$&$W$&Power\\
        Intensity&$I$&$W/sr$&Power per unit solid angle\\
        Irradiance&$E$&$W/m^2$&Incident power per unit area\\
        Exitance&$M$&$W/m^2$&Exiting power per unit area\\
        Radiance&$L$&$W/m^2sr$&Power per unit projectedarea per unit solid angle
    \end{tabular} 
\end{table}
There are some assumptions:
\begin{itemize}[itemsep=0pt,topsep=0pt]
    \item The source is \textbf{incoherent}, meaning that scenes are collection of independently point sources, no interference.
    \item Objects and images on-axis and perpendicular to the optical axis, so that the projected area equals the area.
\end{itemize}

The solid angle $\Omega$ equals the surface area of the unit sphere in a given vecinity. The units are $4\pi$ steradians (sr).
\begin{figure}[h!]
    \centering
    \includegraphics[width=.8\columnwidth]{PartOne/ChapterTwo/solidangle.png}
    \caption{The solid angle of a sphere can be approximated to the solid angle of a cone.}
\end{figure}

\begin{align*}
    d\Omega=\frac{dA}{r^2}=\frac{r^2\sin\theta d\theta d\phi}{r^2}=\sin\theta d\theta d\phi\stackrel{\int}{\longrightarrow}\highlight{\Omega_{\text{sphere}}=4\pi\sin^2(\theta_0/2)}.
\end{align*}
In optics, its common to approximate the solid angle of a sphere to the section of a cone:
\begin{align*}
    \Omega\approx\frac{\pi r_0^2}{d^2}\approx\pi\sin^2\theta_0\approx\pi\theta^2_0\quad(\text{small angle approximation}).
\end{align*}

\subsection{Radiative transfer}
\bfemph{Radiative transfer} uses first-order geometrical principles to determine the amount of light from an object that reaches an image or detector.

Exitance and irradiance are related by the \bfemph{reflectance} of the surface $\rho$:
\begin{align}
    M=\rho E.
\end{align} 
For average scenes, $\rho=18\%$. Exposures are often set using this value, that is, we expose the print to that average so that the prin reflectance ends up with $18\%$.

The irradiance of a \bfemph{Lambertian source} (perfectly diffuse surface) is constant. The intensity falls off with the apparaent source size or the \bfemph{orjected area} (\bfemph{Lambert's law}).
The exitance of a Lambertian source is related to its radiance by $\pi$.
\begin{align*}
    \text{Lambertian source}\qquad\highlight{\begin{array}{ll}
        L(\theta,\phi)=\text{constant}&I=I_0\cos\theta\\
        M=\pi L&\pi L=\rho E
    \end{array}}
\end{align*}

We now analyze the optical power from an object that reaches the image in an optical system.
\begin{figure}[h!]
    \centering
    \includegraphics[width=.8\columnwidth]{PartOne/ChapterTwo/radiativetransfer.png}
\end{figure}

In air, the radiance and the \bfemph{$A\Omega$ product} or \bfemph{throughput} are conserved, and the flux collected by the lens $\Phi$ is transferred to the image area $A'$:
\begin{align*}
    \Phi=L(\text{object area})(\text{solid angle projection in lens})=LA_p\Omega=LA\frac{\pi D_p^2}{4z^2}\stackrel{A'=m^2A}{\longrightarrow}\Phi=\frac{\pi LA'D_p^2}{4m^2z^2}.
\end{align*}

Using gaussian equations and f-number equations, the image plane irradiance $E'$ is
\begin{align}
    \text{Camera equation}\qquad\highlight{E'=\frac{\Phi}{A'}=\frac{\pi L}{4(f/\#_W)^2}\to\pi L(\text{NA})^2,\quad L=\frac{\rho E_0}{\pi}}.
\end{align}

\subsubsection{Spectral dependence}
Spectral dependence can also be added. 
\begin{align*}
    \begin{array}{l}
        E_0(\lambda)=\text{Object irradiance}\\
        \rho(\lambda)=\text{Object reflectance}\\
        L(\lambda)=\text{Object radiance}
    \end{array}\longrightarrow\begin{array}{l}
        L(\lambda)=\frac{M(\lambda)}{\pi}=\frac{\rho(\lambda)E_0(\lambda)}{\pi}\\
        E'(\lambda)=\frac{\rho(\lambda)E_0(\lambda)}{4(1-m)^2(f/\#)^2}
    \end{array}
\end{align*}
This can be integrated over all wavelengths for total irradiance 
\begin{align*}
    E'=\frac{1}{4(1-m)^2(f/\#)^2}\int_{\lambda_1}^{\lambda_2}\rho(\lambda)E_0(\lambda)\;d\lambda.
\end{align*}
%
\subsubsection{Exposure}

Most detectors respond to energy per unit area rathen than power per unit area. Multiplying the image irradiance by the exposure time gives the expoure ($J/m^2$):
\begin{align}
    \text{Exposure}\qquad\highlight{H=E'\Delta t}.
\end{align}
The mean solar constant is $1368\;W/m^2$ outside the atmosphere of the earth, and the solar irradiance on the surface is about $1000\;W/m^2$. 
%%
\subsection{Photometry}
\bfemph{Photometry} is the subset of radiometry that deals with visual measurements, and luminous power is measured in \bfemph{lumens} $lm$. The lumen is a watt
wighted to the visual \bfemph{photopic response}. This peak response occurs at $555\;nm$, where the conversion is $683\;lm/W$. The dark adapted or \bfemph{scotopic response} peaks 
at $507\;nm$ with $1700\;lm/W$.

\begin{table}[h!]
  \centering
  \begin{subtable}[t]{0.45\textwidth}
    \centering
    \caption{Photometric terminology}
    \begin{tabular}{lll}
      Luminous power&$\Phi_V$&$lm$\\
      Luminous intensity&$I_V$&$lm/sr$\\
      Illuminance&$E_V$&$lm/m^2$\\
      Luminous exitance&$M_V$&$lm/m^2$\\
      Luminance&$L_V$&$lm/m^2sr$\\
      Exposure&$H_V$&$lm\;s/m^2$\\
      \hline
      candela (cd)&$I_V$&$lm/sr$\\
      lux (lx)&$E_V$&$lm/m^2$\\
      foot-candle (fc)&&$lm/ft^2$\\
      &\multicolumn{2}{l}{$1fc=10.76\;lx$}\\
      foot-lambert (fL)&$L_V$&$\frac{1}{\pi}cd/ft^2$\\
      nit (nt)&&$=cd/m^2$\\
      &\multicolumn{2}{l}{$1fL=3.426\;nt$}\\
      lux-second (lx s)&$H_V$&$lm \;s/m^2$
    \end{tabular}
  \end{subtable}
  \hfill
  \begin{subtable}[t]{0.45\textwidth}
    \centering
    \caption{Luminous photopic efficacy}
    {\scriptsize
    \begin{tabular}{ll}
      $\lambda\;(nm)$&$lm/W$\\
      \hline
      400&0.3\\
      420&2.7\\
      440&15.7\\
      460&41.0\\
      480&95.0\\
      500&221\\
      520&485\\
      540&652\\
      560&680\\
      580&594\\
      600&425\\
      620&260\\
      640&120\\
      660&41.7\\
      680&11.6\\
      700&2.8\\
      720&0.7
    \end{tabular}}
  \end{subtable}
  \hfill
  \begin{subtable}[t]{\columnwidth}
    \centering 
    \caption{Typical illuminance levels}
    \begin{tabular}{llllllll}
        Sunny day&$10^5\;lx$&Moonlit night&$10^{-1}\;lx$&Overcast day&$10^3\;lx$&Starry night&$10^{-3}\;lx$\\
        Inerior&$10^2\;lx$&Desk lighting&$10^3\;lx$&&&&
    \end{tabular}
  \end{subtable}
\end{table}

The candela (cd) is the fundamental SI unit for luminous intensity.

%%
\subsubsection{$A\Omega$ product}
Recall the flux hrough a system is 
\begin{align*}
    \Phi=LA\Omega.
\end{align*}
The $A\Omega$ product appears to be the geometric portion, while $L$ would be related to the source characteristics.
In an object or an image plane,
\begin{align*}
    \text{object/pupil plane}\qquad A=\pi\bar{y}^2,\quad\theta=u,\quad A\Omega=\pi^2\bar{y}^2u^2=\pi^2\chi^2/n^2,\quad\chi=n\bar{y}u\\
    \text{pupil plane}\qquad A=\pi y^2,\quad\theta=\bar{u},\quad A\Omega=\pi^2y^2\bar{u}^2=\pi^2\chi^2/n^2,\quad\chi=ny\bar{u}
\end{align*}

In the general situation when the index is not unity, the \bfemph{basic throughout} $n^2A\Omega$ and the \bfemph{basic radiance} $L/n^2$ are invariant. Since throughput is 
based on areas, the basic throughput is proportional to the Lagrange invariant squared:
\begin{align}
    n^2A\Omega=\pi^2\chi^2.
\end{align}

For a lossless optical system, the flux through the system is constant. For $n=1$, we have 
\begin{align*}
    \Phi=L_1A_1\Omega_1=L_2A_2\Omega_1=\cdots=\text{constant}.
\end{align*}
Since $A\Omega$ is also constant, the radiance $L$ must also be constant. This allow us to relate different portions of the optical system as the flux is conserved.

However, if the index of refraction is not unity and changes ,the radiance is no longer conserved. It will change at each interface as the solid angle will change.
\begin{figure}[h!]
    \centering
    \includegraphics[width=.5\columnwidth]{PartOne/ChapterTwo/fluxconservation.png}
\end{figure}
We then have that 
\begin{align*}
    A\Omega_1\neq A\Omega_2\land\Omega=L_1A\Omega_1=L_2A\Omega_2\Longrightarrow L_1\neq L_2.
\end{align*}
The flux is still constant. In fact, $L/n^2=\text{constant}$ as well as $n^2A\Omega=\text{constant}$.