\section{Relays and Microscopes}

\subsection{Prisms and relay lenses}
For many applications, it is important that the image have the same orientation as the object. The commonly used methods for image erection is to use prism, and relay sistems.

On the one hand, prisms as porro or Pechan-roof prism may help to erect the image. It is important that the ray bundle must be sized so the entrance and exit faces do not vignette the FOV.
Unfolding the prism make possible to treat it as a plane parallel plate of glass.

On the other hand, the \bfemph{relay lens} consists of an additional lens that takes the image of the objective (object for this lens) and image it to the right. This image is the object for the 
eyepiece. The net MP of the relayed Keplerian telescope in the figure is positive and equals to the product of the magnification of the relay and the MP of the original Keplerian telescope:
\begin{align}
    \text{MP}=m_R\text{MP}_K=-\frac{z_R'f_{\text{OBJ}}}{z_Rf_{\text{EYE}}},\quad\text{where}\quad m_R=\frac{z_R'}{z_R}.
\end{align}
\begin{figure}[h!]
    \centering
    \begin{subfigure}{.45\columnwidth}
        \centering
        \includegraphics[width=.7\columnwidth]{PartOne/ChapterTwo/prism.png}
        \caption{Using a prism}
    \end{subfigure}
    \hfill
    \begin{subfigure}{.45\columnwidth}
        \centering
        \includegraphics[width=\columnwidth]{PartOne/ChapterTwo/relay.png}
        \caption{Using a relay lens}
    \end{subfigure}
\end{figure}

Multiple relay lenses can be used to transfer the imager over long distances, such as in periscoopes, andoescopes and borescopes. Field lenses can also be added at the intermediate lenses.

The functions of a field lens and a relay lens ca be combined into a single \bfemph{erector lens}. This lens will require a diameter larger than the replaced field 
or relay lenses. The relayed image and pupil are shifter from their original positions.
\begin{figure}[h!]
    \centering
    \includegraphics[width=.5\columnwidth]{PartOne/ChapterTwo/erectorlens.png}
\end{figure}

\begin{example}{Keplerian telescope with relay lens}
    A 30X relayed Keplerian telescope is constructed out of three thin lenses in
    air. The relay lens of the telescope operates with a magnification of 1.5. The focal length
    of the objective lens is 400 mm, and the overall telescope length is 500 mm.
    \begin{enumerate}[itemsep=0pt,topsep=0pt,label=\alph*)]
        \item Determine the design of the telescope.
        \item Assuming that the system stop is located at the objective lens, determine the eye relief of the telescope (distance from the eye lens to the XP).
    \end{enumerate}
    \subsubsection{Solution}
    \begin{enumerate}[itemsep=0pt,topsep=0pt,label=\alph*)]
        \item The magnification of the relay lens and the total magnification are:
        \begin{align*}
            m_R=\frac{z_R'}{z_R}=-1.5,\quad \text{MP}=m_R\text{MP}_K=(-1.5)(-\frac{f_1}{f_2})=30\longrightarrow f_2=\frac{400\cdot1.5}{30}=20\;mm.
        \end{align*}
        Therefore, the remaining space is $80\;mm$ which is distribtued for the relay lens.
        \begin{align*}
            -z_R+z_R'=80\;mm.
        \end{align*}
        Using it with the relay magnification we have that
        \begin{align*}
            z_R=-32\;mm,\quad\text{and}\quad z_R'=48\;mm.
        \end{align*}
        The focal length of the relay lens is obtained with the thin-lens equation:
        \begin{align*}
            \frac{1}{z_R'}=\frac{1}{z_R}+\frac{1}{f_R}\longrightarrow f_R=19.2\;mm.
        \end{align*}
        \item The eye relief is located at the image position in image space of the stop. The relay lens take that object located at $z=-(32+400)=-432\;mm$.
        \begin{align*}
            z'=\frac{1}{\frac{1}{-432}+\frac{1}{19.2}}=20.093\;mm.
        \end{align*}
        This image is located at $z=-(20+48-20.093)=-47.907\;mm$ to the eye lens and is seen as the object, which is imaged to:
        \begin{align*}
            z'_{\text{XP}}=\text{ER}=\frac{1}{\frac{1}{-47.907}+\frac{1}{20}}=34.333\;mm.
        \end{align*}
    \end{enumerate}
\end{example}

%%
\subsection{Microscopes}
\subsubsection{Definition}
A \bfemph{microscope} is a sophisticated magnifier consisting of an objective plus an eyepiece.
The \bfemph{visual magnification} is the product of the objective magnification and the eyepiece MP:
\begin{align}
    \text{Visual magnification}\qquad\highlight{m_V=m_{\text{OBJ}}\text{MP}_{\text{EYE}}=\frac{z_0'}{z_0}\frac{250\;mm}{f_{\text{EYE}}}}.
\end{align}
\begin{figure}[h!]
    \centering
    \begin{subfigure}{.6\columnwidth}
        \centering
        \includegraphics[width=.9\columnwidth]{PartOne/ChapterTwo/microscope.png}
        \caption{Structure}
    \end{subfigure}
    \hfill
    \begin{subfigure}{.3\columnwidth}
        \centering
        \includegraphics[width=\columnwidth]{PartOne/ChapterTwo/objectivemicroscope.png}
        \caption{Numerical aperture}
    \end{subfigure}
\end{figure}

The \bfemph{optical tube length} OTL of a microscope is defined as the distance from the rear focal point of the objective to the front focal point of the eyepiece (intermediate image).
Standard values are $160\;mm$ and $125\;mm$. The OTL is a Newtonian mage distance:
\begin{align}
    m_{\text{OBJ}}=-\frac{\text{OTL}}{f_{\text{OBJ}}}.
\end{align}

The NA of a microscope objective is defined in object space by the half-angle of the accepted input ray bundle. Along with the objective magnification, the NA is inscribed on the objective barrel:
\begin{align}
    \text{NA}=n\sin\theta.
\end{align}
Microscope objective are often telecentric in object space. The stop is placed at the rear focal point of the objective so that the magnification does not change with defocus.
%%
\subsubsection{Microscope terminology}
\begin{itemize}[itemsep=0pt,topsep=0pt]
    \item The \bfemph{working distance} WD is the distance from the object to the first element of the objective; can be less than $1\;mm$ for high-power objectives.
    \item The \bfemph{mechanical tube length} is the separation between the shoulder of the threaded mount of the objetctive an the end of the tube into which the eyepiece is inserted.
    \item A set of \bfemph{parfocal objectives} have different magnifications, but the same \bfemph{shoulder height} and the same shoulder-to-intermediate image distance.
    \item \bfemph{Biological objectives} are aberration corrected assuming a cover glass between the object and the objective.
    \item Research-graded microscopes are usually designed using \bfemph{infinity corrected objectives}. The object plane is the front focal plane of the objective, and a collimated beam results for each point. 
\end{itemize}
\begin{figure}[h!]
    \centering
    \begin{subfigure}{.3\columnwidth}
        \centering
        \includegraphics[width=.5\columnwidth]{PartOne/ChapterTwo/shoulderheight.png}
    \end{subfigure}
    \hfill
    \begin{subfigure}{.6\columnwidth}
        \centering
        \includegraphics[width=\columnwidth]{PartOne/ChapterTwo/microscopeinternal.png}
    \end{subfigure}
\end{figure}

The magnification of the objective-tube lens combination is 
\begin{align}
    m_{\text{OBJ}}=-\frac{f_{\text{TUBE}}}{f_{\text{OBJ}}}.
\end{align}
If the objective is object-space telecentric and $f_{\text{TUBE}}$ equals the infinite optical tube length IOTL, the combination is afocal and double telecentric. This is a useful feature when 
using reticles in the eyepiece.


\begin{example}{Microscope}
    As an optical engineer, you are asked to desifn a finite conjugate microscope. The microscope is a combination of the objective lens and 
    the eyepice, which is also composed of a field lens ans an eye lens.
    
    Fortunately, your company has has collected over the years all kinds of parts, uncliding:
    \begin{itemize}[itemsep=0pt,topsep=0pt]
        \item Optical tube of any length ranging from $170-200\;mm$.
        \item Lenses of any diameter between $6-18\;mm$ with any focal length between $5050\;mm$.
    \end{itemize}
    \begin{enumerate}[itemsep=0pt,topsep=0pt,label=\alph*)]
        \item Design a microscope with a visual magnification between $50X-100X$. Pick a value, select the focal length of your eye piece, and determine the working distance in object space of the microscope.
        \item Your microscope will have to operate with a numerical aperture between $0.15-0.25$. Select the stop location such that your microscope operates telecentric 
        in object space, select a nominal value fo NA, and determine the required stop diameter to produce this NA.
        \item Select a lens diameter for your microscope objective and determine the maximum object size $\pm h$ that you can image iwth this instrument while unvignetted.
        If it is not possible with your chosen design, explain why in terms of marignal and chief ray heights.
    \end{enumerate}
    \subsubsection{Solution}
    \begin{enumerate}[itemsep=0pt,topsep=0pt,label=\alph*)]
        \item We pick $m_v=50X$, an $OTL=200\;mm$, and a eye lens with $f_{\text{EYE}}=50\;mm$.
        \begin{align*}
            m_v=\frac{\text{OTL}}{f_{obj}}\frac{250}{f_{\text{EYE}}}=-\frac{z_0'}{z_0}\frac{250}{f_{\text{EYE}}}.
        \end{align*}
        We solve in the above equation for $f_{obj}$:
        \begin{align*}
            f_{obj}=\frac{OTL250}{m_vf_{\text{EYE}}}=\frac{200\cdot250}{50\cdot50}=20\;mm.
        \end{align*}
        The image distance is $z_0'=200+20=220\;mm$. We now solve for $z_0$:
        \begin{align*}
            z_0=WD=-\frac{z_0'}{m_v}\frac{250}{f_{\text{EYE}}}=-22\;mm.
        \end{align*}
        \item The numerical aperture chosen is $NA=0.2$, which we use to get the slope ray and raytrace to the objective lens:
        \begin{align*}
            U=\sin^{-1}(NA)=\sin^{-1}(0.2)=11.54^\circ\longrightarrow u=\tan(11.54^\circ)=0.204.
        \end{align*}
        \begin{align*}
            &y'=y+ut=0+0.204\cdot22=4.49\;mm,\quad u'=u-y\phi=0.204-4.49/20=-0.0205\\
            &y_{\text{stop}}=y'+u'f_{\text{obj}}=4.08\;mm\longrightarrow D_{\text{stop}}=8.16\;mm.
        \end{align*}
        \item We choose a diameter of the lens to be $D_{obj}=10\;mm$. Now, we impose the unvignetting condition to get the maximum height of the object:
        \begin{align*}
            D=2(y+\bar{y})=2(4.49+\bar{y})\longrightarrow\bar{y}=y_{\text{obj height}}=h=\frac{10-2\cdot4.49}{2}=\pm0.51\;mm.
        \end{align*}
    \end{enumerate}
\end{example}