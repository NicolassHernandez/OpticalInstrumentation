\section{Stop and image quality}

\subsection{Diffraction-limited}
Because light is a wave, it does not focus to a perfect point image. Diffraction from the aperture limit the size of the image spot.
For circular aperture, we get an \bfemph{Airy disk} pattern.
\begin{align}
    \text{Airy disk equation}\qquad E=E_0\left[\frac{2J_1(\pi r/\lambda f/\#_W)}{\pi r/\lambda f/\#_W}\right]^2,
\end{align}
where $r$ is the radial coordinate, $J_1$ is the first-kind Bessel function, and $f/\#_W$ the image space working.
\begin{figure}[h!]
    \centering
    \begin{subfigure}{.25\columnwidth}
        \centering
        \includegraphics[width=.7\columnwidth]{PartOne/ChapterTwo/airy.png}
        \caption{Airy disk pattern}
    \end{subfigure}
    \hfill
    \begin{subfigure}{.25\columnwidth}
        \centering
        \includegraphics[width=.5\columnwidth]{PartOne/ChapterTwo/airyprofile.png}
        \caption{Airy disk profile}
    \end{subfigure}
    \hfill
    \begin{subfigure}{.4\columnwidth}
        \centering
        \includegraphics[width=\columnwidth]{PartOne/ChapterTwo/airyzeros}
        \caption{Airy zeros}
    \end{subfigure}
\end{figure}

The diameter of the Airy disk to the first zero is:
\begin{align}
    D=2.44\lambda\cdot f/\#_W.
\end{align}

The \bfemph{Rayleight resolution criterion} states that two point objects can be resolved if the peak of one falls on the first zero on the other:
\begin{align}
    \text{Resolution}=1.22\lambda\cdot f/\#_W.
\end{align}
The \bfemph{angular resolution} is found by dividing by the focal length (or image distance):
\begin{align}
    \text{Angular resolution}=\alpha=1.22\lambda/D_{\text{EP}}. 
\end{align} 

%%
\subsection{Spherical aberration}
\bfemph{Spherical aberration} SA causes the power or focal length of the system to vary with pupil radius. 
For a singlet, the power of the lens increases quadraticallt with pupil radius; the focal length decrease euqdratically.

The image plane can be shifted from paraxial focus to obtain better image quality in the presence of SA. There are different focus criteria as seen in the figure.
\begin{figure}[h!]
    \centering
    \begin{subfigure}{.6\columnwidth}
        \centering
        \includegraphics[width=.8\columnwidth]{PartOne/ChapterTwo/sphericalaberration1.png}
        \caption{Spherical aberration}
    \end{subfigure}
    \hfill
    \begin{subfigure}{.3\columnwidth}
        \centering
        \includegraphics[width=\columnwidth]{PartOne/ChapterTwo/sphericalaberration2.png}
        \caption{Image plane criteria}
    \end{subfigure}
    \caption{Spherical aberration produces different image plane criteria. LA and Ta stand for longitudinal and transverse aberration, respectively.}
\end{figure}

In first-order geometrical optics, each point on the object plane corresponds to a point on the image plane. However, in real life we are not so lucky.
\begin{figure}[h!]
    \centering
    \begin{subfigure}{.6\columnwidth}
        \centering
        \includegraphics[width=.8\columnwidth]{PartOne/ChapterTwo/spotdiagramspherical.png}
        \caption{Spot diagram of focus criteria}
    \end{subfigure}
    \hfill
    \begin{subfigure}{.3\columnwidth}
        \centering
        \includegraphics[width=\columnwidth]{PartOne/ChapterTwo/sphericalfnumber.png}
        \caption{Spot size versus F-number}
    \end{subfigure}
\end{figure}

The spot size scales as the cube of the entrance pupil diameter.
%%
\subsection{Lens bending and minimum spherical aberration}
\bfemph{Bending the lens} of orientation does not change the power, but its aberration do change. The minimum SA occurs when the ray is bent the same at both surfaces.
This is directly analogous to the angle of minimum deviation for prisms. For an object at infinity and $n=1.5$, the corect lens shape is approximately convex-plano.
At finite conjugates, a biconvex lens is used. A trick to further minimize spherical aberration in finite conjugates is to sploit the biconvex into two plano-convex lenses and then flip each of the lenses.

With large apertures, aberrations and depth of field errors are dominant. With small apertures, diffraction dominates with a linear dependence of blut with $f/\#$.

\begin{figure}[h!]
    \centering
    \begin{subfigure}{.35\columnwidth}
        \centering
        \includegraphics[width=.8\columnwidth]{PartOne/ChapterTwo/convexplano.png}
        \caption{Infinite conjugate}
    \end{subfigure}
    \hfill
    \begin{subfigure}{.35\columnwidth}
        \centering
        \includegraphics[width=\columnwidth]{PartOne/ChapterTwo/biconvex}
        \caption{Finite conjugate}
    \end{subfigure}
    \hfill
    \begin{subfigure}{.25\columnwidth}
        \centering
        \includegraphics[width=\columnwidth]{PartOne/ChapterTwo/blurfnumber.png}
        \caption{Blur vs F-number}
    \end{subfigure}
\end{figure}