\section{Stops and pupils}
%
\subsection{Aperture stop}
The \bfemph{aperture stop} is a physical/real surface that limits the cone of light entering and exiting the 
optical system.
\begin{itemize}[itemsep=0pt,topsep=0pt]
    \item The \bfemph{entrance pupil} (EP) is the image of the stop in the object space.
    \item The \bfemph{exit pupil} (XP) is the image of the spot in the image space.
\end{itemize}
\begin{figure}[h!]
    \centering
    \includegraphics[width=.6\columnwidth]{PartOne/ChapterTwo/stop.png}
    \caption{The stop limits the cone of light, and its image in object (image) space creates the 
    entrance (exit) pupil.}
\end{figure}

There is a stop or pupil in each optical space. Intermediate pupils are formed in other spaces.
There are two methods to determine which aperture in a system serves as the system stop:
\begin{enumerate}[itemsep=0pt,topsep=0pt,label=\alph*)]
    \item Image each potential stop into object space. The pupil with the \textbf{smallest} angular size corresponds 
    to the stop. The same can be done in image space.
    \begin{figure}[h!]
        \centering
        \includegraphics[width=.5\columnwidth]{PartOne/ChapterTwo/smallestcone.png}
        \caption{The smallest angular size corrsponds to the stop in object space. Same for image space.}
    \end{figure}
    \item Trace a ray through the system from the axial object point with arbitrary initial angle. At each potential stop,
    determine the ratio of the aperture radio $a_k$ to the ray height at that surface $\tilde{y}_k$.
    \begin{align}
        \text{Aperture stop}=\min\left\{\left|\frac{a_k}{\tilde{y}_k}\right|\right\}.
    \end{align}
    \begin{figure}[h!]
        \centering
        \includegraphics[width=.5\columnwidth]{PartOne/ChapterTwo/minimumslope.png}
        \caption{The minimum slope value corresponds to the aperture stop.}
    \end{figure}
\end{enumerate}

The pupils are the image of the stop and do not change position or size with an off-axis object. Intermediate 
pupils are formed in each optical space for multi-element systems. If there are $N$ elements, there are $N+1$ pupils
(including the stop).

\begin{emphasizer}
    When designing a system, is usually critical that the stop surface does not change over a range of possible object positions
    that the system will be used with.
\end{emphasizer}

%%
\subsection{Marginal and Chief rays}
Rays confined to the $yz$-plane are called \bfemph{meridional rays}. There are two special meridional 
rays that define properties of the object, images and pupils:
\begin{itemize}[itemsep=0pt,topsep=0pt]
    \item The \bfemph{marginal ray} travels from the base of the object to the edge of EP . It defines image locations and pupil sizes. 
    \item The \bfemph{chief ray} travels from the edge of the object to the center of the EP. It defines image heights and pupil locations.
\end{itemize}
\begin{figure}[h!]
    \centering
    \includegraphics[width=.6\columnwidth]{PartOne/ChapterTwo/meridionalrays.png}
    \hfill
    \includegraphics[width=.18\columnwidth]{PartOne/ChapterTwo/heightsmedirional.png}
    \caption{The}
\end{figure}

The heights of the marginal ray and the chief ray can be evaluated at any $z$ in any optical space. When the marginal ray crosses the axis, an image is located, and the size of the image is given by the chief ray height.
Whenever the chief ray crosses the axis, a pupil or stop is located, and the pupil radius is given by the marginal ray height. Intermediate images and pupils 
are often virtual.

%%
\subsection{Pupil locations}
%
\subsubsection{By raytrace}
Once you know which surface is the stop, you have the information to determine the location of EP and XP. The \bfemph{pupil locations} can be found by tracing a paraxial ray starting at the center of the stop and is 
back/forward propagated. The intersections of this ray with the axis in object and image space determine the locations of EP and XP.
\begin{figure}[h!]
    \centering
    \includegraphics[width=.6\columnwidth]{PartOne/ChapterTwo/pupillocations.png}
    \caption{The}
\end{figure}

This ray become the chief ray when it is scaled to the object or image size. The marginal ray gives the pupil sizes.
\begin{example}{Pupil location by paraxial raytrace}
\begin{figure}[h!]
    \centering
    \includegraphics[width=.3\columnwidth]{PartOne/ChapterTwo/example1_pupillocation.png}
    \includegraphics[width=.2\columnwidth]{PartOne/ChapterTwo/example1info_pupillocation.png}
\end{figure} 
\subsubsection{Solution}
The stop is a real object for the formation of both EP and XP. There is a ray that has a height of $0$ at the EP, stop and XP. We first set $y=0$ at the stop, and then 
with arbitrary angle 
\begin{itemize}[itemsep=0pt,topsep=0pt]
    \item[EP] we set $y=0$ for the EP and solve for the distance.
    \item[XP] we set $y=0$ for the XP and solve for the distance.
\end{itemize}
We used a potential chief to find pupil locations. For the pupil sizes, we find the true marginal ray scaling a potential maginal ray. Remember that 
the chief ray was for pupil locations, now with the marginal ray we find the pupil sizes. We can also use it to find the image location.
\begin{figure}[h!]
    \centering
    \includegraphics[width=.3\columnwidth]{PartOne/ChapterTwo/example1dev1_pupillocation.png}
    \includegraphics[width=.5\columnwidth]{PartOne/ChapterTwo/example1dev2_pupillocation.png}
\end{figure} 

Finally, the height of the EP is $13.33\;mm$, the stop $10\;mm$, the XP $15\;mm$.
\end{example}
%
\subsubsection{By Gaussian imagery}
We treat each group independently, considering the stop as our object propagating in the direction of te given group. For EP, the object propagates from right 
to left, so we flip the sign of the refractive index (as in reflection).
\begin{figure}[h!]
    \centering
    \includegraphics[width=.4\columnwidth]{PartOne/ChapterTwo/pupillocation_gaussrear.png}
    \hfill
    \includegraphics[width=.4\columnwidth]{PartOne/ChapterTwo/pupillocation_gaussfront.png}
    \caption{}
\end{figure}
\begin{align}
    \text{For XP}&\qquad\frac{n'}{z'_{XP}}=\frac{n}{Z_{stop}}+\frac{1}{f_{RG}},\quad m_{XP}=\frac{z'_{XP}}{z_{stop}},\quad D_{XP}=|m_{XP}|D_{stop}\\
    \text{For EP}&\qquad\frac{n'}{Z'_{EP}}=\frac{n}{Z_{stop}}+\frac{1}{f_{FG}},\quad m_{EP}=\frac{z'_{EP}}{z_{stop}},\quad D_{EP}=|m_{EP}|D_{stop}\quad(n=n'=-1)
\end{align}

\begin{example}{Pupil locations by Gaussian imagery}
    \begin{figure}[h!]
        \centering
        \includegraphics[width=.7\columnwidth]{PartOne/ChapterTwo/example2_pupillocation.png}
    \end{figure}
\end{example}


\begin{emphasizer}[EP,STOP,XP are invariant to object location]
    Changing the object location does not change the position of the EP, stop, and XP.
\end{emphasizer}
%%
\subsection{Lagrange invariant}
The linearity of paraxial optics provides a relationship between the heights and angles of any two ray s propagating through the system. The \bfemph{Lagrande invariant}
$\Xi$ is formed with the paraxial marginal and chief rays:
\begin{align}
    \text{Lagrange invariant}\qquad\highlight{\Xi=n\bar{u}y-nu\bar{y}=\bar{\omega}y-\omega\bar{y}}.
\end{align}
It is invariant for refraction and transferm and it can be evaluated at any $z$ in any optical space. The Lagrange invariant is particularly simple at 
images or objects ($y=0$) and pupils ($\bar{y}=0$):
\begin{align}
    \text{Image/Object}&\qquad\Xi=-nu\bar{y}=-\omega\bar{y}\\
    \text{Pupils}&\qquad\Xi=n\bar{u}y=\bar{\omega}y
\end{align}
If two rays other than the marginal and chief are used, the more general \bfemph{optical invariant} $I$ is formed.

Given two rays, a third ray can be formed as a linear combination of the two rays. The coefficients are the ratios of the pair-wise invariantas of the values for the three rays at some 
initial $z$. The expressions are valid for any $z$:
\begin{align}
    &y_3=Ay_1+By_2,\quad u_3=Au_1+Bu_2\\
    &A=I_{32}/I_{12},\quad B=I_{13}/I_{12},\quad I_{ij}=nu_iy_j-nu_jy_i.
\end{align}
Changing the Lagrange invariant of a system \textbf{scales} the optical system. Doubling the invariant while maintaining the same object and image sizes and pupil diameters haves all of the 
axial distances (and the focal length).

The \bfemph{throughout}, \bfemph{etedue} of \bfemph{$A\Omega$ product} in \bfemph{radiometry} and \bfemph{radiative transfer} are relalted to the square of the Lagrange invariant:
\begin{align}
    n^2A\Omega=\pi^2\Xi^2.
\end{align}
%%
\subsection{Field of view}
We revisit again the concept of FOV but know using the EP and XP.
\begin{itemize}[itemsep=0pt,topsep=0pt]
    \item \textbf{Field of view FOV} diameter of the object/image. 
    \item \textbf{Half field of view HFOV} radius of the object/image.
\end{itemize}
\begin{figure}[h!]
    \centering
    \includegraphics[width=.8\columnwidth]{PartOne/ChapterTwo/fov.png}
\end{figure}
%%
\subsection{Numerical aperture and F-number}
In an optical space of index $n_k$, the \bfemph{numerical aperture} $N_A$ describes the axial cone of light in terms of the real marginal angle $U_k$:
\begin{align}
    \text{Numerical aperture}\qquad\highlight{NA=n_k|sinU_k|\approx n_k|u_k|}.
\end{align}
The \bfemph{F-number} $f/\#$ describes the image-space cone of light for an object \textbf{at infinity}:
\begin{align}
    \text{F-number}\qquad\highlight{f/\#=\frac{f_E}{D_{EP}}}.
\end{align}
While the $f/\#$ is an image-space, infinite-conjugate measure, the approximate relationship between NA and $f/\#$ allows and $f/\#$ to be defined for other 
optical spaces and conjugates. As a result, an $f/\#$ can be defined for any cone of light. This $f/\#$ is called \bfemph{working F-number} $f/\#_W$.
This previous relationship becomes a definition
\begin{align}
    \text{Working F-number}\qquad f/\#_W=\frac{1}{2NA}\approx\frac{1}{2n|u|}=(1-m)f/\#.
\end{align}
Fast optical system have small numeric values for the $f/\#$. Most lenses with adjustable stops have $f/\#$ of \bfemph{f-stops} labeled in increments of $\sqrt{2}$.
The usual progression is:
\begin{align*}
    f/1.4,\quad f/2,\quad f/2.8,\quad f/4,\quad f/5.6,\quad f/8,\quad f/11,\quad f/16,\quad f/22,\quad\text{etc}.
\end{align*}
Each stop changes the area of the EP (light colelction ability) by a factor of $2$.
%
\begin{emphasizer}
    The Lagrange invariant relates the magnification between two pupils to the chief ray angles at the pupils.
    \begin{align}
        \Xi=n\bar{u}y_{pupil}=n'\bar{u}'y'_{pupil},\quad m_{pupil}=\frac{y'_{pupil}}{y_{pupil}}=\frac{n\bar{u}}{n'\bar{u}'}=\frac{\bar{\omega}}{\bar{\omega}'}.
    \end{align}
\end{emphasizer}
%
\subsubsection{Use of working F-number (left)}
The most common use of the working F-number is to describe the image-forming cone for a finite conjugate optical system. This is the cone formed 
by the XP and the axial image point.



\begin{example}{Determination of stop and pupil}
Determine the location and size of the pupils for th efollowing the system in air.
\begin{figure}[h!]
    \centering
    \includegraphics[width=.6\columnwidth]{PartOne/ChapterTwo/example_stophw6.png}
\end{figure}
\subsubsection{Solution}
\begin{enumerate}[itemsep=0pt,topsep=0pt,label=\alph*)]
  \item We trace the chief ray denoted as CR, and a potential marginal ray MR with unitary height at the stop.
  \begin{table}[h!]
    \centering
    \resizebox{\columnwidth}{!}{%
    \begin{tabular}{ll|l|l|l|l|l|l|l|l|l|l|l}
      &&\shortstack{Object\\space}&EP&&$L_1$&&Stop&&$L_2$&&XP&\shortstack{Image\\space}\\
      \hline
      &$C/R/f$&&&&$250$&&&&$400$&&&\\
      &$t$&&&$z_{EP}=-62.5$&&$50$&&$70$&&$z_{XP}=-84.8$&&\\
      &$n$&1&1&1&1&1&1&1&1&1&1&1\\
      \hline 
      &$-\phi$&&&&$-0.004$&&&&$-0.0025$&&&\\
      &$t/n$&&&$\tau_{EP}=-62.5$&&$50$&&$70$&&$\tau_{XP}=-84.8$&&\\
      \hline 
      &$y$&&$0$&&$-5$&&\textcolor{red}{$0$}&&$7$&&$0$&\\
      CR&$nu$&&&$0.08$&&\textcolor{red}{$0.1$}&&\textcolor{red}{$0.1$}&&$0.0825$&&\\
      &$u$&&&$0.08$&&$0.1$&&$0.1$&&$0.0825$&&\\
      \hline
      &$y$&&$R_{EP}=1.25$&&$1$&&\textcolor{red}{$1$}&&$1$&&$R_{XP}=1.21$&\\
      MR&$nu$&&&$0.004$&&\textcolor{red}{$0$}&&\textcolor{red}{$0$}&&$-0.0025$&&\\
      &$u$&&&&&$0$&&$0$&&&&
    \end{tabular}}
    \caption{Raytrace, with CR=Chief ray, MR=Marginal ray.}
  \end{table}

  Due to the diameter of the stop is $R_{stop}=10\;mm$, we scale the potential marginal ray to give the true marginal ray and therefore obtain the radius of 
  the pupils:
  \begin{align*}
    \begin{array}{l}
    R_{EP}=(10)(1.25)=12.5\;mm\\
    R_{XP}=(10)(1.21)=12.1\;mm
    \end{array}\Longrightarrow
    \begin{array}{l}
    D_{EP}=2R_{EP}=25.0\;mm\\
    D_{XP}=2R_{XP}=24.2\;mm
    \end{array}
  \end{align*}

  \item For Gaussian imagery, we see the stop as the object for the front group and rear group.
  For the EP, we have a backward propagation that is maganed with the flip of the sign in the refrractive indices.
  \begin{align*}
    \frac{-1}{z_{EP}}&=\frac{-1}{Z_{\text{stop}}}+\frac{1}{250}\longrightarrow z_{EP}=62.5\;mm.
  \end{align*}
  This entrance pupil is to the right of the lens $L_1$. The magnification is:
  \begin{align*}
    m_{EP}=\frac{z_{EP}}{z_{\text{stop}}}=\frac{R_{EP}}{R_{\text{stop}}}=-1.25.
  \end{align*}
  The diameter of the entrance pupil is therefore:
  \begin{align*}
    D_{EP}=2R_{EP}=2[|m_{EP}|R_{\text{stop}}]=25\;mm.
  \end{align*}
  For the rear group, we have analogously:
  \begin{align*}
    \frac{1}{z_{XP}}=\frac{1}{Z_{\text{stop}}}+\frac{1}{400}\longrightarrow z_{XP}=-84.848\;mm.
  \end{align*}
  The exit pupil is then to the left of the lens $L_2$. The magnification in this case is 
  \begin{align*}
    m_{XP}=\frac{z_{XP}}{z_{\text{stop}}}=\frac{R_{XP}}{R_{\text{stop}}}=1.21.
  \end{align*}
  The diameter of the exit pupil is:
  \begin{align*}
    D_{XP}=2R_{XP}=2[|m_{XP}|R_{\text{stop}}]=24.2\;mm.
  \end{align*}
  The illustration of each case is illustrated in the figure \ref{gaussianimagery}.

  \begin{figure}[h!]
    \centering
    \begin{subfigure}{.45\columnwidth}
      \centering
      \includegraphics[width=.8\columnwidth]{PartOne/ChapterTwo/problem1_a.png}
      \caption{EP for front group}
      \label{frontgroup}
    \end{subfigure}
    \hfill
    \begin{subfigure}{.45\columnwidth}
      \centering
      \includegraphics[width=.8\columnwidth]{PartOne/ChapterTwo/problem1_a.png}
      \caption{XP for rear group}
      \label{reargroup}
    \end{subfigure}
    \caption{With Gaussian imagery, the computation of the pupils is based on the stop as the object of two optical systems.}
    \label{gaussianimagery}
  \end{figure}
  Using either method, the result is the same and is shown in figure \ref{problem1}
  \begin{figure}
    \centering
    \includegraphics[width=.5\columnwidth]{PartOne/ChapterTwo/problem1.png}
    \caption{Illustration of the stop and pupil in the optical system.}
    \label{problem1}
  \end{figure}
\end{enumerate}
\end{example}

