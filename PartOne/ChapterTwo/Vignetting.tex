\section{Vignetting}
%%
\subsection{Ray bundles}
The \bfemph{ray bundle} for an \textbf{on-axis} object is a rotationally symmetric spindle made up of section of right circular cones.
Each cone section if bounded by the pupil and the object/image in that optical space.
\begin{figure}[h!]
    \centering
    \includegraphics[width=.45\columnwidth]{PartOne/ChapterTwo/raybundle1.png}
    \hfill
    \includegraphics[width=.45\columnwidth]{PartOne/ChapterTwo/raybundle2.png}
\end{figure}
At any $z$, the cross section of the bundle is circular, and the radius of the bundle is the marginal ray value.

For an \textbf{off-axis} object point, the ray bundle skews, and is comprised of section of skew circular cones wwhich are still defined by the same elements.
The cross section of the ray bundle at any $z$ remains circular with a radius equal to the radius of the axial bundle. The off-axis bundle is 
centered about the chief ray height.

The maximum radial extent of the ray bundle at any $z$ is:
\begin{align}
    \text{Maximum radial extent}\qquad|y_{max}|=|y|+|\bar{y}|.
\end{align}
%%
\subsection{Vignetting}
The \bfemph{vignetting} occurs when other apertures in the system (others than the stop) block a proportion of an off-axis ray bundle.
For no vignetting, each aperture radius $a$ must equal or exceed the maximum height of the ray bundle at the aperture.
\begin{figure}[h!]
    \centering
    \includegraphics[width=\columnwidth]{PartOne/ChapterTwo/vignetting.png}
    \caption{The ray bundle is clipped and the beam is no longer circular.}
\end{figure}

The maximum FOV supported by the system occurs when an aperture completely blocks the ray bundle from the object point.

We can have three conditions of vignetting, depending on the proportion of clip of the light beam.
\begin{figure}[h!]
    \centering
    \includegraphics[width=\columnwidth]{PartOne/ChapterTwo/vignettingcases.png}
\end{figure}

The vignetting conditions are used in two different manners:
\begin{itemize}[itemsep=0pt,topsep=0pt]
    \item For a given set of apertures, the FOV that the system will suppport with a prescribed amount of vignetting can be determined. A different chief ray defined each FOV.
    \item For a given FOC and vignetting condition, the required aperture diameters can be determined.
\end{itemize}

A system with vignetting will have an image that has full irradiance or brightness out to a radius corresponding to the unvignetted FOV limit.
The irradiance will then begin to fall off, going to about halr at the half-vignetted FOV, and decreasing to zero at the fully vignetting FOV.
This fully vignetted FOV is the absolute maximum possible.

The diameter of the aperture stop is very important design parameter for an optical system as it controls five separate performance aspects of the 
system:
\begin{itemize}[itemsep=0pt,topsep=0pt]
    \item The system FOV determined by vignetting.
    \item The radiometric or photometric speed of the system or its light collection ability.
    \item The depth of focus and depth of field of the system.
    \item The amount of aberrations degrading image quality.
    \item The diffraction-based performance of the system.
\end{itemize}


\begin{example}{}
    \begin{figure}[h!]
        \centering
        \includegraphics[width=.45\columnwidth]{PartOne/ChapterTwo/vignetting_example1.png}
        \includegraphics[width=.45\columnwidth]{PartOne/ChapterTwo/vignetting_example1b.png}
        \includegraphics[width=.45\columnwidth]{PartOne/ChapterTwo/vignetting_example1c.png}
    \end{figure}
\end{example}

\begin{example}{Vignetting with paraxial raytrace}
    \begin{figure}[h!]
        \centering
        \includegraphics[width=.45\columnwidth]{PartOne/ChapterTwo/vignetting_example2a.png}
        \includegraphics[width=.45\columnwidth]{PartOne/ChapterTwo/vignetting_example2b.png}
        \includegraphics[width=.45\columnwidth]{PartOne/ChapterTwo/vignetting_example2c.png}
        \includegraphics[width=.45\columnwidth]{PartOne/ChapterTwo/vignetting_example2d.png}
        \includegraphics[width=.45\columnwidth]{PartOne/ChapterTwo/vignetting_example2e.png}
        \includegraphics[width=.45\columnwidth]{PartOne/ChapterTwo/vignetting_example2f.png}
    \end{figure} 
\end{example}


In general,
\begin{emphasizer}[Key points in solving problems]
    \begin{itemize}[itemsep=0pt,topsep=0pt]
        \item Trace the potential chief ray (CR) to know the locations of the pupils (and image size).
        \item Trace the potential marginal ray (MR) to determine image location and pupil sizes.
        \item If the MR comes parallel, then it can be used to obtain the first-order properties.
        \item The F-number gives us the real size of EP so we can scale the MR.
        \item The image size allows us to get the real CR.
        \item The HFOV is determined with the incident angle $\bar{u}$ at EP in the real CR: $\text{HFOV}=\tan^{-1}\bar{u}$.
        \item The vignetting is found by looking at $y,\bar{y}$ in the real MR and CR and applying the criteria.
        \item We can arbitraryly define a dummy surface to our convenience.
    \end{itemize}
\end{emphasizer}

%%
\subsection{Dummy surfaces}
In a raytrace, a zero-power surface can be inserted at any location to examine the ray properties.

An example of its application is the following Cassegrain objective, where we require to find the size of the hole.
For that, we place a dummy surface \textbf{at the hole}.
\begin{figure}[h!]
    \centering
    \includegraphics[width=.8\columnwidth]{PartOne/ChapterTwo/dummysurfaces.png}
\end{figure}